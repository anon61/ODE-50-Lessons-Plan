\documentclass[12pt]{article}
\usepackage{amsmath, amssymb, amsthm, tikz, pgfplots}
\usepackage{geometry, enumitem, mdframed, array, xcolor}
\geometry{margin=1in}

% Custom environments
\newtheorem{definition}{Definition}
\newtheorem{theorem}{Theorem}
\newtheorem{lemma}{Lemma}
\newtheorem{method}{Method}
\newtheorem{example}{Example}
\newmdenv[linecolor=blue,linewidth=2pt]{keypoint}
\newmdenv[linecolor=red,linewidth=2pt]{warning}
\newmdenv[linecolor=green,linewidth=2pt]{insight}
\newmdenv[linecolor=purple,linewidth=2pt]{examtip}
\newmdenv[linecolor=orange,linewidth=2pt]{formula}

\title{Lesson 44: Repeated Roots - Why We Get $t$^{k}$$ Terms}
\author{ODE 1 with Prof. Adi Ditkowski}
\date{Tel Aviv University}

\begin{document}

\maketitle

\section{Introduction: The Problem of Repeated Roots}

When the characteristic equation of a linear ODE with constant coefficients has repeated roots, the naive approach of using only exponential solutions fails to generate enough linearly independent solutions.

\begin{definition}[Multiplicity of a Root]
If the characteristic polynomial $p(r)$ can be factored as $p(r) = (r - r$_{0}$)$^{m}$ q(r)$ where $q(r$_{0}$) \neq 0$, then $r$_{0}$$ is a root of multiplicity $m$.
\end{definition}

\begin{theorem}[Dimension of Solution Space]
An $n$-th order linear homogeneous ODE has an $n$-dimensional solution space. Therefore, we need exactly $n$ linearly independent solutions.
\end{theorem}

\section{Failure of Simple Exponentials}

\begin{example}[The Problem]
Consider $y'' - 4y' + 4y = 0$ with characteristic equation $(r-2)$^{2}$ = 0$.

If we only use $y = e^{2t}$, we have just one solution. But we need two linearly independent solutions for this second-order equation.
\end{example}

\begin{keypoint}
When a root $r$ has multiplicity $m > 1$, the solution $e^{rt}$ alone cannot span the $m$-dimensional subspace of solutions corresponding to that root.
\end{keypoint}

\section{Derivation via Reduction of Order}

\begin{method}[Reduction of Order for Repeated Roots]
Given one solution $y$_{1}$ = e^{rt}$ where $r$ is a repeated root, we seek a second solution of the form:
$$y$_{2}$ = v(t) y$_{1}$ = v(t) e^{rt}$$
\end{method}

\begin{theorem}[Second Solution for Double Root]
If $r$ is a double root of the characteristic equation for $y'' + py' + qy = 0$, and $y$_{1}$ = e^{rt}$ is one solution, then $y$_{2}$ = te^{rt}$ is a second, linearly independent solution.
\end{theorem}

\begin{proof}
Let $y$_{2}$ = v(t)e^{rt}$. Then:
\begin{align}
y_{2}' &= v'e^{rt} + rve^{rt} \\
y_{2}'' &= v''e^{rt} + 2rv'e^{rt} + r^{2ve}^{rt}
\end{align}

Substituting into the differential equation:
$$v''e^{rt} + 2rv'e^{rt} + r$^{2ve}$^{rt} + p(v'e^{rt} + rve^{rt}) + qve^{rt} = 0$$

Factoring out $e^{rt}$:
$$v'' + v'(2r + p) + v(r$^{2}$ + pr + q) = 0$$

Since $r$ is a double root of $\lambda$^{2}$ + p\lambda + q = 0$:
\begin{enumerate}
\item $r$^{2}$ + pr + q = 0$ (r is a root)
\item $2r + p = 0$ (derivative of characteristic polynomial at r equals 0)
\end{enumerate}

Therefore: $v'' = 0$, giving $v = c$_{1t}$ + c$_{2}$$.

Taking $c$_{1}$ = 1, c$_{2}$ = 0$ yields $y$_{2}$ = te^{rt}$.
\end{proof}

\section{The Differential Operator Approach}

\begin{insight}
The differential equation $(D - r)$^{m}$[y] = 0$ naturally produces solutions $t$^{k}$ e^{rt}$ for $k = 0, 1, \ldots, m-1$.
\end{insight}

\begin{theorem}[Operator Factorization]
If the characteristic polynomial factors as $p(\lambda) = (\lambda - r)$^{m}$$, then the differential operator factors as:
$$L = (D - r)$^{m}$$$
where $D = \frac{d}{dt}$.
\end{theorem}

\begin{lemma}[Kernel of $(D - r)$^{m}$$]
The kernel (null space) of the operator $(D - r)$^{m}$$ is:
$$\text{ker}((D - r)$^{m}$) = \text{span}\{e^{rt}, te^{rt}, t$^{2e}$^{rt}, \ldots, t^{m-1}e^{rt}\}$$
\end{lemma}

\begin{proof}[Proof by Induction]
Base case ($m = 1$): $(D - r)[y] = 0 \Rightarrow y = ce^{rt}$.

Inductive step: Assume true for $m = k$. For $m = k + 1$:
If $(D - r)^{k+1}[y] = 0$, let $w = (D - r)[y]$.
Then $(D - r)$^{k}$[w] = 0$, so $w = p$_{k}$(t)e^{rt}$ where $p$_{k}$$ is a polynomial of degree $\leq k-1$.

Solving $(D - r)[y] = p$_{k}$(t)e^{rt}$:
Using integrating factor $e^{-rt}$: $\frac{d}{dt}[e^{-rt}y] = p$_{k}$(t)$

Integrating: $e^{-rt}y = P_{k+1}(t) + C$ where $P_{k+1}$ has degree $\leq k$.

Therefore: $y = [P_{k+1}(t) + C]e^{rt} = p_{k+1}(t)e^{rt}$ where $\deg(p_{k+1}) \leq k$.
\end{proof}

\section{The Limit Perspective}

\begin{theorem}[Repeated Roots as Limits]
The solution $te^{rt}$ can be understood as the limit of solutions for nearby distinct roots.
\end{theorem}

\begin{proof}[Heuristic Derivation]
Consider two roots $r$_{1}$ = r$ and $r$_{2}$ = r + \epsilon$ with solutions:
$$y = c$_{1}$ e^{rt} + c$_{2}$ e^{(r+\epsilon)t}$$

Rewriting:
$$y = c$_{1}$ e^{rt} + c$_{2}$ e^{rt} e^{\epsilon t}$$

For small $\epsilon$: $e^{\epsilon t} \approx 1 + \epsilon t$

Thus:
$$y \approx (c$_{1}$ + c$_{2}$)e^{rt} + c$_{2}$ \epsilon t e^{rt}$$

As $\epsilon \to 0$, setting $A = c$_{1}$ + c$_{2}$$ and $B = \lim_{\epsilon \to 0} c$_{2}$ \epsilon$:
$$y = Ae^{rt} + Bte^{rt}$$
\end{proof}

\section{Higher Multiplicities}

\begin{theorem}[General Multiplicity Case]
If $r$ is a root of multiplicity $m$, then the $m$ linearly independent solutions are:
$$e^{rt}, \, te^{rt}, \, t$^{2e}$^{rt}, \, \ldots, \, t^{m-1}e^{rt}$$
\end{theorem}

\begin{formula}
For a root $r$ of multiplicity $m$:
$$\text{Solutions} = \{t$^{k}$ e^{rt} : k = 0, 1, 2, \ldots, m-1\}$$
\end{formula}

\section{Complex Repeated Roots}

\begin{theorem}[Complex Repeated Roots]
If $\alpha \pm i\beta$ are complex conjugate roots each of multiplicity $m$, the $2m$ real-valued linearly independent solutions are:
$$t$^{k}$ e^{\alpha t} \cos(\beta t), \, t$^{k}$ e^{\alpha t} \sin(\beta t) \quad \text{for } k = 0, 1, \ldots, m-1$$
\end{theorem}

\begin{warning}
For complex repeated roots, both the sine AND cosine terms need all powers of $t$ up to $t^{m-1}$.
\end{warning}

\section{The Wronskian for Repeated Roots}

\begin{theorem}[Linear Independence via Wronskian]
The functions $\{t$^{k}$ e^{rt} : k = 0, 1, \ldots, m-1\}$ are linearly independent.
\end{theorem}

\begin{proof}[Wronskian Calculation]
For simplicity, consider $m = 2$ with solutions $e^{rt}$ and $te^{rt}$:

$$W(t) = \begin{vmatrix}
e^{rt} & te^{rt} \\
re^{rt} & (1 + rt)e^{rt}
\end{vmatrix}$$

$$W(t) = e^{rt} \cdot (1 + rt)e^{rt} - te^{rt} \cdot re^{rt} = e^{2rt}(1 + rt - rt) = e^{2rt} \neq 0$$

For general $m$, the Wronskian is:
$$W(t) = e^{mrt} \cdot \prod_{0 \leq i < j < m} (j - i) \neq 0$$
\end{proof}

\section{Connection to Jordan Normal Form}

\begin{insight}
The appearance of $t$^{k}$$ terms in solutions corresponds exactly to the Jordan block structure of the companion matrix.
\end{insight}

\begin{theorem}[Jordan Form and Solutions]
For a system $\mathbf{x}' = A\mathbf{x}$ where $A$ has a Jordan block:
$$J = \begin{pmatrix}
\lambda & 1 & 0 & \cdots & 0 \\
0 & \lambda & 1 & \cdots & 0 \\
\vdots & \vdots & \ddots & \ddots & \vdots \\
0 & 0 & 0 & \lambda & 1 \\
0 & 0 & 0 & 0 & \lambda
\end{pmatrix}$$

The matrix exponential is:
$$e^{Jt} = e^{\lambda t} \begin{pmatrix}
1 & t & \frac{t$^{2}$}{2!} & \cdots & \frac{t^{m-1}}{(m-1)!} \\
0 & 1 & t & \cdots & \frac{t^{m-2}}{(m-2)!} \\
\vdots & \vdots & \ddots & \ddots & \vdots \\
0 & 0 & 0 & 1 & t \\
0 & 0 & 0 & 0 & 1
\end{pmatrix}$$
\end{theorem}

\section{Algorithm for Repeated Roots}

\begin{method}[Complete Solution Construction]
\begin{enumerate}
\item Factor the characteristic polynomial completely.
\item For each distinct root $r$_{i}$$ with multiplicity $m$_{i}$$:
    \begin{itemize}
    \item If $r$_{i}$$ is real: Include solutions $t$^{k}$ e^{r$_{i}$ t}$ for $k = 0, 1, \ldots, m$_{i}$ - 1$
    \item If $r$_{i}$ = \alpha + i\beta$ is complex (with conjugate $\alpha - i\beta$):
        Include solutions $t$^{k}$ e^{\alpha t} \cos(\beta t)$ and $t$^{k}$ e^{\alpha t} \sin(\beta t)$ for $k = 0, 1, \ldots, m$_{i}$ - 1$
    \end{itemize}
\item Form the general solution as a linear combination of all these functions.
\end{enumerate}
\end{method}

\section{Worked Examples}

\begin{example}[Triple Root]
Solve: $y''' - 3y'' + 3y' - y = 0$

\textbf{Solution:}
Characteristic equation: $r$^{3}$ - 3r$^{2}$ + 3r - 1 = 0$

Recognize this as $(r-1)$^{3}$ = 0$ (expand to verify).

Root: $r = 1$ with multiplicity 3.

Solutions: $e$^{t}$, te$^{t}$, t$^{2e}$^t$

General solution: $y(t) = (c$_{1}$ + c$_{2}$ t + c$_{3}$ t$^{2}$)e$^{t}$$
\end{example}

\begin{example}[Complex Double Roots]
Solve: $y^{(4)} + 8y'' + 16y = 0$

\textbf{Solution:}
Characteristic equation: $r$^{4}$ + 8r$^{2}$ + 16 = 0$

This is $(r$^{2}$ + 4)$^{2}$ = 0$, giving $(r - 2i)$^{2}$(r + 2i)$^{2}$ = 0$

Roots: $\pm 2i$, each with multiplicity 2.

Real solutions:
\begin{itemize}
\item From $2i$ (mult. 2): $\cos(2t), t\cos(2t), \sin(2t), t\sin(2t)$
\end{itemize}

General solution: $y(t) = (c$_{1}$ + c$_{2}$ t)\cos(2t) + (c$_{3}$ + c$_{4}$ t)\sin(2t)$
\end{example}

\begin{examtip}
Prof. Ditkowski often presents the characteristic polynomial already factored. Don't waste time trying to factor what's already factored - just identify multiplicities and write solutions systematically.
\end{examtip}

\end{document}