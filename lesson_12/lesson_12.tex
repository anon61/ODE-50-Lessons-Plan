\documentclass[12pt]{article}
\usepackage{amsmath, amssymb, amsthm, tikz, pgfplots}
\usepackage{geometry, enumitem, mdframed, array, xcolor}
\geometry{margin=1in}

% Custom environments
\newtheorem{definition}{Definition}
\newtheorem{theorem}{Theorem}
\newtheorem{method}{Method}
\newtheorem{example}{Example}
\newmdenv[linecolor=blue,linewidth=2pt]{keypoint}
\newmdenv[linecolor=red,linewidth=2pt]{warning}
\newmdenv[linecolor=green,linewidth=2pt]{insight}
\newmdenv[linecolor=purple,linewidth=2pt]{examtip}

\title{Separable Equations: Complete Method and Analysis}
\author{ODE 1 - Prof. Adi Ditkowski}
\date{Lesson 12}

\begin{document}
\maketitle

\section{Definition and Recognition}

\begin{definition}[Separable Equation]
A first-order ODE is \textbf{separable} if it can be written in the form:
$$\frac{dy}{dx} = g(x)h(y)$$
where $g(x)$ is a function of $x$ only and $h(y)$ is a function of $y$ only.
\end{definition}

\begin{keypoint}
Alternative forms that are separable:
\begin{itemize}
    \item $M(x)N(y)dx + P(x)Q(y)dy = 0$
    \item $\frac{dy}{dx} = \frac{f(x)}{g(y)}$
    \item Autonomous: $\frac{dy}{dx} = f(y)$
\end{itemize}
\end{keypoint}

\section{Solution Method}

\begin{method}[Separation of Variables]
\begin{enumerate}
    \item Write in standard form: $\frac{dy}{dx} = g(x)h(y)$
    \item Separate variables: $\frac{dy}{h(y)} = g(x)dx$ \quad (assuming $h(y) \neq 0$)
    \item Integrate both sides: $\int \frac{dy}{h(y)} = \int g(x)dx$
    \item Include integration constant: $H(y) = G(x) + C$
    \item Solve for $y$ if possible (explicit solution)
    \item Check for singular solutions where $h(y) = 0$
\end{enumerate}
\end{method}

\begin{warning}
Division by $h(y)$ loses solutions where $h(y) = 0$. These must be checked separately!
\end{warning}

\section{Complete Solution Structure}

\begin{theorem}[General and Singular Solutions]
For $\frac{dy}{dx} = g(x)h(y)$:
\begin{itemize}
    \item \textbf{General solution}: Obtained by separation (family of curves)
    \item \textbf{Singular solutions}: Constants $y = c$ where $h(c) = 0$
    \item \textbf{Complete solution}: Union of general and singular solutions
\end{itemize}
\end{theorem}

\begin{example}[Finding All Solutions]
Solve: $\frac{dy}{dx} = 2x(y^2 - 1)$

\textbf{Step 1:} Separate (assuming $y^2 - 1 \neq 0$):
$$\frac{dy}{y^2 - 1} = 2x\,dx$$

\textbf{Step 2:} Use partial fractions:
$$\frac{1}{y^2-1} = \frac{1}{2}\left(\frac{1}{y-1} - \frac{1}{y+1}\right)$$

\textbf{Step 3:} Integrate:
$$\frac{1}{2}\ln\left|\frac{y-1}{y+1}\right| = x^2 + C$$

\textbf{Step 4:} General solution:
$$\frac{y-1}{y+1} = Ae^{2x^2}$$ where $A = \pm e^{2C}$

\textbf{Step 5:} Check $y^2 - 1 = 0$:
\begin{itemize}
    \item If $y = 1$: $\frac{dy}{dx} = 0 = 2x(1-1)$ \checkmark
    \item If $y = -1$: $\frac{dy}{dx} = 0 = 2x(1-1)$ \checkmark
\end{itemize}

\textbf{Complete solution:} $\frac{y-1}{y+1} = Ae^{2x^2}$ and $y = \pm 1$
\end{example}

\section{Autonomous Equations}

\begin{definition}[Autonomous Equation]
An ODE $\frac{dy}{dx} = f(y)$ depending only on $y$ is \textbf{autonomous}.
\end{definition}

\begin{method}[Solving Autonomous Equations]
For $\frac{dy}{dx} = f(y)$:
\begin{enumerate}
    \item Find equilibria: solve $f(y) = 0$
    \item For non-equilibrium solutions: $\int \frac{dy}{f(y)} = \int dx = x + C$
    \item This gives $x$ as a function of $y$ (inverse solution)
    \item Invert if possible to get $y(x)$
\end{enumerate}
\end{method}

\begin{insight}
Autonomous equations have time-translation symmetry: if $y(x)$ is a solution, so is $y(x - x_0)$ for any constant $x_0$.
\end{insight}

\section{Common Integration Patterns}

\subsection{Essential Integrals for Separable Equations}

\begin{center}
\begin{tabular}{|l|l|l|}
\hline
\textbf{Integral} & \textbf{Result} & \textbf{Notes} \\
\hline
$\int \frac{dy}{y}$ & $\ln|y| + C$ & $y \neq 0$ \\
\hline
$\int \frac{dy}{y^2}$ & $-\frac{1}{y} + C$ & $y \neq 0$ \\
\hline
$\int \frac{dy}{1-y^2}$ & $\frac{1}{2}\ln\left|\frac{1+y}{1-y}\right| + C$ & $|y| \neq 1$ \\
\hline
$\int \frac{dy}{1+y^2}$ & $\arctan(y) + C$ & All $y$ \\
\hline
$\int \frac{dy}{\sqrt{1-y^2}}$ & $\arcsin(y) + C$ & $|y| < 1$ \\
\hline
$\int \frac{dy}{y(1-y)}$ & $\ln\left|\frac{y}{1-y}\right| + C$ & $y \neq 0, 1$ \\
\hline
\end{tabular}
\end{center}

\section{Implicit Solutions}

\begin{definition}[Implicit Solution]
A relation $F(x,y) = C$ that cannot be solved explicitly for $y$ but satisfies the ODE.
\end{definition}

\begin{example}[Necessarily Implicit]
For $\frac{dy}{dx} = \frac{-2xy}{x^2 + 2y^2}$:

Separating gives: $x^2 + 2y^2 = C$ (circles and ellipses)

Cannot solve explicitly for $y$ as a single-valued function.
\end{example}

\begin{examtip}
Prof. Ditkowski accepts implicit solutions. Don't waste time trying to solve for $y$ if the algebra becomes messy!
\end{examtip}

\section{Existence and Uniqueness Issues}

\begin{theorem}[Uniqueness Violation at Equilibria]
For $\frac{dy}{dx} = f(y)$ with $f(y_0) = 0$:
\begin{itemize}
    \item The constant solution $y = y_0$ exists
    \item Non-constant solutions may touch $y = y_0$
    \item Uniqueness fails if $f'(y_0) = 0$
\end{itemize}
\end{theorem}

\begin{example}[Non-unique Solution]
$\frac{dy}{dx} = 2\sqrt{|y|}$ with $y(0) = 0$

Solutions include:
\begin{itemize}
    \item $y = 0$ (singular)
    \item $y = x^2$ for $x \geq 0$
    \item $y = -x^2$ for $x \leq 0$
    \item Combinations thereof
\end{itemize}
\end{example}

\section{Solution Verification}

\begin{method}[Verification Protocol]
\begin{enumerate}
    \item Differentiate the solution (implicit differentiation if needed)
    \item Substitute into original ODE
    \item Verify algebraic identity
    \item Check initial conditions
    \item Verify singular solutions separately
    \item State domain of validity
\end{enumerate}
\end{method}

\section{Common Errors}

\begin{warning}
Critical mistakes that cost points:
\begin{enumerate}
    \item Forgetting to check for singular solutions
    \item Losing absolute value signs in logarithms
    \item Incorrect partial fraction decomposition
    \item Not including integration constant
    \item Domain errors (e.g., $\ln(y)$ requires $y > 0$)
    \item Sign errors when separating
\end{enumerate}
\end{warning}

\end{document}