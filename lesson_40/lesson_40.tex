\documentclass[12pt]{article}
\usepackage{amsmath, amssymb, amsthm, tikz, pgfplots}
\usepackage{geometry, enumitem, mdframed, array, xcolor}
\usepackage{contour}
\geometry{margin=1in}

% Custom environments
\newtheorem{definition}{Definition}
\newtheorem{theorem}{Theorem}
\newtheorem{method}{Method}
\newtheorem{example}{Example}
\newmdenv[linecolor=blue,linewidth=2pt]{keypoint}
\newmdenv[linecolor=red,linewidth=2pt]{warning}
\newmdenv[linecolor=green,linewidth=2pt]{insight}
\newmdenv[linecolor=purple,linewidth=2pt]{examtip}
\newmdenv[linecolor=brown,linewidth=2pt]{jacobian}

\title{Lesson 40: Critical Points and Linearization\\
\large ODE 1 - Prof. Adi Ditkowski}
\author{Nonlinear Systems Analysis}
\date{}

\begin{document}
\maketitle

\section{Introduction to Nonlinear Systems}

\begin{definition}[Nonlinear Autonomous System]
A two-dimensional nonlinear autonomous system has the form:
\begin{align}
\dot{x} &= f(x,y) \\
\dot{y} &= g(x,y)
\end{align}
where $f$ and $g$ are continuously differentiable nonlinear functions.
\end{definition}

\begin{keypoint}
Unlike linear systems where solutions can be found explicitly using eigenvalues and eigenvectors, nonlinear systems generally cannot be solved in closed form. Linearization provides a powerful tool to understand local behavior.
\end{keypoint}

\section{Critical Points}

\begin{definition}[Critical Point/Equilibrium]
A point $(x_0, y_0)$ is a critical point (or equilibrium point) of the system if:
$$f(x_0, y_0) = 0 \quad \text{and} \quad g(x_0, y_0) = 0$$
\end{definition}

\begin{method}[Finding Critical Points]
\begin{enumerate}
    \item Set $f(x,y) = 0$ and $g(x,y) = 0$
    \item Solve the resulting algebraic system
    \item Check each solution for validity
    \item List all critical points systematically
\end{enumerate}
\end{method}

\begin{warning}
Nonlinear systems can have:
\begin{itemize}
    \item No critical points
    \item Finitely many critical points (most common)
    \item Infinitely many critical points (degenerate cases)
\end{itemize}
Always verify you've found ALL critical points!
\end{warning}

\section{The Jacobian Matrix}

\begin{definition}[Jacobian Matrix]
For the system $\dot{x} = f(x,y)$, $\dot{y} = g(x,y)$, the Jacobian matrix is:
$$J(x,y) = \begin{pmatrix}
\frac{\partial f}{\partial x} & \frac{\partial f}{\partial y} \\[8pt]
\frac{\partial g}{\partial x} & \frac{\partial g}{\partial y}
\end{pmatrix}$$
\end{definition}

\begin{jacobian}
\textbf{Critical Computation Rule:}
\begin{enumerate}
    \item Compute partial derivatives \textit{symbolically}
    \item \textit{Then} evaluate at the critical point $(x_0, y_0)$
    \item The result $J(x_0, y_0)$ is a constant matrix
\end{enumerate}
\end{jacobian}

\section{Linearization Process}

\begin{theorem}[Linearization near Equilibrium]
Near a critical point $(x_0, y_0)$, the nonlinear system can be approximated by the linear system:
$$\begin{pmatrix} \dot{\xi} \\ \dot{\eta} \end{pmatrix} = J(x_0, y_0) \begin{pmatrix} \xi \\ \eta \end{pmatrix}$$
where $\xi = x - x_0$ and $\eta = y - y_0$ represent small deviations from equilibrium.
\end{theorem}

\begin{insight}
The linearization is derived from the Taylor expansion:
\begin{align}
f(x,y) &= f(x_0,y_0) + \frac{\partial f}{\partial x}\bigg|_{(x_0,y_0)}(x-x_0) + \frac{\partial f}{\partial y}\bigg|_{(x_0,y_0)}(y-y_0) + \text{h.o.t.} \\
&= 0 + \text{linear terms} + \text{higher order terms}
\end{align}
Near the critical point, higher order terms are negligible!
\end{insight}

\section{Classification of Critical Points}

\begin{method}[Classification Procedure]
Given a critical point $(x_0, y_0)$:
\begin{enumerate}
    \item Compute $J(x_0, y_0)$
    \item Find eigenvalues $\lambda_1, \lambda_2$ of $J(x_0, y_0)$
    \item Classify using:
\end{enumerate}
\end{method}

\begin{center}
\begin{tabular}{|l|l|l|}
\hline
\textbf{Eigenvalues} & \textbf{Type} & \textbf{Stability} \\
\hline
$\lambda_1, \lambda_2 < 0$ (real) & Node & Asymptotically stable \\
$\lambda_1, \lambda_2 > 0$ (real) & Node & Unstable \\
$\lambda_1 < 0 < \lambda_2$ (real) & Saddle & Unstable \\
$\lambda = \alpha \pm i\beta$, $\alpha < 0$ & Spiral & Asymptotically stable \\
$\lambda = \alpha \pm i\beta$, $\alpha > 0$ & Spiral & Unstable \\
$\lambda = \pm i\beta$ (pure imaginary) & Center & Linearization inconclusive \\
\hline
\end{tabular}
\end{center}

\begin{examtip}
Prof. Ditkowski expects you to:
\begin{itemize}
    \item State the type (node, saddle, spiral, center)
    \item State stability (stable, unstable, or inconclusive)
    \item Show eigenvalue computation explicitly
\end{itemize}
\end{examtip}

\section{Complete Examples}

\begin{example}[Van der Pol Oscillator]
Consider the system:
\begin{align}
\dot{x} &= y \\
\dot{y} &= -x + \mu(1-x^2)y
\end{align}
where $\mu > 0$ is a parameter.

\textbf{Step 1:} Find critical points\\
Setting $y = 0$ and $-x + \mu(1-x^2) \cdot 0 = -x = 0$ gives $(0,0)$ as the only critical point.

\textbf{Step 2:} Compute Jacobian
$$J(x,y) = \begin{pmatrix} 0 & 1 \\ -1 - 2\mu xy & \mu(1-x^2) \end{pmatrix}$$

\textbf{Step 3:} Evaluate at $(0,0)$
$$J(0,0) = \begin{pmatrix} 0 & 1 \\ -1 & \mu \end{pmatrix}$$

\textbf{Step 4:} Find eigenvalues\\
Characteristic equation: $\lambda^2 - \mu\lambda + 1 = 0$\\
Eigenvalues: $\lambda = \frac{\mu \pm \sqrt{\mu^2 - 4}}{2}$

\textbf{Step 5:} Classification
\begin{itemize}
    \item If $0 < \mu < 2$: Complex with positive real part $\Rightarrow$ Unstable spiral
    \item If $\mu = 2$: Repeated positive eigenvalue $\Rightarrow$ Unstable node
    \item If $\mu > 2$: Two positive real eigenvalues $\Rightarrow$ Unstable node
\end{itemize}
\end{example}

\section{Validity of Linearization}

\begin{warning}
Linearization provides accurate local behavior EXCEPT when:
\begin{itemize}
    \item Eigenvalues have zero real part (centers, degenerate nodes)
    \item The system has special symmetries
    \item Nonlinear terms dominate near the critical point
\end{itemize}
\end{warning}

\begin{insight}
The region where linearization is valid shrinks as:
\begin{itemize}
    \item Eigenvalues approach the imaginary axis
    \item Nonlinearity becomes stronger
    \item Other critical points are nearby
\end{itemize}
\end{insight}

\section{Summary Flowchart}

\begin{center}
\tikzset{
    block/.style = {draw, rectangle, minimum height=2em, minimum width=8em},
    decision/.style = {draw, diamond, aspect=2},
    arrow/.style = {->, thick}
}

\begin{tikzpicture}[node distance=2cm]
    \node[block] (start) {Nonlinear System};
    \node[block, below of=start] (critical) {Find Critical Points};
    \node[block, below of=critical] (jacobian) {Compute Jacobian};
    \node[block, below of=jacobian] (evaluate) {Evaluate at $(x_0,y_0)$};
    \node[block, below of=evaluate] (eigen) {Find Eigenvalues};
    \node[block, below of=eigen] (classify) {Classify \& State Stability};

    \draw[arrow] (start) -- (critical);
    \draw[arrow] (critical) -- (jacobian);
    \draw[arrow] (jacobian) -- (evaluate);
    \draw[arrow] (evaluate) -- (eigen);
    \draw[arrow] (eigen) -- (classify);
\end{tikzpicture}
\end{center}

\end{document}