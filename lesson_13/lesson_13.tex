\documentclass[12pt]{article}
\usepackage{amsmath, amssymb, amsthm, tikz, pgfplots}
\usepackage{geometry, enumitem, mdframed, array, xcolor}
\geometry{margin=1in}

% Custom environments
\newtheorem{definition}{Definition}
\newtheorem{theorem}{Theorem}
\newtheorem{method}{Method}
\newtheorem{example}{Example}
\newmdenv[linecolor=blue,linewidth=2pt]{keypoint}
\newmdenv[linecolor=red,linewidth=2pt]{warning}
\newmdenv[linecolor=green,linewidth=2pt]{insight}
\newmdenv[linecolor=purple,linewidth=2pt]{examtip}

\title{Implicit Solutions and Singular Solutions}
\author{ODE 1 - Prof. Adi Ditkowski}
\date{Lesson 13}

\begin{document}
\maketitle

\section{Implicit Solutions}

\begin{definition}[Implicit Solution]
An \textbf{implicit solution} of the ODE $\frac{dy}{dx} = f(x,y)$ is a relation $F(x,y) = C$ that:
\begin{enumerate}
    \item Defines $y$ as a function of $x$ (possibly multi-valued)
    \item Satisfies the ODE when differentiated implicitly
\end{enumerate}
\end{definition}

\begin{method}[Verifying Implicit Solutions]
Given $F(x,y) = C$ and ODE $\frac{dy}{dx} = f(x,y)$:
\begin{enumerate}
    \item Differentiate $F(x,y) = C$ implicitly: $\frac{\partial F}{\partial x} + \frac{\partial F}{\partial y}\frac{dy}{dx} = 0$
    \item Solve for $\frac{dy}{dx}$: $\frac{dy}{dx} = -\frac{\partial F/\partial x}{\partial F/\partial y}$
    \item Verify this equals $f(x,y)$
\end{enumerate}
\end{method}

\begin{example}[Implicit Verification]
Verify that $x^2 + xy + y^2 = C$ solves some ODE and find it.

\textbf{Solution:}
Differentiating implicitly:
$$2x + y + x\frac{dy}{dx} + 2y\frac{dy}{dx} = 0$$

Solving for $\frac{dy}{dx}$:
$$\frac{dy}{dx} = -\frac{2x + y}{x + 2y}$$

Therefore, $x^2 + xy + y^2 = C$ solves $\frac{dy}{dx} = -\frac{2x + y}{x + 2y}$
\end{example}

\begin{keypoint}
Implicit solutions are preferable when:
\begin{itemize}
    \item Explicit solution involves complex expressions
    \item Solution curves have vertical tangents
    \item Multiple branches exist
    \item The implicit form has geometric meaning
\end{itemize}
\end{keypoint}

\section{Singular Solutions}

\begin{definition}[Singular Solution]
A \textbf{singular solution} is a solution that:
\begin{enumerate}
    \item Satisfies the ODE
    \item Cannot be obtained from the general solution for any value of the arbitrary constant
    \item Is typically the envelope of the general solution family
\end{enumerate}
\end{definition}

\begin{theorem}[Existence of Singular Solutions]
Singular solutions may exist when:
\begin{itemize}
    \item The ODE is not linear in $y'$
    \item Uniqueness conditions fail
    \item The general solution involves parameters nonlinearly
\end{itemize}
\end{theorem}

\section{Methods for Finding Singular Solutions}

\subsection{C-Discriminant Method}

\begin{method}[C-Discriminant]
Given general solution $F(x,y,C) = 0$:
\begin{enumerate}
    \item Form the system:
    \begin{align}
    F(x,y,C) &= 0\\
    \frac{\partial F}{\partial C} &= 0
    \end{align}
    \item Eliminate $C$ between these equations
    \item The result is the C-discriminant
    \item Test if it satisfies the original ODE
\end{enumerate}
\end{method}

\begin{example}[C-Discriminant Application]
For $(y - Cx)^2 = C^2 + 1$ (general solution):

\textbf{Step 1:} Compute $\frac{\partial F}{\partial C} = 0$:
$$2(y - Cx)(-x) = 2C$$
$$-x(y - Cx) = C$$

\textbf{Step 2:} Substitute into original:
From $C = -x(y - Cx) = -xy + Cx^2$, we get $C(1 - x^2) = -xy$

If $x^2 \neq 1$: $C = \frac{-xy}{1 - x^2}$

\textbf{Step 3:} Substituting back gives the singular solution.
\end{example}

\subsection{p-Discriminant Method}

\begin{method}[p-Discriminant]
From ODE $F(x,y,y') = 0$:
\begin{enumerate}
    \item Let $p = y'$ and write $F(x,y,p) = 0$
    \item Form the system:
    \begin{align}
    F(x,y,p) &= 0\\
    \frac{\partial F}{\partial p} &= 0
    \end{align}
    \item Eliminate $p$ to get the p-discriminant
    \item Verify it satisfies the ODE
\end{enumerate}
\end{method}

\section{Clairaut's Equation}

\begin{definition}[Clairaut's Equation]
An ODE of the form:
$$y = xy' + f(y')$$
where $f$ is a function of $y'$ alone.
\end{definition}

\begin{theorem}[Solutions of Clairaut's Equation]
Clairaut's equation $y = xy' + f(y')$ has:
\begin{enumerate}
    \item \textbf{General solution:} $y = Cx + f(C)$ (family of straight lines)
    \item \textbf{Singular solution:} Obtained by eliminating $p$ from:
    \begin{align}
    y &= xp + f(p)\\
    0 &= x + f'(p)
    \end{align}
\end{enumerate}
\end{theorem}

\begin{example}[Complete Clairaut Solution]
Solve: $y = xy' - (y')^2$

\textbf{General solution:} $y = Cx - C^2$ (family of lines)

\textbf{Singular solution:}
From $x + f'(p) = 0$: $x - 2p = 0$, so $p = x/2$

Substituting: $y = x(x/2) - (x/2)^2 = x^2/4$

The parabola $y = x^2/4$ is the envelope of all lines.
\end{example}

\section{Envelope Theory}

\begin{definition}[Envelope]
The \textbf{envelope} of a family of curves $F(x,y,C) = 0$ is a curve that:
\begin{enumerate}
    \item Is tangent to each member of the family
    \item At each point, is tangent to exactly one family member
\end{enumerate}
\end{definition}

\begin{insight}
Singular solutions are often envelopes of the general solution family. They represent the boundary of the region covered by all general solutions.
\end{insight}

\begin{method}[Finding Envelopes]
To find the envelope of $F(x,y,C) = 0$:
\begin{enumerate}
    \item Solve simultaneously:
    $$F(x,y,C) = 0, \quad \frac{\partial F}{\partial C} = 0$$
    \item Eliminate $C$ to get envelope equation
    \item Verify tangency conditions
\end{enumerate}
\end{method}

\section{Parametric Solutions}

\begin{definition}[Parametric Solution]
A solution expressed as:
$$x = x(t), \quad y = y(t)$$
where $t$ is a parameter.
\end{definition}

\begin{example}[When Parametric is Better]
For $(y')^2 + y' = x$:

Let $p = y'$, then $x = p^2 + p$

Differentiating: $dx = (2p + 1)dp$

Since $dy = p\,dx = p(2p + 1)dp$:
$$y = \int p(2p + 1)dp = \frac{2p^3}{3} + \frac{p^2}{2} + C$$

Parametric solution:
$$x = p^2 + p, \quad y = \frac{2p^3}{3} + \frac{p^2}{2} + C$$
\end{example}

\section{Verification Techniques}

\begin{method}[Complete Verification Protocol]
\begin{enumerate}
    \item \textbf{For implicit solutions:} Use implicit differentiation
    \item \textbf{For singular solutions:}
        \begin{itemize}
            \item Direct substitution into ODE
            \item Show it's not in general family
        \end{itemize}
    \item \textbf{For parametric solutions:} Use $\frac{dy}{dx} = \frac{dy/dt}{dx/dt}$
\end{enumerate}
\end{method}

\begin{examtip}
Prof. Ditkowski always awards points for proper verification, even if the solution is incorrect. Show all steps!
\end{examtip}

\section{Common Errors}

\begin{warning}
Critical mistakes:
\begin{enumerate}
    \item Assuming all solutions can be made explicit
    \item Missing singular solutions in nonlinear ODEs
    \item Incorrect implicit differentiation
    \item Confusing singular solutions with particular solutions
    \item Not checking if p-discriminant satisfies ODE
\end{enumerate}
\end{warning}

\end{document}