\documentclass[12pt]{article}
\usepackage{amsmath, amssymb, amsthm, tikz, pgfplots}
\usepackage{geometry, enumitem, mdframed, array, xcolor}
\geometry{margin=1in}

\newtheorem{definition}{Definition}
\newtheorem{theorem}{Theorem}
\newtheorem{method}{Method}
\newtheorem{example}{Example}
\newmdenv[linecolor=blue,linewidth=2pt]{keypoint}
\newmdenv[linecolor=red,linewidth=2pt]{warning}
\newmdenv[linecolor=green,linewidth=2pt]{insight}
\newmdenv[linecolor=purple,linewidth=2pt]{examtip}
\newmdenv[linecolor=orange,linewidth=2pt]{transformation}

\title{Lesson 17: Homogeneous First-Order Differential Equations}
\author{ODE 1 - Prof. Adi Ditkowski}
\date{}

\begin{document}
\maketitle

\section{Recognition and Definition}

\begin{definition}[Homogeneous Function]
A function $f(x,y)$ is homogeneous of degree $n$ if:
$$f(tx, ty) = t^n f(x,y) \quad \text{for all } t > 0$$
\end{definition}

\begin{definition}[Homogeneous Differential Equation]
A first-order ODE is homogeneous if it can be written as:
$$\frac{dy}{dx} = F\left(\frac{y}{x}\right)$$
or equivalently, if $M(x,y)dx + N(x,y)dy = 0$ where $M$ and $N$ are homogeneous functions of the same degree.
\end{definition}

\begin{keypoint}
\textbf{Quick Recognition Tests:}
\begin{enumerate}
    \item Check if all terms have the same total degree in $x$ and $y$
    \item Try to factor out powers to write as $F(y/x)$
    \item Apply the scaling test: $f(tx, ty) = t^n f(x,y)$
\end{enumerate}
\end{keypoint}

\section{The Substitution Method}

\begin{transformation}
\textbf{The $v = y/x$ Substitution Algorithm:}
\begin{enumerate}
    \item Set $v = \frac{y}{x}$, so $y = vx$
    \item Differentiate: $\frac{dy}{dx} = v + x\frac{dv}{dx}$ \textcolor{red}{(Product Rule!)}
    \item Substitute into the original equation
    \item Simplify to get: $x\frac{dv}{dx} = F(v) - v$
    \item Separate variables: $\frac{dv}{F(v) - v} = \frac{dx}{x}$
    \item Integrate both sides
    \item Back-substitute $v = y/x$
\end{enumerate}
\end{transformation}

\begin{warning}
\textbf{Critical Points:}
\begin{itemize}
    \item The derivative $\frac{dy}{dx} = v + x\frac{dv}{dx}$ comes from the product rule
    \item Check for singular solutions where $F(v) - v = 0$
    \item The substitution fails along $x = 0$ (use $u = x/y$ instead if needed)
\end{itemize}
\end{warning}

\section{Detailed Examples}

\begin{example}[Standard Homogeneous]
Solve: $\frac{dy}{dx} = \frac{x^2 + xy + y^2}{x^2}$

\textbf{Solution:}
\begin{enumerate}
    \item Verify homogeneity: $\frac{dy}{dx} = 1 + \frac{y}{x} + \left(\frac{y}{x}\right)^2 = F(y/x)$ \checkmark
    \item Let $v = y/x$, then $y = vx$ and $\frac{dy}{dx} = v + x\frac{dv}{dx}$
    \item Substitute: $v + x\frac{dv}{dx} = 1 + v + v^2$
    \item Simplify: $x\frac{dv}{dx} = 1 + v^2$
    \item Separate: $\frac{dv}{1 + v^2} = \frac{dx}{x}$
    \item Integrate: $\arctan(v) = \ln|x| + C$
    \item Back-substitute: $\arctan\left(\frac{y}{x}\right) = \ln|x| + C$
\end{enumerate}
\end{example}

\begin{example}[Disguised Homogeneous]
Solve: $(x - y)\frac{dy}{dx} = x + y$

\textbf{Solution:}
\begin{enumerate}
    \item Rewrite: $\frac{dy}{dx} = \frac{x + y}{x - y} = \frac{1 + y/x}{1 - y/x}$ (homogeneous!)
    \item Let $v = y/x$: $v + x\frac{dv}{dx} = \frac{1 + v}{1 - v}$
    \item Simplify: $x\frac{dv}{dx} = \frac{1 + v}{1 - v} - v = \frac{1 + v - v(1 - v)}{1 - v} = \frac{1 + v^2}{1 - v}$
    \item Separate: $\frac{1 - v}{1 + v^2}dv = \frac{dx}{x}$
    \item Use partial fractions on the left side
    \item Final solution involves $\arctan$ and $\ln$ terms
\end{enumerate}
\end{example}

\section{Special Cases and Variations}

\begin{insight}
\textbf{Alternative Substitution:}
When the equation has more $y$ terms, try $u = \frac{x}{y}$:
\begin{itemize}
    \item $x = uy$ implies $\frac{dx}{dy} = u + y\frac{du}{dy}$
    \item The equation becomes separable in $u$ and $y$
\end{itemize}
\end{insight}

\begin{examtip}
\textbf{Prof. Ditkowski's Exam Patterns:}
\begin{itemize}
    \item Often combines homogeneous with initial conditions
    \item May ask to verify homogeneity before solving
    \item Likes equations of the form $(ax + by)dx + (cx + dy)dy = 0$
    \item Tests recognition with trigonometric terms like $\sin(y/x)$
    \item Partial credit for correct substitution setup
\end{itemize}
\end{examtip}

\section{Geometric Interpretation}

Solution curves of homogeneous equations have the property that they look similar under scaling from the origin. If $(x,y)$ is on a solution curve, then $(kx, ky)$ is on a geometrically similar curve.

\section{Recognition Flowchart}

The key steps for recognition are:
\begin{enumerate}
\item Check if the equation can be written as $\frac{dy}{dx} = F\left(\frac{y}{x}\right)$
\item Verify that all terms have the same total degree in $x$ and $y$
\item Apply the scaling test: $f(tx, ty) = t^n f(x,y)$
\end{enumerate}

If any test confirms homogeneity, proceed with the $v = y/x$ substitution.

\end{document}