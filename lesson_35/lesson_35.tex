\documentclass[12pt]{article}
\usepackage{amsmath, amssymb, amsthm, tikz, pgfplots}
\usepackage{geometry, enumitem, mdframed, array, xcolor}
\usepackage{nicematrix, systeme}
\geometry{margin=1in}

% Custom environments
\newtheorem{definition}{Definition}
\newtheorem{theorem}{Theorem}
\newtheorem{method}{Method}
\newtheorem{example}{Example}
\newmdenv[linecolor=blue,linewidth=2pt]{keypoint}
\newmdenv[linecolor=red,linewidth=2pt]{warning}
\newmdenv[linecolor=green,linewidth=2pt]{insight}
\newmdenv[linecolor=purple,linewidth=2pt]{examtip}
\newmdenv[linecolor=orange,linewidth=2pt]{duhamel}

\title{ODE Lesson 35: Duhamel's Principle - The Complete Method}
\author{ODE 1 - Prof. Adi Ditkowski}
\date{}

\begin{document}
\maketitle

\section{The Master Formula}

\begin{duhamel}
\textbf{Duhamel's Principle:}
The unique solution to the initial value problem
\[\mathbf{x}'(t) = A\mathbf{x}(t) + \mathbf{f}(t), \quad \mathbf{x}(t_{0}) = \mathbf{x}_{0}\]
is given by
\[\boxed{\mathbf{x}(t) = $e^{A(t-t_{0}$)}\mathbf{x}_{0} + \int_{t_{0}}^{t} e^{A(t-s)}\mathbf{f}(s)\,ds}\]
\end{duhamel}

\begin{theorem}[Components of the Solution]
The complete solution consists of:
\begin{enumerate}
\item \textbf{Homogeneous contribution:} \mathbf{x}_{h}(t) = $e^{A(t-t_{0}$)}\mathbf{x}_{0}
   \begin{itemize}
   \item Evolves initial condition forward
   \item Depends only on system dynamics (matrix $A$)
   \item Satisfies $\mathbf{x}_{h}' = A\mathbf{x}_{h}$ with $\mathbf{x}_{h}(t_{0}) = \mathbf{x}_{0}$
   \end{itemize}

\item \textbf{Particular contribution:} \mathbf{x}_{p}(t) = \int_{t_{0}}^{t} e^{A(t-s)}\mathbf{f}(s)\,ds
   \begin{itemize}
   \item Response to forcing
   \item Zero initial condition: $\mathbf{x}_{p}(t_{0}) = \mathbf{0}$
   \item Convolution of forcing with impulse response
   \end{itemize}
\end{enumerate}
\end{theorem}

\section{Solution Operator Perspective}

\begin{definition}[Solution Operator]
The solution operator (or propagator) $S(t,s)$ maps the state at time $s$ to the state at time $t:
\[S(t,s) = e^{A(t-s)}\]
Properties:
\begin{itemize}
\item S(t,t) = I$ (identity at equal times)
\item $S(t,r)S(r,s) = S(t,s)$ (semigroup property)
\item $S(t,s)^{-1} = S(s,t)$ (time reversal)
\end{itemize}
\end{definition}

\begin{keypoint}
\textbf{Physical Interpretation:}
\[\mathbf{x}(t) = \underbrace{S(t,t_{0})\mathbf{x}_{0}}_{\text{Evolution of IC}} + \underbrace{\int_{t_{0}}^{t} S(t,s)\mathbf{f}(s)\,ds}_{\text{Accumulated forcing effects}}\]
Each infinitesimal forcing $\mathbf{f}(s)ds$ creates an impulse at time $s$ that propagates to time $t$ via $S(t,s)$.
\end{keypoint}

\section{Complete Worked Examples}

\begin{example}[2$\times2 System with Exponential Forcing]
Solve: \mathbf{x}' = \begin{pmatrix} 0 & 1 \\ 0 & 0 \end{pmatrix}\mathbf{x} + \begin{pmatrix} 0 \\ $e^{t}$ \end{pmatrix} with \mathbf{x}(0) = \begin{pmatrix} 1 \\ 0 \end{pmatrix}

\textbf{Solution:}
\begin{enumerate}
\item \textbf{Find $e^{At}$:}
Since A^{2} = 0 (nilpotent), we have:
\[$e^{At}$ = I + At = \begin{pmatrix} 1 & t \\ 0 & 1 \end{pmatrix}\]

\item \textbf{Homogeneous solution:}
\[\mathbf{x}_{h}(t) = e^{At}\mathbf{x}_{0} = \begin{pmatrix} 1 & t \\ 0 & 1 \end{pmatrix}\begin{pmatrix} 1 \\ 0 \end{pmatrix} = \begin{pmatrix} 1 \\ 0 \end{pmatrix}\]

\item \textbf{Particular solution via Duhamel:}
\begin{align}
\mathbf{x}_{p}(t) &= \int_{0}^t e^{A(t-s)}\mathbf{f}(s)\,ds \\
&= \int_{0}^t \begin{pmatrix} 1 & t-s \\ 0 & 1 \end{pmatrix}\begin{pmatrix} 0 \\ $e^{s}$ \end{pmatrix}ds \\
&= \int_{0}^t \begin{pmatrix} (t-s)$e^{s}$ \\ $e^{s}$ \end{pmatrix}ds
\end{align}

Computing the integrals:
\begin{itemize}
\item Second component: \int_{0}^t e^{s}\,ds = $e^{t}$ - 1
\item First component: \int_{0}^t (t-s)e^{s}\,ds = t\int_{0}^t e^{s}\,ds - \int_{0}^t se^{s}\,ds

Using integration by parts for \int se^{s}\,ds = se^{s} - $e^{s}$:
\[= t($e^{t}$ - 1) - [(se^{s} - $e^{s}$)]_{0}^t = t($e^{t}$ - 1) - (te^{t} - $e^{t}$ - (0 - 1))\]
\[= te^{t} - t - te^{t} + $e^{t}$ + 1 = $e^{t}$ - t + 1\]
\end{itemize}

So: \mathbf{x}_{p}(t) = \begin{pmatrix} $e^{t}$ - t + 1 \\ $e^{t}$ - 1 \end{pmatrix}

\item \textbf{Complete solution:}
\[\mathbf{x}(t) = \begin{pmatrix} 1 \\ 0 \end{pmatrix} + \begin{pmatrix} $e^{t}$ - t + 1 \\ $e^{t}$ - 1 \end{pmatrix} = \begin{pmatrix} $e^{t}$ - t + 2 \\ $e^{t}$ - 1 \end{pmatrix}\]
\end{enumerate}
\end{example}

\begin{example}[System with Discontinuous Forcing]
Solve: \mathbf{x}' = \begin{pmatrix} -1 & 0 \\ 0 & -2 \end{pmatrix}\mathbf{x} + \begin{pmatrix} H(t-1) \\ 0 \end{pmatrix}$ with $\mathbf{x}(0) = \begin{pmatrix} 0 \\ 1 \end{pmatrix}$

where $H(t-1) is the Heaviside step function.

\textbf{Solution:}
\begin{enumerate}
\item $e^{At}$ = \begin{pmatrix} $e^{-t}$ & 0 \\ 0 & $e^{-2t}$ \end{pmatrix}$

\item Homogeneous: \mathbf{x}_{h}(t) = \begin{pmatrix} 0 \\ $e^{-2t}$ \end{pmatrix}

\item Particular (for $t > 1):
\begin{align}
\mathbf{x}_{p}(t) &= \int_{0}^t \begin{pmatrix} $e^{-(t-s)}$ & 0 \\ 0 & $e^{-2(t-s)}$ \end{pmatrix}\begin{pmatrix} H(s-1) \\ 0 \end{pmatrix}ds \\
&= \int_{1}^t \begin{pmatrix} $e^{-(t-s)}$ \\ 0 \end{pmatrix}ds = \begin{pmatrix} e^{-t}\int_{1}^t e^{s}\,ds \\ 0 \end{pmatrix} \\
&= \begin{pmatrix} $e^{-t}$(e^{t} - e) \\ 0 \end{pmatrix} = \begin{pmatrix} 1 - $e^{1-t}$ \\ 0 \end{pmatrix}
\end{align}

\item Complete solution:
\[\mathbf{x}(t) = \begin{cases}
\begin{pmatrix} 0 \\ $e^{-2t}$ \end{pmatrix} & t < 1 \\
\begin{pmatrix} 1 - $e^{1-t}$ \\ $e^{-2t}$ \end{pmatrix} & t \geq 1
\end{cases}\]
\end{enumerate}
\end{example}

\section{Special Forcing Functions}

\begin{insight}
\textbf{Impulse Response:}
For a Dirac delta forcing \mathbf{f}(t) = \delta(t-a)\mathbf{b}:
\[\mathbf{x}(t) = e^{At}\mathbf{x}_{0} + H(t-a)e^{A(t-a)}\mathbf{b}\]
The impulse creates an instantaneous jump in the state at time a$.
\end{insight}

\begin{method}[Periodic Forcing]
For periodic forcing $\mathbf{f}(t + T) = \mathbf{f}(t)$, the steady-state solution (if stable) is also periodic with period $T:
\[\mathbf{x}_{ss}(t) = \frac{1}{I - $e^{-AT}$}\int_{0}^T e^{A(T-s)}\mathbf{f}(s)\,ds\]
(Valid when all eigenvalues have negative real parts)
\end{method}

\section{Computational Strategies}

\begin{examtip}
\textbf{Prof. Ditkowski's Exam Strategy:}
\begin{enumerate}
\item Write Duhamel's formula immediately
\item Compute $e^{At}$ using the best method:
   \begin{itemize}
   \item Diagonal A$ \rightarrow direct
   \item Diagonalizable \rightarrow Pe^{Dt}P^{-1}
   \item Jordan blocks \rightarrow polynomial terms
   \item Nilpotent \rightarrow finite series
   \end{itemize}
\item Apply to initial condition
\item Set up the convolution integral
\item Evaluate (often simplifies nicely on exams)
\item Combine homogeneous and particular parts
\end{enumerate}
\end{examtip}

\begin{warning}
\textbf{Common Errors:}
\begin{itemize}
\item Writing e^{At}$e^{-As}$ instead of e^{A(t-s)}$
\item Forgetting that $t$ is a parameter in the integral (integrate over $s)
\item Missing the initial condition contribution
\item Wrong bounds of integration
\item Not checking that the solution satisfies the ODE
\end{itemize}
\end{warning}

\section{Connection to Previous Methods}

\begin{center}
\begin{tabular}{|l|l|}
\hline
\textbf{Method} & \textbf{Relation to Duhamel} \\
\hline
Eigenvalue method & Used to compute $e^{At}$ \\
Matrix exponential & Core of the solution operator \\
Variation of parameters & Second term in Duhamel's formula \\
Fundamental matrix & \Phi(t) = e^{At}\Phi(0)$ \\
Green's function & G(t,s) = e^{A(t-s)}H(t-s) \\
\hline
\end{tabular}
\end{center}

\section{Advanced Properties}

\begin{theorem}[Uniqueness]
Duhamel's formula gives the UNIQUE solution to the IVP. This follows from:
\begin{itemize}
\item Existence: The formula always produces a solution
\item Uniqueness: Follows from Picard-Lindelöf theorem
\item The solution is continuously differentiable if $\mathbf{f}$ is continuous
\end{itemize}
\end{theorem}

\begin{theorem}[Stability Implications]
If all eigenvalues of $A$ have negative real parts:
\begin{itemize}
\item \|e^{At}\| \leq Me^{-\alpha t} for some $M, \alpha > 0$
\item The homogeneous part decays exponentially
\item Bounded forcing produces bounded solutions
\item System has a unique steady-state for constant forcing
\end{itemize}
\end{theorem}

\end{document}