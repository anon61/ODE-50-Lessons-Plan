\documentclass[12pt]{article}
\usepackage{amsmath, amssymb, amsthm, tikz, pgfplots}
\usepackage{geometry, enumitem, mdframed, array, xcolor}
\geometry{margin=1in}

% Custom environments
\newtheorem{definition}{Definition}
\newtheorem{theorem}{Theorem}
\newtheorem{proposition}{Proposition}
\newtheorem{example}{Example}
\newmdenv[linecolor=blue,linewidth=2pt]{keypoint}
\newmdenv[linecolor=red,linewidth=2pt]{warning}
\newmdenv[linecolor=green,linewidth=2pt]{insight}
\newmdenv[linecolor=purple,linewidth=2pt]{examtip}

\title{ODE Lesson 36: Phase Space and Trajectories: Geometric Theory of ODEs}
\author{ODE 1 - Prof. Adi Ditkowski}
\date{}

\begin{document}
\maketitle

\section{Phase Space Fundamentals}

\begin{definition}[Phase Space]
For a system of $n$ first-order ODEs:
$$\frac{d\mathbf{x}}{dt} = \mathbf{f}(\mathbf{x}, t)$$
where $\mathbf{x} = (x_1, x_2, \ldots, x_n)^T$, the \textbf{phase space} is the $n$-dimensional space with coordinates $(x_1, x_2, \ldots, x_n)$.
\end{definition}

\begin{keypoint}
\textbf{Key Insight:} Phase space represents all possible states of the system. Each point corresponds to a unique state, regardless of time.
\end{keypoint}

\begin{definition}[Trajectory/Orbit]
A \textbf{trajectory} (or \textbf{orbit}) is the curve traced out in phase space by a solution $\mathbf{x}(t)$ as $t$ varies. Formally, it is the set:
$$\Gamma = \{\mathbf{x}(t) : t \in I\}$$
where $I$ is the interval of existence.
\end{definition}

\begin{warning}
\textbf{Critical Distinction:}
\begin{itemize}
    \item \textbf{Solution:} $\mathbf{x}(t)$ - includes time parametrization
    \item \textbf{Trajectory:} The geometric curve - no time information
\end{itemize}
Multiple solutions can give the same trajectory (time-shifted solutions)!
\end{warning}

\section{Direction Fields and Flow}

\begin{definition}[Direction Field]
The \textbf{direction field} (or \textbf{vector field}) of the system $\dot{\mathbf{x}} = \mathbf{f}(\mathbf{x})$ assigns to each point $\mathbf{x}$ the vector $\mathbf{f}(\mathbf{x})$, indicating the instantaneous direction and speed of motion.
\end{definition}

\begin{theorem}[Non-Intersection of Trajectories]
If $\mathbf{f}(\mathbf{x})$ satisfies the conditions for existence and uniqueness, then trajectories cannot intersect except at equilibrium points.
\end{theorem}

\begin{proof}
Suppose two trajectories $\Gamma_1$ and $\Gamma_2$ intersect at point $\mathbf{x}_0$ which is not an equilibrium. By uniqueness, the solution starting at $\mathbf{x}_0$ is unique, so $\Gamma_1 = \Gamma_2$. Contradiction.
\end{proof}

\section{Equilibrium Points and Invariant Sets}

\begin{definition}[Equilibrium Point]
A point $\mathbf{x}^*$ is an \textbf{equilibrium point} (or \textbf{critical point}, \textbf{fixed point}) if:
$$\mathbf{f}(\mathbf{x}^*) = \mathbf{0}$$
\end{definition}

\begin{examtip}
\textbf{Finding Equilibria - Prof. Ditkowski's Method:}
\begin{enumerate}
    \item Set all derivatives equal to zero
    \item Solve the resulting algebraic system
    \item Check each solution carefully
    \item State coordinates explicitly: "Equilibrium at $(x^*, y^*) = (a, b)$"
\end{enumerate}
\end{examtip}

\begin{definition}[Invariant Set]
A set $S \subset \mathbb{R}^n$ is \textbf{invariant} under the flow if:
$$\mathbf{x}(0) \in S \implies \mathbf{x}(t) \in S \text{ for all } t$$
\end{definition}

\section{Special Types of Trajectories}

\begin{definition}[Closed Orbit]
A trajectory $\Gamma$ is a \textbf{closed orbit} if it is homeomorphic to a circle and the solution is periodic:
$$\exists T > 0 : \mathbf{x}(t + T) = \mathbf{x}(t) \text{ for all } t$$
\end{definition}

\begin{insight}
\textbf{Dimension Restriction:} Closed orbits cannot exist in 1D phase space! In 1D, trajectories are confined to the real line and cannot loop back without violating uniqueness.
\end{insight}

\begin{definition}[Heteroclinic and Homoclinic Orbits]
\begin{itemize}
    \item \textbf{Heteroclinic orbit:} Connects two different equilibria
    \item \textbf{Homoclinic orbit:} Starts and ends at the same equilibrium
\end{itemize}
\end{definition}

\section{Phase Portraits for 2D Systems}

For a 2D autonomous system:
\begin{align}
\frac{dx}{dt} &= f(x, y) \\
\frac{dy}{dt} &= g(x, y)
\end{align}

\begin{keypoint}
\textbf{Construction Steps:}
\begin{enumerate}
    \item Find all equilibria: solve $f(x,y) = 0$ and $g(x,y) = 0$
    \item Compute the direction field at selected points
    \item Identify special trajectories (if any)
    \item Sketch trajectories following the direction field
    \item Add arrows indicating flow direction
\end{enumerate}
\end{keypoint}

\section{Converting Higher-Order Equations}

\begin{example}[Second-Order to First-Order System]
Convert $\ddot{x} + p(x)\dot{x} + q(x) = 0$ to phase space form:

Let $x_1 = x$ and $x_2 = \dot{x}$. Then:
\begin{align}
\dot{x}_1 &= x_2 \\
\dot{x}_2 &= -p(x_1)x_2 - q(x_1)
\end{align}
Phase space is the $(x_1, x_2)$-plane, often relabeled as $(x, \dot{x})$-plane.
\end{example}

\section{Nullclines Method}

\begin{definition}[Nullclines]
The \textbf{nullclines} are curves where one component of the vector field vanishes:
\begin{itemize}
    \item \textbf{$x$-nullcline:} $f(x,y) = 0$ (vertical flow)
    \item \textbf{$y$-nullcline:} $g(x,y) = 0$ (horizontal flow)
\end{itemize}
\end{definition}

\begin{insight}
\textbf{Using Nullclines:}
\begin{itemize}
    \item Equilibria occur at nullcline intersections
    \item Trajectories cross nullclines vertically or horizontally
    \item Nullclines divide phase space into regions with consistent flow direction
\end{itemize}
\end{insight}

\section{Exam-Critical Formulas}

\begin{examtip}
\textbf{Must-Know for Prof. Ditkowski's Exam:}

\begin{tabular}{|l|l|}
\hline
\textbf{Concept} & \textbf{Formula/Property} \\
\hline
Equilibrium condition & $\mathbf{f}(\mathbf{x}^*) = \mathbf{0}$ \\
\hline
Trajectory uniqueness & No intersections except at equilibria \\
\hline
Closed orbit period & $\mathbf{x}(t+T) = \mathbf{x}(t)$ \\
\hline
Direction at point $(x,y)$ & Vector $(f(x,y), g(x,y))$ \\
\hline
Speed along trajectory & $|\mathbf{f}(\mathbf{x})| = \sqrt{f^2 + g^2}$ \\
\hline
\end{tabular}
\end{examtip}

\end{document}