\documentclass[12pt]{article}
\usepackage{amsmath, amssymb, amsthm, tikz, pgfplots}
\usepackage{geometry, enumitem, mdframed, array, xcolor}
\usepackage{nicematrix}
\geometry{margin=1in}

\newtheorem{definition}{Definition}
\newtheorem{theorem}{Theorem}
\newtheorem{proposition}{Proposition}
\newtheorem{lemma}{Lemma}
\newtheorem{example}{Example}
\newmdenv[linecolor=blue,linewidth=2pt]{keypoint}
\newmdenv[linecolor=red,linewidth=2pt]{warning}
\newmdenv[linecolor=green,linewidth=2pt]{insight}
\newmdenv[linecolor=purple,linewidth=2pt]{examtip}
\newmdenv[linecolor=brown,linewidth=2pt]{computation}

\title{Lesson 28: Wronskian and Linear Independence}
\author{ODE 1 - Prof. Adi Ditkowski}
\date{}

\begin{document}
\maketitle

\section{The Wronskian Matrix and Determinant}

\begin{definition}[Wronskian Matrix]
For $n$ vector solutions $\mathbf{x}$_{1}$(t), \mathbf{x}$_{2}$(t), \ldots, \mathbf{x}$_{n}$(t)$ of a system, the \textbf{Wronskian matrix} is:
$$W(t) = [\mathbf{x}$_{1}$(t) \mid \mathbf{x}$_{2}$(t) \mid \cdots \mid \mathbf{x}$_{n}$(t)]$$
\end{definition}

\begin{definition}[Wronskian Determinant]
The \textbf{Wronskian determinant} (or simply Wronskian) is:
$$W(t) = \det(W(t))$$
\end{definition}

\begin{keypoint}
For solutions of $\mathbf{x}' = A(t)\mathbf{x}$ with continuous $A(t)$:
\begin{itemize}
\item Either $W(t) = 0$ for all $t$ (linearly dependent)
\item Or $W(t) \neq 0$ for all $t$ (linearly independent)
\end{itemize}
This is the \textbf{all-or-nothing property}.
\end{keypoint}

\section{Abel's Identity}

\begin{theorem}[Abel's Identity for Systems]
If $\mathbf{x}$_{1}$(t), \ldots, \mathbf{x}$_{n}$(t)$ are solutions to $\mathbf{x}' = A(t)\mathbf{x}$, then:
$$W(t) = W(t$_{0}$) \exp\left(\int_{t$_{0}$}$^{t}$ \text{tr}(A(s))\,ds\right)$$
where $\text{tr}(A)$ is the trace (sum of diagonal elements).
\end{theorem}

\begin{insight}
Abel's identity shows that:
\begin{itemize}
\item $W(t) = 0$ for some $t \Rightarrow W(t) = 0$ for all $t$
\item $W(t) \neq 0$ for some $t \Rightarrow W(t) \neq 0$ for all $t$
\item The growth/decay rate of $W(t)$ depends only on $\text{tr}(A)$
\end{itemize}
\end{insight}

\section{Wronskian for Scalar Equations}

\begin{definition}[Scalar Wronskian]
For $n$ solutions $y$_{1}$(t), \ldots, y$_{n}$(t)$ of an $n$th-order linear ODE:
$$W[y$_{1}$, \ldots, y$_{n}$](t) = \begin{vmatrix}
y$_{1}$ & y$_{2}$ & \cdots & y$_{n}$ \\
y$_{1}$' & y$_{2}$' & \cdots & y$_{n}$' \\
\vdots & \vdots & \ddots & \vdots \\
y$_{1}$^{(n-1)} & y$_{2}$^{(n-1)} & \cdots & y$_{n}$^{(n-1)}
\end{vmatrix}$$
\end{definition}

\begin{example}[Second-Order Wronskian]
For two solutions $y$_{1}$, y$_{2}$$ of a second-order equation:
$$W[y$_{1}$, y$_{2}$](t) = \begin{vmatrix} y$_{1}$ & y$_{2}$ \\ y$_{1}$' & y$_{2}$' \end{vmatrix} = y$_{1}$ y$_{2}$' - y$_{2}$ y$_{1}$'$$
\end{example}

\section{Linear Independence Criteria}

\begin{theorem}[Independence Characterization]
Solutions $\mathbf{x}$_{1}$(t), \ldots, \mathbf{x}$_{n}$(t)$ are linearly independent on interval $I$ if and only if $W(t) \neq 0$ for all $t \in I$.
\end{theorem}

\begin{proposition}[Fundamental Matrix Connection]
A matrix $\Phi(t)$ is a fundamental matrix if and only if:
\begin{enumerate}
\item $\Phi'(t) = A(t)\Phi(t)$
\item $\det(\Phi(t)) \neq 0$ for all $t$
\end{enumerate}
\end{proposition}

\section{Computational Techniques}

\begin{computation}
For a $2 \times 2$ system with solutions $\mathbf{x}$_{1}$ = \begin{bmatrix} x_{11} \\ x_{21} \end{bmatrix}$, $\mathbf{x}$_{2}$ = \begin{bmatrix} x_{12} \\ x_{22} \end{bmatrix}$:
$$W(t) = x_{11}x_{22} - x_{12}x_{21}$$
\end{computation}

\begin{example}[Using Abel's Identity]
For the system $\mathbf{x}' = \begin{bmatrix} 2 & 1 \\ 0 & 3 \end{bmatrix}\mathbf{x}$:
\begin{enumerate}
\item $\text{tr}(A) = 2 + 3 = 5$
\item $W(t) = W(0) \cdot e^{5t}$
\item If $W(0) = 1$, then $W(t) = e^{5t}$
\end{enumerate}
\end{example}

\section{Special Cases and Properties}

\begin{lemma}[Constant Coefficient Systems]
For $\mathbf{x}' = A\mathbf{x}$ with constant $A$:
$$W(t) = W(0) \cdot e^{\text{tr}(A) \cdot t}$$
\end{lemma}

\begin{warning}
Common errors:
\begin{itemize}
\item Assuming $W(t) = 0$ at one point means dependent only at that point
\item Wrong row ordering in scalar Wronskian (derivatives increase downward)
\item Forgetting that $W(t)$ can be negative (it's the absolute value that matters for independence)
\item Not using Abel's identity when it simplifies calculations
\end{itemize}
\end{warning}

\section{Wronskian Applications}

\begin{theorem}[Finding Differential Equations]
If $y$_{1}$, \ldots, y$_{n}$$ are $n$ linearly independent solutions, then $y$ is also a solution if and only if:
$$W[y, y$_{1}$, \ldots, y$_{n}$](t) = 0$$
\end{theorem}

\begin{example}[Constructing ODEs]
Given solutions $e$^{t}$$ and $e^{-2t}$, find the differential equation:
$$\begin{vmatrix} y & e$^{t}$ & e^{-2t} \\ y' & e$^{t}$ & -2e^{-2t} \\ y'' & e$^{t}$ & 4e^{-2t} \end{vmatrix} = 0$$
Expanding: $y'' + y' - 2y = 0$
\end{example}

\begin{examtip}
Prof. Ditkowski's favorite Wronskian problems:
\begin{itemize}
\item Compute $W(t)$ for given solutions
\item Use Abel's identity to find $W(t)$ without direct computation
\item Test linear independence
\item Find differential equation from solutions
\item Verify all-or-nothing property
\end{itemize}
\end{examtip}

\section{Connection to Linear Algebra}

\begin{insight}
The Wronskian connects ODE theory to linear algebra:
\begin{itemize}
\item Linear independence of solutions $\Leftrightarrow$ Non-zero Wronskian
\item Fundamental matrix $\Leftrightarrow$ Wronskian never zero
\item Dimension of solution space = number of independent solutions
\item Wronskian = Volume of parallelepiped in solution space
\end{itemize}
\end{insight}

\section{Quick Wronskian Tests}

\begin{proposition}[Obvious Dependencies]
Without calculation:
\begin{itemize}
\item If $y$_{2}$ = c \cdot y$_{1}$$ (proportional), then $W = 0$
\item If solutions have different exponential growth rates, $W \neq 0$
\item For $e^{\lambda$_{1}$ t}, \ldots, e^{\lambda$_{n}$ t}$ with distinct $\lambda$_{i}$$: always independent
\end{itemize}
\end{proposition}

\end{document}