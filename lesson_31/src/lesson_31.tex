\documentclass[12pt]{article}
\usepackage{amsmath, amssymb, amsthm, tikz, pgfplots}
\usepackage{geometry, enumitem, mdframed, array, xcolor}
\geometry{margin=1in}

% Custom environments
\newtheorem{definition}{Definition}
\newtheorem{theorem}{Theorem}
\newtheorem{algorithm}{Algorithm}
\newtheorem{example}{Example}
\newmdenv[linecolor=blue,linewidth=2pt]{keypoint}
\newmdenv[linecolor=red,linewidth=2pt]{warning}
\newmdenv[linecolor=green,linewidth=2pt]{insight}
\newmdenv[linecolor=purple,linewidth=2pt]{examtip}

\title{ODE Lesson 31: Repeated Eigenvalues and Jordan Forms}
\author{ODE 1 - Prof. Adi Ditkowski}
\date{}

\begin{document}
\maketitle

\section{The Defective Eigenvalue Problem}

\begin{definition}[Algebraic and Geometric Multiplicity]
For an eigenvalue $\lambda$:
\begin{itemize}
\item \textbf{Algebraic multiplicity}: The multiplicity of $\lambda$ as a root of the characteristic polynomial
\item \textbf{Geometric multiplicity}: The dimension of the eigenspace $\ker(A - \lambda I)$
\item \textbf{Defect}: Algebraic multiplicity minus geometric multiplicity
\end{itemize}
\end{definition}

\begin{definition}[Generalized Eigenvector]
A \textbf{generalized eigenvector} of order $k$ associated with eigenvalue $\lambda$ is a vector $\mathbf{v}$ satisfying:
\[(A - \lambda I)^{k} \mathbf{v} = \mathbf{0} \quad \text{but} \quad (A - \lambda I)^{k-1} \mathbf{v} \neq \mathbf{0}\]
\end{definition}

\begin{keypoint}
\textbf{Jordan Block Structure:}
A Jordan block of size $n$ for eigenvalue $\lambda$ is:
\[J_{n}(\lambda) = \begin{pmatrix}
\lambda & 1 & 0 & \cdots & 0 \\
0 & \lambda & 1 & \cdots & 0 \\
\vdots & \vdots & \ddots & \ddots & \vdots \\
0 & 0 & 0 & \lambda & 1 \\
0 & 0 & 0 & 0 & \lambda
\end{pmatrix}\]
\end{keypoint}

\begin{algorithm}[Jordan Chain Construction]
To find a Jordan chain for repeated eigenvalue $\lambda$:
\begin{enumerate}
\item Start with an eigenvector: $(A - \lambda I)\mathbf{v}_{1} = \mathbf{0}$
\item Find generalized eigenvector: $(A - \lambda I)\mathbf{v}_{2} = \mathbf{v}_{1}$
\item Continue the chain: $(A - \lambda I)\mathbf{v}_{k+1} = \mathbf{v}_{k}$
\item Stop when you have enough vectors to match algebraic multiplicity
\end{enumerate}
\end{algorithm}

\begin{theorem}[Solution Form for Repeated Eigenvalues]
For eigenvalue $\lambda$ with a Jordan chain $\{\mathbf{v}_{1}, \mathbf{v}_{2}, \ldots, \mathbf{v}_{k}\}$:
\begin{gather}
\mathbf{x}_{1}(t) = e^{\lambda t}\mathbf{v}_{1} \\
\mathbf{x}_{2}(t) = e^{\lambda t}(t\mathbf{v}_{1} + \mathbf{v}_{2}) \\
\mathbf{x}_{3}(t) = e^{\lambda t}\left(\frac{t^{2}}{2!}\mathbf{v}_{1} + t\mathbf{v}_{2} + \mathbf{v}_{3}\right) \\
\vdots \\
\mathbf{x}_{k}(t) = e^{\lambda t}\sum_{j=0}^{k-1} \frac{t^{j}}{j!}\mathbf{v}_{k-j}
\end{gather}
\end{theorem}

\begin{keypoint}
\textbf{Why Polynomial Terms Appear:}\\
The polynomial terms arise from the nilpotent part $(A - \lambda I)$. When we write $A = \lambda I + N$ where $N = A - \lambda I$ is nilpotent:
\[e^{At} = e^{\lambda t} e^{Nt} = e^{\lambda t}\left(I + Nt + \frac{N^{2}t^2}{2!} + \cdots\right)\]
The series terminates because $N^{k} = 0$ for some $k$!
\end{keypoint}

\section{Complete Examples}

\begin{example}[$2\times2$ System with Double Eigenvalue]
Solve $\mathbf{x}' = \begin{pmatrix} 3 & 1 \\ -1 & 5 \end{pmatrix}\mathbf{x}$ with $\mathbf{x}(0) = \begin{pmatrix} 2 \\ 1 \end{pmatrix}$

\textbf{Solution:}
\begin{enumerate}
\item \textbf{Find eigenvalues:} $\det(A - \lambda I) = 0$ gives $\lambda = 3$ (double)

\item \textbf{Find eigenvector:}
$(A - 3I)\mathbf{v}_{1} = \begin{pmatrix} 0 & 1 \\ -1 & 2 \end{pmatrix}\mathbf{v}_{1} = \mathbf{0}$
This gives $\mathbf{v}_{1} = \begin{pmatrix} 1 \\ 0 \end{pmatrix}$

\item \textbf{Find generalized eigenvector:}
Solve $(A - 3I)\mathbf{w} = \mathbf{v}_{1}$:
\[\begin{pmatrix} 0 & 1 \\ -1 & 2 \end{pmatrix}\begin{pmatrix} w_{1} \\ w_{2} \end{pmatrix} = \begin{pmatrix} 1 \\ 0 \end{pmatrix}\]

From the first equation: $w_{2} = 1$. From the second: $-w_{1} + 2(1) = 0$, so $w_{1} = 2$.
Therefore $\mathbf{w} = \begin{pmatrix} 2 \\ 1 \end{pmatrix}$

\item \textbf{General Solution:}
\[\mathbf{x}(t) = c_{1} e^{3t}\begin{pmatrix} 1 \\ 0 \end{pmatrix} + c_{2} e^{3t}\left(t\begin{pmatrix} 1 \\ 0 \end{pmatrix} + \begin{pmatrix} 2 \\ 1 \end{pmatrix}\right)\]

\item \textbf{Apply Initial Conditions:}
$\mathbf{x}(0) = c_{1}\begin{pmatrix} 1 \\ 0 \end{pmatrix} + c_{2}\begin{pmatrix} 2 \\ 1 \end{pmatrix} = \begin{pmatrix} 2 \\ 1 \end{pmatrix}$

This gives: $c_{1} + 2c_{2} = 2$ and $c_{2} = 1$
Therefore $c_{1} = 0$ and $c_{2} = 1$

\item \textbf{Final Solution:}
\[\mathbf{x}(t) = e^{3t}\left(t\begin{pmatrix} 1 \\ 0 \end{pmatrix} + \begin{pmatrix} 2 \\ 1 \end{pmatrix}\right) = e^{3t}\begin{pmatrix} t + 2 \\ 1 \end{pmatrix}\]
\end{enumerate}
\end{example}

\begin{warning}
\textbf{Critical Mistakes to Avoid:}
\begin{itemize}
\item Do NOT solve $(A - \lambda I)^{2}\mathbf{v} = \mathbf{0}$ directly for generalized eigenvectors
\item The order matters in Jordan chains: always solve sequentially
\item Check that $(A - \lambda I)\mathbf{v}_{k+1} = \mathbf{v}_{k}$ is satisfied
\item The coefficient of $t^{k}$ is $1/k!$, not just $t^{k}$
\end{itemize}
\end{warning}

\begin{insight}
\textbf{Physical Interpretation:}
The polynomial terms in $t$ arise from the system's failure to diagonalize completely. The Jordan form captures the "almost diagonal" structure, and the polynomial terms represent the coupling between modes that can't be completely separated.
\end{insight}

\begin{examtip}
Prof. Ditkowski's exams typically include:
\begin{itemize}
\item One $2\times2$ system with repeated eigenvalues
\item Finding both eigenvectors and generalized eigenvectors
\item Initial value problems (most common)
\item Questions about Jordan canonical form
\item Verification of solutions
\end{itemize}
\end{examtip}

\end{document}