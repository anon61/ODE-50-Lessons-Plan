\documentclass[12pt]{article}
\usepackage{amsmath, amssymb, amsthm, tikz, pgfplots}
\usepackage{geometry, enumitem, mdframed, array, xcolor}
\usepackage{nicematrix, systeme}
\geometry{margin=1in}

% Custom environments
\newtheorem{definition}{Definition}
\newtheorem{theorem}{Theorem}
\newtheorem{algorithm}{Algorithm}
\newtheorem{example}{Example}
\newmdenv[linecolor=blue,linewidth=2pt]{keypoint}
\newmdenv[linecolor=red,linewidth=2pt]{warning}
\newmdenv[linecolor=green,linewidth=2pt]{insight}
\newmdenv[linecolor=purple,linewidth=2pt]{examtip}
\newmdenv[linecolor=orange,linewidth=2pt]{jordan}

\title{ODE Lesson 31: Repeated Eigenvalues and Jordan Forms}
\author{ODE 1 - Prof. Adi Ditkowski}
\date{}

\begin{document}
\maketitle

\section{The Defective Eigenvalue Problem}

\begin{definition}[Algebraic and Geometric Multiplicity]
For an eigenvalue $\lambda$:
\begin{itemize}
\item \textbf{Algebraic multiplicity}: The multiplicity of $\lambda$ as a root of the characteristic polynomial
\item \textbf{Geometric multiplicity}: The dimension of the eigenspace $\ker(A - \lambda I)$
\item \textbf{Defect}: Algebraic multiplicity minus geometric multiplicity
\end{itemize}
\end{definition}

\begin{definition}[Generalized Eigenvector]
A \textbf{generalized eigenvector} of order $k$ associated with eigenvalue $\lambda$ is a vector $\mathbf{v}$ satisfying:
\[(A - \lambda I)^{k} \mathbf{v} = \mathbf{0} \quad \text{but} \quad (A - \lambda I)^{k-1} \mathbf{v} \neq \mathbf{0}\]
\end{definition}

\begin{jordan}
\textbf{Jordan Block Structure:}
A Jordan block of size $n$ for eigenvalue $\lambda$ is:
\[J_{n}(\lambda) = \begin{pmatrix}
\lambda & 1 & 0 & \cdots & 0 \\
0 & \lambda & 1 & \cdots & 0 \\
\vdots & \vdots & \ddots & \ddots & \vdots \\
0 & 0 & 0 & \lambda & 1 \\
0 & 0 & 0 & 0 & \lambda
\end{pmatrix}\]
\end{jordan}

\section{Finding Generalized Eigenvectors}

\begin{algorithm}[Jordan Chain Construction]
Given eigenvalue $\lambda$ with defect $d > 0$:
\begin{enumerate}
\item Find all regular eigenvectors: $(A - \lambda I)\mathbf{v}_{1} = \mathbf{0}$
\item Find first generalized eigenvector: $(A - \lambda I)\mathbf{v}_{2} = \mathbf{v}_{1}$
\item Continue the chain: $(A - \lambda I)\mathbf{v}_{k+1} = \mathbf{v}_{k}$
\item Stop when you have enough vectors to match algebraic multiplicity
\end{enumerate}
\end{algorithm}

\begin{theorem}[Solution Form for Repeated Eigenvalues]
For eigenvalue $\lambda$ with a Jordan chain $\{\mathbf{v}_{1}, \mathbf{v}_{2}, \ldots, \mathbf{v}_{k}\}:
\begin{align}
\mathbf{x}_{1}(t) &= e^{\lambda t}\mathbf{v}_{1} \\
\mathbf{x}_{2}(t) &= $e^{\lambda t}$(t\mathbf{v}_{1} + \mathbf{v}_{2}) \\
\mathbf{x}_{3}(t) &= e^{\lambda t}\left(\frac{t^{2}}{2!}\mathbf{v}_{1} + t\mathbf{v}_{2} + \mathbf{v}_{3}\right) \\
&\vdots \\
\mathbf{x}_{k}(t) &= e^{\lambda t}\sum_{j=0}^{k-1} \frac{t^{j}}{j!}\mathbf{v}_{k-j}
\end{align}
\end{theorem}

\begin{keypoint}
\textbf{Why Polynomial Terms Appear:}\\
The polynomial terms arise from the nilpotent part (A - \lambda I)$. When we write $A = \lambda I + N$ where $N = A - \lambda I is nilpotent:
\[$e^{At}$ = $e^{\lambda t}$ e^{Nt} = e^{\lambda t}\left(I + Nt + \frac{N^{2t}^2}{2!} + \cdots\right)\]
The series terminates because N^{k} = 0$ for some $k$!
\end{keypoint}

\section{Complete Examples}

\begin{example}[2$\times$2 System with Repeated Eigenvalue]
Solve $\mathbf{x}' = \begin{pmatrix} 5 & 1 \\ -4 & 1 \end{pmatrix}\mathbf{x}$ with $\mathbf{x}(0) = \begin{pmatrix} 2 \\ 1 \end{pmatrix}$

\textbf{Solution:}
\begin{enumerate}
\item \textbf{Eigenvalues:}
\[\det(A - \lambda I) = \det\begin{pmatrix} 5-\lambda & 1 \\ -4 & 1-\lambda \end{pmatrix} = (5-\lambda)(1-\lambda) + 4\]
\[= 5 - 6\lambda + \lambda^{2} + 4 = \lambda^{2} - 6\lambda + 9 = (\lambda - 3)^{2}\]
So $\lambda = 3$ with algebraic multiplicity 2.

\item \textbf{Eigenvectors:}
\[(A - 3I) = \begin{pmatrix} 2 & 1 \\ -4 & -2 \end{pmatrix}\]
Row reduction: $R_{2} + 2R_{1} \to R_{2}$ gives $\begin{pmatrix} 2 & 1 \\ 0 & 0 \end{pmatrix}$

So $2v_{1} + v_{2} = 0 \implies v_{2} = -2v_{1}$. Choose $v_{1} = 1$: $\mathbf{v}_{1} = \begin{pmatrix} 1 \\ -2 \end{pmatrix}$

Geometric multiplicity = 1, Defect = 1.

\item \textbf{Generalized Eigenvector:}
Solve $(A - 3I)\mathbf{w} = \mathbf{v}_{1}$:
\[\begin{pmatrix} 2 & 1 \\ -4 & -2 \end{pmatrix}\begin{pmatrix} w_{1} \\ w_{2} \end{pmatrix} = \begin{pmatrix} 1 \\ -2 \end{pmatrix}\]

From row 1: $2w_{1} + w_{2} = 1$. Choose $w_{1} = 0$: $w_{2} = 1$, so $\mathbf{w} = \begin{pmatrix} 0 \\ 1 \end{pmatrix}

\item \textbf{General Solution:}
\[\mathbf{x}(t) = c_{1} e^{3t}\begin{pmatrix} 1 \\ -2 \end{pmatrix} + c_{2} e^{3t}\left(t\begin{pmatrix} 1 \\ -2 \end{pmatrix} + \begin{pmatrix} 0 \\ 1 \end{pmatrix}\right)\]
\[= e^{3t}\begin{pmatrix} c_{1} + c_{2t} \\ -2c_{1} - 2c_{2t} + c_{2} \end{pmatrix}\]

\item \textbf{Apply Initial Conditions:}
\[\begin{pmatrix} 2 \\ 1 \end{pmatrix} = \begin{pmatrix} c_{1} \\ -2c_{1} + c_{2} \end{pmatrix}\]
So c_{1} = 2$ and $-2(2) + c_{2} = 1 \implies c_{2} = 5

\item \textbf{Final Solution:}
\[\mathbf{x}(t) = e^{3t}\begin{pmatrix} 2 + 5t \\ -4 - 10t + 5 \end{pmatrix} = e^{3t}\begin{pmatrix} 2 + 5t \\ 1 - 10t \end{pmatrix}\]
\end{enumerate}
\end{example}

\begin{example}[3\times$3 System with Triple Eigenvalue]
Solve $\mathbf{x}' = \begin{pmatrix} 1 & 1 & 0 \\ 0 & 1 & 1 \\ 0 & 0 & 1 \end{pmatrix}\mathbf{x}$

\textbf{Solution:}
The matrix is upper triangular, so $\lambda = 1$ (triple).

\textbf{First eigenvector:}
$(A - I)\mathbf{v}_{1} = \begin{pmatrix} 0 & 1 & 0 \\ 0 & 0 & 1 \\ 0 & 0 & 0 \end{pmatrix}\mathbf{v}_{1} = \mathbf{0}$

This gives $v_{2} = v_{3} = 0$, so $\mathbf{v}_{1} = \begin{pmatrix} 1 \\ 0 \\ 0 \end{pmatrix}$

\textbf{First generalized eigenvector:}
$(A - I)\mathbf{v}_{2} = \mathbf{v}_{1}$:
$\begin{pmatrix} 0 & 1 & 0 \\ 0 & 0 & 1 \\ 0 & 0 & 0 \end{pmatrix}\begin{pmatrix} x \\ y \\ z \end{pmatrix} = \begin{pmatrix} 1 \\ 0 \\ 0 \end{pmatrix}$

This gives $y = 1, z = 0$, so $\mathbf{v}_{2} = \begin{pmatrix} 0 \\ 1 \\ 0 \end{pmatrix}$

\textbf{Second generalized eigenvector:}
$(A - I)\mathbf{v}_{3} = \mathbf{v}_{2}$:
$\begin{pmatrix} 0 & 1 & 0 \\ 0 & 0 & 1 \\ 0 & 0 & 0 \end{pmatrix}\begin{pmatrix} x \\ y \\ z \end{pmatrix} = \begin{pmatrix} 0 \\ 1 \\ 0 \end{pmatrix}$

This gives $y = 0, z = 1$, so \mathbf{v}_{3} = \begin{pmatrix} 0 \\ 0 \\ 1 \end{pmatrix}

\textbf{General Solution:}
\begin{align}
\mathbf{x}(t) = $e^{t}$ &\left[c_{1}\begin{pmatrix} 1 \\ 0 \\ 0 \end{pmatrix} + c_{2}\left(t\begin{pmatrix} 1 \\ 0 \\ 0 \end{pmatrix} + \begin{pmatrix} 0 \\ 1 \\ 0 \end{pmatrix}\right)\right. \\
&\left.+ c_{3}\left(\frac{t^{2}}{2}\begin{pmatrix} 1 \\ 0 \\ 0 \end{pmatrix} + t\begin{pmatrix} 0 \\ 1 \\ 0 \end{pmatrix} + \begin{pmatrix} 0 \\ 0 \\ 1 \end{pmatrix}\right)\right]
\end{align}
\end{example}

\begin{warning}
\textbf{Critical Mistakes to Avoid:}
\begin{itemize}
\item Do NOT solve (A - \lambda I)^{2}\mathbf{v} = \mathbf{0} directly for generalized eigenvectors
\item The order matters in Jordan chains: always solve sequentially
\item Check that $(A - \lambda I)\mathbf{v}_{k+1} = \mathbf{v}_{k}$ is satisfied
\item The coefficient of $t^{k}$ is $1/k!$, not just $t^{k}$
\end{itemize}
\end{warning}

\begin{insight}
\textbf{Matrix Exponential for Jordan Blocks:}
For a Jordan block $J = \lambda I + N$ where $N is the nilpotent part:
\[$e^{Jt}$ = e^{\lambda t}$e^{Nt}$ = e^{\lambda t}\begin{pmatrix}
1 & t & \frac{t^{2}}{2!} & \cdots \\
0 & 1 & t & \cdots \\
0 & 0 & 1 & \cdots \\
\vdots & \vdots & \vdots & \ddots
\end{pmatrix}\]
\end{insight}

\begin{examtip}
Prof. Ditkowski's favorite Jordan form problems:
\begin{itemize}
\item 2\times2 with double eigenvalue (very common)
\item 3\times3 upper triangular matrices
\item Finding Jordan normal form of a given matrix
\item Systems where you must identify the defect
\item Initial value problems requiring all generalized eigenvectors
\end{itemize}
\end{examtip}

\section{Solution Behavior}

For repeated eigenvalue \lambda$ with Jordan block structure:
\begin{itemize}
\item If $\lambda < 0$: Solutions decay to zero (polynomial growth dominated by exponential decay)
\item If $\lambda = 0$: Polynomial growth in time
\item If $\lambda > 0$: Exponential growth enhanced by polynomial factors
\end{itemize}

The polynomial terms cause solutions to:
\begin{itemize}
\item Initially grow even if $\lambda < 0$ (transient growth)
\item Approach eigenvector directions more slowly
\item Create more complex trajectories in phase space
\end{itemize}

\end{document}