\documentclass[12pt]{article}
\usepackage{amsmath, amssymb, amsthm, tikz, pgfplots}
\usepackage{geometry, enumitem, mdframed, array, xcolor}
\geometry{margin=1in}

% Custom environments
\newtheorem{definition}{Definition}
\newtheorem{theorem}{Theorem}
\newtheorem{corollary}{Corollary}
\newtheorem{example}{Example}
\newtheorem{remark}{Remark}
\newmdenv[linecolor=blue,linewidth=2pt]{keypoint}
\newmdenv[linecolor=red,linewidth=2pt]{warning}
\newmdenv[linecolor=green,linewidth=2pt]{insight}
\newmdenv[linecolor=purple,linewidth=2pt]{examtip}
\newmdenv[linecolor=orange,linewidth=2pt]{application}

\title{ODE Lesson 41: The Hartman-Grobman Theorem\\
\large When Linearization Works}
\author{ODE 1 - Prof. Adi Ditkowski}
\date{}

\begin{document}
\maketitle

\section{Hyperbolicity - The Key Condition}

\begin{definition}[Hyperbolic Equilibrium]
A critical point $(x_0, y_0)$ of the nonlinear system
$$\dot{x} = f(x,y), \quad \dot{y} = g(x,y)$$
is called \textbf{hyperbolic} if all eigenvalues of the Jacobian matrix $J(x_0, y_0)$ have non-zero real parts. That is, if $\lambda_1, \lambda_2$ are the eigenvalues, then:
$$\text{Re}(\lambda_1) \neq 0 \quad \text{and} \quad \text{Re}(\lambda_2) \neq 0$$
\end{definition}

\begin{keypoint}
\textbf{Hyperbolic equilibria:}
\begin{itemize}
    \item Nodes (all eigenvalues real with same sign)
    \item Saddles (real eigenvalues with opposite signs)
    \item Spirals (complex eigenvalues with non-zero real part)
\end{itemize}
\textbf{Non-hyperbolic equilibria:}
\begin{itemize}
    \item Centers (purely imaginary eigenvalues)
    \item Degenerate nodes (zero eigenvalue)
    \item Any case with $\text{Re}(\lambda) = 0$ for some eigenvalue
\end{itemize}
\end{keypoint}

\section{The Hartman-Grobman Theorem}

\begin{theorem}[Hartman-Grobman]
Let $(x_0, y_0)$ be a hyperbolic equilibrium point of the $C^1$ system:
$$\dot{x} = f(x,y), \quad \dot{y} = g(x,y)$$
Then there exists a neighborhood $U$ of $(x_0, y_0)$ and a homeomorphism $h: U \to V$ (where $V$ is a neighborhood of the origin) such that $h$ maps trajectories of the nonlinear system to trajectories of the linearized system:
$$\dot{\xi} = J(x_0, y_0) \cdot \xi$$
preserving the direction of time.
\end{theorem}

\begin{insight}
The homeomorphism $h$ provides a \textbf{topological conjugacy} between the nonlinear and linear systems. This means:
\begin{enumerate}
    \item The qualitative behavior is identical
    \item Stable/unstable manifolds correspond
    \item The phase portrait structure is preserved
    \item But geometric properties (angles, distances) may change
\end{enumerate}
\end{insight}

\section{What Hartman-Grobman Tells Us}

\subsection{What IS Preserved}

\begin{itemize}
    \item \textbf{Stability type:} Stable remains stable, unstable remains unstable
    \item \textbf{Equilibrium type:} Nodes remain nodes, saddles remain saddles, spirals remain spirals
    \item \textbf{Invariant manifolds:} Stable and unstable manifolds exist with same dimensions
    \item \textbf{Local dynamics:} The direction of flow and separation of trajectories
\end{itemize}

\subsection{What is NOT Preserved}

\begin{itemize}
    \item \textbf{Trajectory shape:} Straight lines may become curves
    \item \textbf{Time parametrization:} Speed along trajectories may change
    \item \textbf{Metric properties:} Distances and angles are not preserved
    \item \textbf{Special structures:} Hamiltonian or gradient structure may be lost
\end{itemize}

\section{Applications and Examples}

\begin{example}[Hyperbolic Saddle]
Consider the system:
$$\dot{x} = x + y^2, \quad \dot{y} = -y + x^2$$

At the origin:
$$J(0,0) = \begin{pmatrix} 1 & 0 \\ 0 & -1 \end{pmatrix}$$

Eigenvalues: $\lambda_1 = 1 > 0$, $\lambda_2 = -1 < 0$

Since both eigenvalues have non-zero real parts, the origin is hyperbolic. By Hartman-Grobman:
\begin{itemize}
    \item The origin is a saddle point for the nonlinear system
    \item There exists a 1D stable manifold (tangent to eigenvector for $\lambda = -1$)
    \item There exists a 1D unstable manifold (tangent to eigenvector for $\lambda = 1$)
    \item Near the origin, trajectories behave qualitatively like the linear saddle
\end{itemize}
\end{example}

\begin{example}[Non-hyperbolic Center]
Consider:
$$\dot{x} = -y + x(x^2 + y^2), \quad \dot{y} = x + y(x^2 + y^2)$$

At the origin:
$$J(0,0) = \begin{pmatrix} 0 & -1 \\ 1 & 0 \end{pmatrix}$$

Eigenvalues: $\lambda = \pm i$ (purely imaginary)

The origin is NOT hyperbolic. Hartman-Grobman does not apply! Indeed:
\begin{itemize}
    \item Linearization predicts a center (neutral stability)
    \item The actual nonlinear system has an unstable spiral!
    \item In polar coordinates: $\dot{r} = r^3 > 0$ for $r \neq 0$
\end{itemize}
\end{example}

\begin{warning}
\textbf{Common Exam Mistakes:}
\begin{enumerate}
    \item Applying Hartman-Grobman when eigenvalues are purely imaginary
    \item Forgetting to verify hyperbolicity before concluding
    \item Claiming exact trajectory shapes are preserved
    \item Not recognizing when additional analysis is needed
\end{enumerate}
\end{warning}

\section{The Center Problem}

When linearization yields purely imaginary eigenvalues ($\lambda = \pm i\omega$), the nonlinear system near the equilibrium could be:

\begin{center}
\begin{tabular}{|l|l|}
\hline
\textbf{Possibility} & \textbf{Determining Factor} \\
\hline
Center & Nonlinear terms preserve area/energy \\
Stable spiral & Nonlinear terms dissipate energy \\
Unstable spiral & Nonlinear terms add energy \\
More complex & Multiple timescale dynamics \\
\hline
\end{tabular}
\end{center}

\begin{examtip}
When Prof. Ditkowski gives you a system with purely imaginary eigenvalues:
\begin{enumerate}
    \item State clearly: "The equilibrium is non-hyperbolic"
    \item Write: "Hartman-Grobman theorem does not apply"
    \item Say: "Linearization alone cannot determine stability"
    \item If asked for more, use Lyapunov functions or compute higher-order terms
\end{enumerate}
\end{examtip}

\section{Structural Stability}

\begin{definition}[Structural Stability]
A system is \textbf{structurally stable} near an equilibrium if small perturbations to the system preserve the qualitative dynamics.
\end{definition}

\begin{theorem}[Consequence of Hartman-Grobman]
Hyperbolic equilibria are structurally stable. Small perturbations to $f$ and $g$ will:
\begin{itemize}
    \item Slightly move the equilibrium location
    \item Slightly change eigenvalues (keeping signs of real parts)
    \item Preserve the topological type
\end{itemize}
\end{theorem}

\begin{application}
In real-world modeling:
\begin{itemize}
    \item \textbf{Hyperbolic equilibria} are robust to modeling errors
    \item \textbf{Non-hyperbolic equilibria} are sensitive to perturbations
    \item This explains why centers are rarely observed in practice
    \item Bifurcations occur when equilibria lose hyperbolicity
\end{itemize}
\end{application}

\section{Algorithm for Applying Hartman-Grobman}

\begin{mdframed}[linecolor=black, linewidth=1pt]
\textbf{Step-by-Step Procedure:}
\begin{enumerate}
    \item Find the critical point $(x_0, y_0)$
    \item Compute the Jacobian $J(x_0, y_0)$
    \item Calculate eigenvalues $\lambda_1, \lambda_2$
    \item Check hyperbolicity:
    \begin{itemize}
        \item If $\text{Re}(\lambda_1) \neq 0$ AND $\text{Re}(\lambda_2) \neq 0$: HYPERBOLIC
        \item Otherwise: NON-HYPERBOLIC
    \end{itemize}
    \item If hyperbolic:
    \begin{itemize}
        \item State: "By Hartman-Grobman, linearization determines local behavior"
        \item Classify using linear theory
        \item Conclude about stability
    \end{itemize}
    \item If non-hyperbolic:
    \begin{itemize}
        \item State: "Hartman-Grobman does not apply"
        \item Note that additional analysis is required
        \item Consider Lyapunov functions or normal forms
    \end{itemize}
\end{enumerate}
\end{mdframed}

\section{Connection to Bifurcation Theory}

\begin{insight}
Bifurcations occur when a parameter change causes an equilibrium to lose hyperbolicity. Common scenarios:
\begin{itemize}
    \item \textbf{Saddle-node:} Real eigenvalue crosses zero
    \item \textbf{Hopf:} Complex pair crosses imaginary axis
    \item \textbf{Pitchfork/Transcritical:} Zero eigenvalue appears
\end{itemize}
At bifurcation points, Hartman-Grobman fails and nonlinear terms determine the behavior!
\end{insight}

\section{Summary Table}

\begin{center}
\begin{tabular}{|c|c|c|c|}
\hline
\textbf{Eigenvalues} & \textbf{Hyperbolic?} & \textbf{H-G Applies?} & \textbf{Conclusion} \\
\hline
$\lambda_1, \lambda_2 < 0$ & Yes & Yes & Stable node \\
$\lambda_1, \lambda_2 > 0$ & Yes & Yes & Unstable node \\
$\lambda_1 < 0 < \lambda_2$ & Yes & Yes & Saddle \\
$\alpha \pm i\beta$, $\alpha \neq 0$ & Yes & Yes & Spiral (sign of $\alpha$) \\
$\pm i\beta$ & No & No & Inconclusive \\
$\lambda_1 = 0$, $\lambda_2 \neq 0$ & No & No & Inconclusive \\
\hline
\end{tabular}
\end{center}

\end{document}