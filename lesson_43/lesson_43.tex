\documentclass[12pt]{article}
\usepackage{amsmath, amssymb, amsthm, tikz, pgfplots}
\usepackage{geometry, enumitem, mdframed, array, xcolor}
\geometry{margin=1in}

% Custom environments
\newtheorem{definition}{Definition}
\newtheorem{theorem}{Theorem}
\newtheorem{method}{Method}
\newtheorem{example}{Example}
\newmdenv[linecolor=blue,linewidth=2pt]{keypoint}
\newmdenv[linecolor=red,linewidth=2pt]{warning}
\newmdenv[linecolor=green,linewidth=2pt]{insight}
\newmdenv[linecolor=purple,linewidth=2pt]{examtip}
\newmdenv[linecolor=orange,linewidth=2pt]{formula}

\title{ODE Lesson 43: Characteristic Equation Method - All Cases}
\author{ODE 1 with Prof. Adi Ditkowski}
\date{Tel Aviv University}

\begin{document}

\maketitle

\section{Introduction}

The characteristic equation method is the fundamental technique for solving linear ordinary differential equations with constant coefficients. This method transforms the differential equation into an algebraic problem, making it both computationally efficient and theoretically elegant.

\begin{definition}[Linear ODE with Constant Coefficients]
An $n$-th order linear homogeneous ODE with constant coefficients has the form:
$$a_n y^{(n)} + a_{n-1} y^{(n-1)} + \cdots + a_1 y' + a_0 y = 0$$
where $a_0, a_1, \ldots, a_n \in \mathbb{R}$ (or $\mathbb{C}$) are constants with $a_n \neq 0$.
\end{definition}

\section{The Characteristic Equation}

\begin{theorem}[Exponential Ansatz]
For the differential equation
$$a_n y^{(n)} + a_{n-1} y^{(n-1)} + \cdots + a_1 y' + a_0 y = 0$
if we assume a solution of the form y = $e^{rt}$ where r \in \mathbb{C}, then $r$ must satisfy the characteristic equation:
$$a_n r^n + a_{n-1} r^{n-1} + \cdots + a_1 r + a_0 = 0$
\end{theorem}

\begin{proof}
Substituting y = $e^{rt}$ into the differential equation:
\begin{align}
y &= $e^{rt}$ \\
y' &= re^{rt} \\
y'' &= r^2e^{rt} \\
&\vdots \\
y^{(k)} &= r^k $e^{rt}$
\end{align}

Therefore:
$a_n r^n $e^{rt}$ + a_{n-1} r^{n-1} $e^{rt}$ + \cdots + a_1 r $e^{rt}$ + a_0 $e^{rt}$ = 0

Since $e^{rt}$ \neq 0 for all t, we can divide by $e^{rt}$:
$a_n r^n + a_{n-1} r^{n-1} + \cdots + a_1 r + a_0 = 0$$
\end{proof}

\begin{keypoint}
The characteristic equation is a polynomial equation of degree $n$. By the Fundamental Theorem of Algebra, it has exactly $n$ roots (counting multiplicities) in $\mathbb{C}$.
\end{keypoint}

\section{Solution Construction Based on Root Types}

\subsection{Case 1: Distinct Real Roots}

\begin{theorem}[Distinct Real Roots]
If the characteristic equation has $n$ distinct real roots $r_1, r_2, \ldots, r_n$, then the general solution is:
$y(t) = c_1 $e^{r_1 t}$ + c_2 $e^{r_2 t}$ + \cdots + c_n e^{r_n t}$
where $c_1, c_2, \ldots, c_n$ are arbitrary constants.
\end{theorem}

\begin{insight}
The functions \{$e^{r_1 t}$, $e^{r_2 t}$, \ldots, e^{r_n t}\} form a fundamental set of solutions. Their Wronskian is:
$W(t) = \prod_{i<j} (r_j - r_i) \cdot $e^{(r_1 + r_2 + \cdots + r_n)t}$ \neq 0$
proving linear independence.
\end{insight}

\subsection{Case 2: Complex Conjugate Roots}

\begin{theorem}[Complex Roots for Real Equations]
If the differential equation has real coefficients and $r = \alpha + i\beta$ is a complex root, then $\bar{r} = \alpha - i\beta$ is also a root. The corresponding real-valued solutions are:
$y_1(t) = $e^{\alpha t}$ \cos(\beta t), \quad y_2(t) = $e^{\alpha t}$ \sin(\beta t)
\end{theorem}

\begin{proof}
From the complex solutions $e^{(\alpha + i\beta)t}$ and $e^{(\alpha - i\beta)t}$, we use Euler's formula:
$e^{(\alpha \pm i\beta)t}$ = $e^{\alpha t}$ e^{\pm i\beta t} = $e^{\alpha t}$[\cos(\beta t) \pm i\sin(\beta t)]

Taking linear combinations:
\begin{align}
y_1 &= \frac{1}{2}[$e^{(\alpha + i\beta)t}$ + $e^{(\alpha - i\beta)t}$] = e^{\alpha t}\cos(\beta t) \\
y_2 &= \frac{1}{2i}[$e^{(\alpha + i\beta)t}$ - $e^{(\alpha - i\beta)t}$] = e^{\alpha t}\sin(\beta t)
\end{align}
\end{proof}

\subsection{Case 3: Repeated Roots (Overview)}

\begin{theorem}[Repeated Roots]
If r$ is a root of multiplicity $k$, then the $k$ linearly independent solutions are:
$e^{rt}$, \, te^{rt}, \, t^2e^{rt}, \, \ldots, \, t^{k-1}e^{rt}$
\end{theorem}

\begin{warning}
The full proof of the repeated roots case requires reduction of order or the theory of generalized eigenvectors. We will cover this in detail in Lesson 44.
\end{warning}

\section{Complete Solution Algorithm}

\begin{method}[Characteristic Equation Method]
\begin{enumerate}
    \item Write the characteristic equation by replacing $y^{(k)}$ with $r^k$.
    \item Find all roots of the characteristic polynomial.
    \item For each root, construct the corresponding solutions:
    \begin{itemize}
        \item Real root $r of multiplicity 1: $e^{rt}$
        \item Real root r$ of multiplicity $m: $e^{rt}$, te^{rt}, \ldots, t^{m-1}$e^{rt}$
        \item Complex conjugate pair \alpha \pm i\beta of multiplicity 1: e^{\alpha t}\cos(\beta t), e^{\alpha$ t}\sin(\beta t)$
        \item Complex conjugate pair $\alpha \pm i\beta$ of multiplicity $m$:
        e^{\alpha t}\cos(\beta t), te^{\alpha t}\cos(\beta t), \ldots, t^{m-1}e^{\alpha t}\cos(\beta t)$$
        e^{\alpha t}\sin(\beta t), te^{\alpha t}\sin(\beta t), \ldots, t^{m-1}e^{\alpha t}\sin(\beta t)$$
    \end{itemize}
    \item The general solution is the linear combination of all these solutions.
\end{enumerate}
\end{method}

\section{Worked Examples}

\begin{example}[Second-Order with Distinct Real Roots]
Solve: $y'' - 5y' + 6y = 0$

\textbf{Solution:}
\begin{enumerate}
    \item Characteristic equation: $r^2 - 5r + 6 = 0$
    \item Factor: $(r-2)(r-3) = 0$
    \item Roots: $r_1 = 2, r_2 = 3$
    \item General solution: y(t) = c_1 $e^{2t}$ + c_2 $e^{3t}$
\end{enumerate}
\end{example}

\begin{example}[Second-Order with Complex Roots]
Solve: y'' + 2y' + 5y = 0$

\textbf{Solution:}
\begin{enumerate}
    \item Characteristic equation: $r^2 + 2r + 5 = 0$
    \item Using quadratic formula: $r = \frac{-2 \pm \sqrt{4-20}}{2} = \frac{-2 \pm 4i}{2} = -1 \pm 2i$
    \item Roots: $r_1 = -1 + 2i, r_2 = -1 - 2i$
    \item General solution: y(t) = $e^{-t}$[c_1 \cos(2t) + c_2 \sin(2t)]
\end{enumerate}
\end{example}

\begin{example}[Third-Order Mixed Roots]
Solve: $y''' - y'' - y' + y = 0$

\textbf{Solution:}
\begin{enumerate}
    \item Characteristic equation: $r^3 - r^2 - r + 1 = 0$
    \item Factor by grouping: $r^2(r-1) - (r-1) = (r^2-1)(r-1) = (r-1)^2(r+1) = 0$
    \item Roots: $r_1 = 1$ (multiplicity 2), $r_2 = -1$
    \item General solution: y(t) = (c_1 + c_2 t)e^t + c_3 $e^{-t}$
\end{enumerate}
\end{example}

\section{Connection to Linear Systems}

\begin{insight}
Consider the second-order equation y'' + py' + qy = 0$. Converting to a system with $x_1 = y$ and $x_2 = y'$:
$$\begin{pmatrix} x_1' \\ x_2' \end{pmatrix} = \begin{pmatrix} 0 & 1 \\ -q & -p \end{pmatrix} \begin{pmatrix} x_1 \\ x_2 \end{pmatrix}$$

The characteristic polynomial of this companion matrix is:
$$\det\begin{pmatrix} -\lambda & 1 \\ -q & -p-\lambda \end{pmatrix} = \lambda^2 + p\lambda + q$$

This is exactly our characteristic equation! The roots are the eigenvalues of the companion matrix.
\end{insight}

\section{Initial Value Problems}

\begin{method}[Solving IVPs]
For an $n$-th order IVP with initial conditions $y(t_0) = y_0, y'(t_0) = y_1, \ldots, y^{(n-1)}(t_0) = y_{n-1}$:
\begin{enumerate}
    \item Find the general solution using the characteristic equation method.
    \item Apply the initial conditions to get $n$ linear equations in the $n$ unknown constants.
    \item Solve the linear system for the constants.
\end{enumerate}
\end{method}

\begin{examtip}
Prof. Ditkowski often gives initial conditions at $t = 0. This simplifies calculations since e^0 = 1, \cos(0) = 1$, and $\sin(0) = 0$.
\end{examtip}

\section{Common Errors to Avoid}

\begin{warning}
\begin{enumerate}
    \item \textbf{Sign errors:} The signs in the characteristic equation match those in the differential equation.
    \item \textbf{Complex roots:} Always convert to real form for real differential equations.
    \item \textbf{Multiplicities:} Count multiplicities carefully - $(r-2)^3 = 0$ means multiplicity 3, not 1.
    \item \textbf{Quadratic formula:} Check discriminant sign to determine root type.
    \item \textbf{Initial conditions:} Don't forget to differentiate when applying $y'(t_0) = y_1$.
\end{enumerate}
\end{warning}

\section{Solution Forms Summary Table}

\begin{center}
\begin{tabular}{|l|l|l|}
\hline
\textbf{Root Type} & \textbf{Multiplicity} & \textbf{Solution Terms} \\
\hline
Real $r & 1 & $e^{rt}$ \\
\hline
Real r$ & $k & $e^{rt}$, te^{rt}, \ldots, t^{k-1}$e^{rt}$ \\
\hline
Complex \alpha \pm i\beta & 1 & e^{\alpha t}\cos(\beta t), e^{\alpha$ t}\sin(\beta t)$ \\
\hline
Complex $\alpha \pm i\beta$ & $k$ & t^j e^{\alpha t}\cos(\beta t), t^j e^{\alpha t}\sin(\beta t) \\
& & for $j = 0, 1, \ldots, k-1$ \\
\hline
\end{tabular}
\end{center}

\end{document}