\documentclass[12pt]{article}
\usepackage{amsmath, amssymb, enumitem, geometry, xcolor}
\geometry{margin=1in}

\title{Lesson 25: Practice Problems - Orthogonal Trajectories}
\author{ODE 1 - Prof. Adi Ditkowski}
\date{}

\usepackage{mdframed}
\usepackage{tikz}
\begin{document}
\maketitle

\section*{Part A: Basic Orthogonal Trajectories (Problems 1-6)}

\begin{enumerate}
\item Find the orthogonal trajectories of the family $y = cx^{2}$.

\item Find the orthogonal trajectories of $x^{2} + y^{2} = c$.

\item Find the orthogonal trajectories of $y = c$e^{-x$}$.

\item Find the orthogonal trajectories of $xy = c$.

\item Find the orthogonal trajectories of $y^{2} = cx$.

\item Find the orthogonal trajectories of $y = c\sin x$.
\end{enumerate}

\section*{Part B: Eliminating Parameters (Problems 7-11)}

\begin{enumerate}[start=7]
\item For the family $x^{2} + y^{2} = 2cx$:
\begin{enumerate}[label=(\alph*)]
    \item Eliminate $c$ to find the differential equation
    \item Find the orthogonal trajectories
\end{enumerate}

\item For the family $y = c(x-1)^{2}$:
\begin{enumerate}[label=(\alph*)]
    \item Find the differential equation of the family
    \item Determine the orthogonal trajectories
\end{enumerate}

\item Given $(x-c)^{2} + y^{2} = c^{2}$, find orthogonal trajectories.

\item For $y^{2} = c(x+c)$, eliminate $c$ and find orthogonal curves.

\item The family $y = \tan(x + c)$ - find its orthogonal trajectories.
\end{enumerate}

\section*{Part C: Self-Orthogonal Families (Problems 12-14)}

\begin{enumerate}[start=12]
\item Show that the family of rectangular hyperbolas $x^{2} - y^{2} = c$ is self-orthogonal.

\item Prove that confocal parabolas $y^{2} = 4c(x+c)$ are self-orthogonal.

\item Find all self-orthogonal families of the form $ax^{2} + by^{2} = c$.
\end{enumerate}

\section*{Part D: Physical Applications (Problems 15-19)}

\begin{enumerate}[start=15]
\item The equipotential lines in a 2D electric field are given by $x^{2} - y^{2} = c$. Find the electric field lines.

\item Temperature distribution in a plate: $T = x^{2} + y^{2} = c$. Find the heat flow lines.

\item Stream function: $\psi = xy = c$. Find the velocity potential lines.

\item Given electric field lines $y = ce$^{x}$$, find the equipotentials.

\item In a magnetic field, the flux lines are circles $x^{2} + y^{2} = c^{2}$. What are the constant magnetic potential curves?
\end{enumerate}

\section*{Part E: Complex Orthogonal Trajectories (Problems 20-24)}

\begin{enumerate}[start=20]
\item Find orthogonal trajectories of $y = c(1 + x^{2})$.

\item Find orthogonal trajectories of the family of curves $r = c\sin\theta$ in polar coordinates.

\item For the family $e$^{x}$\cos y = c$:
\begin{enumerate}[label=(\alph*)]
    \item Find the differential equation
    \item Determine orthogonal trajectories
\end{enumerate}

\item Find orthogonal trajectories of $x^{3} + 3xy^{2} = c$.

\item The family of curves is given implicitly by $x$^{2y}$ + xy^{2} = c$. Find orthogonal trajectories.
\end{enumerate}

\section*{Part F: Exam-Style Problems (Problems 25-30)}

\begin{enumerate}[start=25]
\item (Prof. Ditkowski 2023) Show that the families $x^{2} + y^{2} = ax$ and $x^{2} + y^{2} = by$ are orthogonal, where $a$ and $b$ are parameters.

\item Given that two families of curves are orthogonal:
Family 1: $y = f(x,c)$
Family 2: $y = g(x,k)$
Prove that at any intersection point, $\frac{\partial f}{\partial x} \cdot \frac{\partial g}{\partial x} = -1$.

\item A family of curves satisfies the differential equation $\frac{dy}{dx} = \frac{2xy}{x^{2} - y^{2}}$.
\begin{enumerate}[label=(\alph*)]
    \item Find the orthogonal differential equation
    \item Show that both equations are exact
    \item Solve both and verify orthogonality
\end{enumerate}

\item In complex analysis, $f(z) = z^{3}$.
\begin{enumerate}[label=(\alph*)]
    \item Find the families $u(x,y) = c$ and $v(x,y) = k$
    \item Verify they are orthogonal trajectories
    \item Sketch several curves from each family
\end{enumerate}

\item Find all functions $F(x,y)$ such that the family $F(x,y) = c$ is self-orthogonal.

\item Two families of curves in polar coordinates are given by:
Family 1: $r = c(1 + \cos\theta)$ (cardioids)
Family 2: $r = k(1 - \cos\theta)$
\begin{enumerate}[label=(\alph*)]
    \item Are these families orthogonal?
    \item If not, find the orthogonal trajectories of Family 1
\end{enumerate}
\end{enumerate}

\section*{Solutions and Key Insights}

\textbf{Problem 1:} $y = cx^{2}$
Differentiate: $\frac{dy}{dx} = 2cx$
Eliminate $c$: From $y = cx^{2}$, we get $c = \frac{y}{x^{2}}$
So $\frac{dy}{dx} = \frac{2y}{x}$
Orthogonal: $-\frac{dx}{dy} = \frac{2y}{x}$ or $\frac{dx}{dy} = -\frac{x}{2y}$
Solving: $2y dy = -x dx$
$y^{2} = -\frac{x^{2}}{2} + C$
Result: $x^{2} + 2y^{2} = K$ (ellipses)

\textbf{Problem 4:} $xy = c$
Differentiate: $y + x\frac{dy}{dx} = 0$
So $\frac{dy}{dx} = -\frac{y}{x}$
Orthogonal: $-\frac{dx}{dy} = -\frac{y}{x}$ or $\frac{dx}{dy} = \frac{y}{x}$
This gives $x dx = y dy$
Result: $x^{2} - y^{2} = K$ (rectangular hyperbolas rotated 45°)

\textbf{Problem 7:} $x^{2} + y^{2} = 2cx$
Differentiate: $2x + 2y\frac{dy}{dx} = 2c$
From original: $c = \frac{x^{2} + y^{2}}{2x}$
Substitute: $2x + 2y\frac{dy}{dx} = \frac{x^{2} + y^{2}}{x}$
Simplify: $\frac{dy}{dx} = \frac{y^{2} - x^{2}}{2xy}$
Orthogonal: $\frac{dx}{dy} = \frac{2xy}{x^{2} - y^{2}}$

\textbf{Problem 15:} Electric field lines are orthogonal to equipotentials.
Given: $x^{2} - y^{2} = c$
Differentiate: $2x - 2y\frac{dy}{dx} = 0$
So $\frac{dy}{dx} = \frac{x}{y}$
Orthogonal (field lines): $\frac{dx}{dy} = -\frac{x}{y}$
Solving: $\frac{x dx}{y dy} = -1$, so $x dx = -y dy$
Result: $x^{2} + y^{2} = K$ (circles - radial field!)

\textbf{Key Strategy for Problem 12:} For self-orthogonal, show that the orthogonal differential equation, when solved, gives the same family with different parameter.

\textbf{Warning for Problem 28:} For $f(z) = z^{3} = (x+iy)^{3}$, expand carefully:
$u + iv = x^{3} - 3xy^{2} + i(3x$^{2y}$ - y^{3})$
So $u = x^{3} - 3xy^{2}$ and $v = 3x$^{2y}$ - y^{3}$

\textbf{Insight for Problem 25:} These are circles passing through the origin. Use the fact that circles through origin with centers on perpendicular axes are orthogonal.

\end{document}