\documentclass[12pt]{article}
\usepackage{amsmath, amssymb, amsthm, tikz, pgfplots}
\usepackage{geometry, enumitem, mdframed, array, xcolor}
\usepackage{nicematrix, systeme}
\geometry{margin=1in}

% Custom environments
\newtheorem{definition}{Definition}
\newtheorem{theorem}{Theorem}
\newtheorem{method}{Method}
\newtheorem{example}{Example}
\newmdenv[linecolor=blue,linewidth=2pt]{keypoint}
\newmdenv[linecolor=red,linewidth=2pt]{warning}
\newmdenv[linecolor=green,linewidth=2pt]{insight}
\newmdenv[linecolor=purple,linewidth=2pt]{examtip}
\newmdenv[linecolor=orange,linewidth=2pt]{matexp}

\title{ODE Lesson 33: Matrix Exponential - Computing $e^{At}$}
\author{ODE 1 - Prof. Adi Ditkowski}
\date{}

\begin{document}
\maketitle

\section{Definition and Basic Properties}

\begin{definition}[Matrix Exponential]
For an $n \times n$ matrix $A, the matrix exponential is defined by the convergent series:
\[$e^{At}$ = \sum_{k=0}^{\infty} \frac{(At)^{k}}{k!} = I + At + \frac{(At)^{2}}{2!} + \frac{(At)^{3}}{3!} + \cdots\]
\end{definition}

\begin{theorem}[Fundamental Properties]
The matrix exponential satisfies:
\begin{enumerate}
\item $e^{A \cdot 0}$ = I (initial condition)
\item \frac{d}{dt}$e^{At}$ = Ae^{At} = e^{At}A (derivative property)
\item $e^{A(s+t)}$ = e^{As}e^{At}$ (semigroup property)
\item ($e^{At}$)^{-1} = $e^{-At}$ (inverse property)
\item \det($e^{At}$) = $e^{\text{tr}$(A)t}$ (determinant formula)
\item If $AB = BA, then $e^{A+B}$ = $e^{Ae}$^B (commutativity requirement)
\end{enumerate}
\end{theorem}

\begin{matexp}
\textbf{Fundamental Solution Property:}
The unique solution to the initial value problem
\[\mathbf{x}' = A\mathbf{x}, \quad \mathbf{x}(0) = \mathbf{x}_{0}\]
is given by
\[\mathbf{x}(t) = e^{At}\mathbf{x}_{0}\]
\end{matexp}

\section{Method 1: Diagonalization}

\begin{method}[Diagonalization Method]
If A = PDP^{-1}$ where $D = \text{diag}(\lambda_{1}, \ldots, \lambda_{n}):
\[$e^{At}$ = Pe^{Dt}P^{-1}\]
where
\[$e^{Dt}$ = \begin{pmatrix}
$e^{\lambda_{1}$ t} & 0 & \cdots & 0 \\
0 & $e^{\lambda_{2}$ t} & \cdots & 0 \\
\vdots & \vdots & \ddots & \vdots \\
0 & 0 & \cdots & $e^{\lambda_{n}$ t}
\end{pmatrix}\]
\end{method}

\begin{example}[Diagonalizable 2\times2]
Compute $e^{At}$ for A = \begin{pmatrix} 1 & 2 \\ 2 & 1 \end{pmatrix}$

\textbf{Solution:}
\begin{enumerate}
\item Eigenvalues: $\det(A - \lambda I) = (1-\lambda)^{2} - 4 = \lambda^{2} - 2\lambda - 3 = 0$

$(\lambda - 3)(\lambda + 1) = 0 \Rightarrow \lambda_{1} = 3, \lambda_{2} = -1$

\item Eigenvectors:
For $\lambda_{1} = 3$: $\mathbf{v}_{1} = \begin{pmatrix} 1 \\ 1 \end{pmatrix}$
For $\lambda_{2} = -1$: $\mathbf{v}_{2} = \begin{pmatrix} 1 \\ -1 \end{pmatrix}$

\item Matrices:
$P = \begin{pmatrix} 1 & 1 \\ 1 & -1 \end{pmatrix}$, $P^{-1} = \frac{1}{2}\begin{pmatrix} 1 & 1 \\ 1 & -1 \end{pmatrix}

\item Result:
\begin{align}
$e^{At}$ &= \begin{pmatrix} 1 & 1 \\ 1 & -1 \end{pmatrix}\begin{pmatrix} $e^{3t}$ & 0 \\ 0 & $e^{-t}$ \end{pmatrix}\frac{1}{2}\begin{pmatrix} 1 & 1 \\ 1 & -1 \end{pmatrix} \\
&= \frac{1}{2}\begin{pmatrix} $e^{3t}$ + $e^{-t}$ & $e^{3t}$ - $e^{-t}$ \\ $e^{3t}$ - $e^{-t}$ & $e^{3t}$ + $e^{-t}$ \end{pmatrix}
\end{align}
\end{enumerate}
\end{example}

\section{Method 2: Jordan Form}

\begin{method}[Jordan Form Method]
For a Jordan block J_{n}(\lambda)$ of size $n:
\[$e^{J_{n}$(\lambda)t} = e^{\lambda t}\begin{pmatrix}
1 & t & \frac{t^{2}}{2!} & \cdots & \frac{t^{n-1}}{(n-1)!} \\
0 & 1 & t & \cdots & \frac{t^{n-2}}{(n-2)!} \\
0 & 0 & 1 & \cdots & \frac{t^{n-3}}{(n-3)!} \\
\vdots & \vdots & \vdots & \ddots & \vdots \\
0 & 0 & 0 & \cdots & 1
\end{pmatrix}\]
\end{method}

\begin{example}[2$\times$2 Jordan Block]
Compute $e^{At}$ for A = \begin{pmatrix} 2 & 1 \\ 0 & 2 \end{pmatrix}$

\textbf{Solution:}
This is a Jordan block with $\lambda = 2. Using the formula:
\[$e^{At}$ = e^{2t}\begin{pmatrix} 1 & t \\ 0 & 1 \end{pmatrix} = \begin{pmatrix} $e^{2t}$ & te^{2t} \\ 0 & $e^{2t}$ \end{pmatrix}\]

Verification: $e^{A \cdot 0}$ = \begin{pmatrix} 1 & 0 \\ 0 & 1 \end{pmatrix} = I \checkmark$
\end{example}

\section{Method 3: Nilpotent Matrices}

\begin{keypoint}
\textbf{Nilpotent Matrix Property:}
If $N^{k} = 0$ for some $k, then:
\[$e^{Nt}$ = I + Nt + \frac{N^{2t}^2}{2!} + \cdots + \frac{N^{k-1}t^{k-1}}{(k-1)!}\]
The series terminates!
\end{keypoint}

\begin{example}[3$\times$3 Nilpotent]
Compute $e^{At}$ for A = \begin{pmatrix} 0 & 1 & 0 \\ 0 & 0 & 1 \\ 0 & 0 & 0 \end{pmatrix}$

\textbf{Solution:}
$A^{2} = \begin{pmatrix} 0 & 0 & 1 \\ 0 & 0 & 0 \\ 0 & 0 & 0 \end{pmatrix}$, $A^{3} = 0

Therefore:
\[$e^{At}$ = I + At + \frac{A^{2t}^2}{2} = \begin{pmatrix} 1 & t & \frac{t^{2}}{2} \\ 0 & 1 & t \\ 0 & 0 & 1 \end{pmatrix}\]
\end{example}

\section{Method 4: Cayley-Hamilton}

\begin{theorem}[Cayley-Hamilton Application]
For a 2\times2 matrix with characteristic polynomial p(\lambda) = \lambda^{2} - \text{tr}(A)\lambda + \det(A)$:
\begin{itemize}
\item If $\lambda_{1} \neq \lambda_{2}: $e^{At}$ = \frac{$e^{\lambda_{1}$ t} - $e^{\lambda_{2}$ t}}{\lambda_{1} - \lambda_{2}}A + \frac{\lambda_{1} $e^{\lambda_{2}$ t} - \lambda_{2} $e^{\lambda_{1}$ t}}{\lambda_{1} - \lambda_{2}}I$
\item If $\lambda_{1} = \lambda_{2} = \lambda: $e^{At}$ = $e^{\lambda t}$[I + t(A - \lambda I)]
\end{itemize}
\end{theorem}

\section{Method 5: Complex Eigenvalues}

\begin{example}[Complex Eigenvalues]
Compute $e^{At}$ for A = \begin{pmatrix} 0 & -1 \\ 1 & 0 \end{pmatrix}$

\textbf{Solution:}
Eigenvalues: $\lambda = \pm i

Using the formula for complex eigenvalues:
\[$e^{At}$ = \begin{pmatrix} \cos t & -\sin t \\ \sin t & \cos t \end{pmatrix}\]

This is a rotation matrix!
\end{example}

\begin{insight}
\textbf{Computational Strategies:}
\begin{itemize}
\item Diagonal matrix: Use $e^{Dt}$ directly
\item Diagonalizable: Use Pe^{Dt}P^{-1}
\item Jordan blocks: Use Jordan exponential formula
\item Nilpotent: Truncate the series
\item 2\times2: Use direct formulas
\item Complex eigenvalues: Get rotation matrices
\end{itemize}
\end{insight}

\begin{warning}
\textbf{Common Computational Errors:}
\begin{itemize}
\item Matrix multiplication is NOT commutative
\item $e^{A+B}$ \neq $e^{Ae}$^B unless AB = BA
\item Don't forget factorials in the series
\item Jordan block exponentials have specific patterns
\item Always verify $e^{A \cdot 0}$ = I
\end{itemize}
\end{warning}

\begin{examtip}
Prof. Ditkowski's typical problems:
\begin{itemize}
\item Compute $e^{At}$ for 2\times2 diagonal matrices
\item Find $e^{At}$ for 2\times2 Jordan blocks
\item Use diagonalization for 2\times2 systems
\item Verify properties of matrix exponential
\item Apply $e^{At}$ to solve IVPs
\end{itemize}
\end{examtip}

\section{Quick Reference Table}

\begin{center}
\begin{tabular}{|l|c|}
\hline
\textbf{Matrix Type} & \textbf{Formula for $e^{At}$} \\
\hline
Diagonal D = \text{diag}(\lambda_{i})$ & \text{diag}($e^{\lambda_{i}$ t}) \\
\hline
Diagonalizable $A = PDP^{-1}$ & Pe^{Dt}P^{-1} \\
\hline
2\times2 Jordan block $\begin{pmatrix} \lambda & 1 \\ 0 & \lambda \end{pmatrix} & e^{\lambda t}\begin{pmatrix} 1 & t \\ 0 & 1 \end{pmatrix} \\
\hline
Nilpotent N^{k} = 0$ & $\sum_{j=0}^{k-1} \frac{N^{j} t^{j}}{j!}$ \\
\hline
Rotation $\begin{pmatrix} 0 & -\omega \\ \omega & 0 \end{pmatrix}$ & $\begin{pmatrix} \cos(\omega t) & -\sin(\omega t) \\ \sin(\omega t) & \cos(\omega t) \end{pmatrix}$ \\
\hline
\end{tabular}
\end{center}

\end{document}