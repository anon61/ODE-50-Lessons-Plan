\documentclass[12pt]{article}
\usepackage{amsmath, amssymb, amsthm, tikz, pgfplots}
\usepackage{geometry, enumitem, mdframed, array, xcolor}
\geometry{margin=1in}

% Custom environments
\newtheorem{definition}{Definition}
\newtheorem{theorem}{Theorem}
\newtheorem{method}{Method}
\newtheorem{example}{Example}
\newmdenv[linecolor=blue,linewidth=2pt]{keypoint}
\newmdenv[linecolor=red,linewidth=2pt]{warning}
\newmdenv[linecolor=green,linewidth=2pt]{insight}
\newmdenv[linecolor=purple,linewidth=2pt]{examtip}
\newmdenv[linecolor=cyan,linewidth=2pt]{algorithm}

\title{ODE Lesson 24: Special Integrating Factors - $\mu(xy)$ and Beyond}
\author{ODE 1 - Prof. Adi Ditkowski}
\date{}

\begin{document}
\maketitle

\section{Beyond Simple Integrating Factors}

When neither $\mu(x)$ nor $\mu(y)$ exists, we look for integrating factors of special forms:
\begin{itemize}
    \item $\mu(xy)$ - depends on the product $xy$
    \item $\mu(x^2 + y^2)$ - depends on radial distance
    \item $\mu = x^a y^b$ - power form
    \item $\mu = 1/(xM + yN)$ - for homogeneous equations
\end{itemize}

\begin{keypoint}
Each special form has its own existence test. Master the tests to quickly identify which form to use!
\end{keypoint}

\section{Case 3: Integrating Factor $\mu(xy)$}

\begin{theorem}[Existence of $\mu(xy)$]
An integrating factor depending only on $z = xy$ exists if and only if
\[\frac{1}{xN - yM}\left(\frac{\partial M}{\partial y} - \frac{\partial N}{\partial x}\right) = F(xy)\]
where $F$ is a function of $xy alone. The integrating factor is:
\[\mu(xy) = $e^{\int F(z)\,dz}$ \text{ where } z = xy\]
\end{theorem}

\begin{proof}[Proof Sketch]
If \mu = \mu(z)$ where $z = xy$, then:
\[\frac{\partial \mu}{\partial x} = \frac{d\mu}{dz} \cdot y, \quad \frac{\partial \mu}{\partial y} = \frac{d\mu}{dz} \cdot x\]

Substituting into the exactness condition:
\[M \cdot \frac{d\mu}{dz} \cdot x - N \cdot \frac{d\mu}{dz} \cdot y = \mu(N_x - M_y)\]
\[\frac{1}{\mu}\frac{d\mu}{dz} = \frac{M_y - N_x}{xN - yM}\]

This requires the right side to be a function of $z = xy$ only.
\end{proof}

\begin{algorithm}
\textbf{Testing for $\mu(xy)$:}
\begin{enumerate}
    \item Compute $M_y - N_x$ (numerator)
    \item Compute $xN - yM$ (denominator - note the pattern!)
    \item Form the ratio $\frac{M_y - N_x}{xN - yM}$
    \item Check if this can be expressed as $F(xy)$
    \item If yes, solve $\frac{d\mu}{dz} = \mu \cdot F(z)$ where $z = xy$
\end{enumerate}
\end{algorithm}

\begin{example}[$\mu(xy)$ Application]
Solve $(2y^2 + 3xy)dx + (2xy + x^2)dy = 0$

\textbf{Step 1:} Check exactness: $M_y = 4y + 3x$, $N_x = 2y + 2x$. Not exact.

\textbf{Step 2:} Test $\mu(x)$: $\frac{M_y - N_x}{N} = \frac{2y + x}{2xy + x^2} = \frac{2y + x}{x(2y + x)} = \frac{1}{x}$

Actually, $\mu(x) = x$ works here! But let's continue for illustration...

\textbf{Step 3:} Test $\mu(xy)$:
\[\frac{M_y - N_x}{xN - yM} = \frac{2y + x}{x(2xy + x^2) - y(2y^2 + 3xy)}\]
\[= \frac{2y + x}{2x^2y + x^3 - 2y^3 - 3xy^2}\]

This expression is complex and not a clear function of $xy$.
\end{example}

\section{Case 4: Integrating Factor $\mu(x^2 + y^2)$}

\begin{theorem}[Existence of $\mu(x^2 + y^2)$]
An integrating factor depending only on $r^2 = x^2 + y^2$ exists if and only if
\[\frac{1}{xM + yN}\left(\frac{\partial M}{\partial y} - \frac{\partial N}{\partial x}\right) = G(x^2 + y^2)\]
where $G$ is a function of $x^2 + y^2$ alone.
\end{theorem}

\begin{insight}
The denominator $xM + yN$ appears naturally in polar coordinate transformations. This test often succeeds for equations with circular symmetry.
\end{insight}

\begin{example}[Radial Integrating Factor]
Consider $(x^2 + y^2 + x)dx + (x^2 + y^2 + y)dy = 0$

Let's test for $\mu(x^2 + y^2)$:
\[M_y - N_x = 2y - 2x\]
\[xM + yN = x(x^2 + y^2 + x) + y(x^2 + y^2 + y) = (x^2 + y^2)(x + y) + x^2 + y^2\]
\[= (x^2 + y^2)(x + y + 1)\]

The ratio doesn't simplify to a function of $x^2 + y^2$ easily, but if we had $(x^2 + y^2)$ as a common factor throughout, it would work.
\end{example}

\section{Case 5: Power Form $\mu = x^a y^b$}

\begin{method}[Finding $\mu = x^a y^b$]
To find an integrating factor of the form $\mu = x^a y^b$:
\begin{enumerate}
    \item Multiply the equation by $x^a y^b$
    \item Apply the exactness condition
    \item Compare powers of $x$ and $y$ on both sides
    \item Solve the resulting system for $a$ and $b$
\end{enumerate}
\end{method}

\begin{example}[Power Form]
Find $\mu = x^a y^b$ for $y dx + 2x dy = 0$

After multiplication: $x^a y^{b+1} dx + 2x^{a+1} y^b dy = 0$

Exactness requires:
\[\frac{\partial}{\partial y}(x^a y^{b+1}) = \frac{\partial}{\partial x}(2x^{a+1} y^b)\]
\[(b+1)x^a y^b = 2(a+1)x^a y^b\]
\[b + 1 = 2(a + 1)\]
\[b = 2a + 1\]

Choosing $a = -1$ gives $b = -1$, so $\mu = \frac{1}{xy}$ works.
\end{example}

\section{Case 6: Homogeneous Equations}

\begin{theorem}[Integrating Factor for Homogeneous Equations]
If $M(x,y)$ and $N(x,y)$ are homogeneous functions of the same degree $n$, then
\[\mu = \frac{1}{xM + yN}\]
is an integrating factor (provided $xM + yN \neq 0$).
\end{theorem}

\begin{proof}[Proof Outline]
Using Euler's theorem for homogeneous functions of degree $n$:
\[x\frac{\partial M}{\partial x} + y\frac{\partial M}{\partial y} = nM\]
\[x\frac{\partial N}{\partial x} + y\frac{\partial N}{\partial y} = nN\]

The exactness condition with $\mu = 1/(xM + yN)$ can be verified using these relations.
\end{proof}

\begin{keypoint}
\textbf{Quick Recognition of Homogeneous Equations:}
\begin{itemize}
    \item All terms have the same total degree in $x$ and $y$
    \item $M(tx, ty) = t^n M(x,y)$ for some $n$
    \item Common forms: rational functions where numerator and denominator have same degree
\end{itemize}
\end{keypoint}

\section{Strategy Flowchart}

\begin{algorithm}
\textbf{Complete Integrating Factor Strategy:}
\begin{enumerate}
    \item Test for exactness - if exact, solve directly
    \item Test for $\mu(x)$: Is $(M_y - N_x)/N$ a function of $x$ only?
    \item Test for $\mu(y)$: Is $(N_x - M_y)/M$ a function of $y$ only?
    \item Check for homogeneity - if yes, use $\mu = 1/(xM + yN)$
    \item Test for $\mu(xy)$: Is $(M_y - N_x)/(xN - yM)$ a function of $xy$?
    \item Test for $\mu(x^2+y^2)$: Is $(M_y - N_x)/(xM + yN)$ a function of $x^2+y^2$?
    \item Try $\mu = x^a y^b$ by comparing powers
    \item Look for patterns or use inspection
\end{enumerate}
\end{algorithm}

\section{Memory Aids and Patterns}

\begin{examtip}
\textbf{Denominator Patterns:}
\begin{center}
\begin{tabular}{|l|c|l|}
\hline
\textbf{Form} & \textbf{Test Denominator} & \textbf{Mnemonic} \\
\hline
$\mu(x)$ & $N$ & "N for x" \\
$\mu(y)$ & $M$ & "M for y" \\
$\mu(xy)$ & $xN - yM$ & "Cross product" \\
$\mu(x^2+y^2)$ & $xM + yN$ & "Dot product" \\
Homogeneous & $xM + yN$ & "Euler's friend" \\
\hline
\end{tabular}
\end{center}
\end{examtip}

\section{Common Exam Patterns}

\begin{warning}
Prof. Ditkowski often gives hints about the form:
\begin{itemize}
    \item "Find an integrating factor of the form $x^a y^b$"
    \item "Show that the equation has an integrating factor depending on $xy$"
    \item "Find $\mu$ assuming it depends only on $x^2 + y^2$"
\end{itemize}
When you see these hints, skip the testing phase and work directly with the given form!
\end{warning}

\end{document}