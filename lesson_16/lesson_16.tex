\documentclass[12pt]{article}
\usepackage{amsmath, amssymb, amsthm, tikz, pgfplots}
\usepackage{geometry, enumitem, mdframed, array, xcolor}
\geometry{margin=1in}

% Custom environments
\newtheorem{definition}{Definition}
\newtheorem{theorem}{Theorem}
\newtheorem{method}{Method}
\newtheorem{example}{Example}
\newmdenv[linecolor=blue,linewidth=2pt]{keypoint}
\newmdenv[linecolor=red,linewidth=2pt]{warning}
\newmdenv[linecolor=green,linewidth=2pt]{insight}
\newmdenv[linecolor=purple,linewidth=2pt]{examtip}
\newmdenv[linecolor=orange,linewidth=2pt]{formula}

\title{Lesson 16: Variation of Constants for First-Order ODEs}
\author{ODE 1 - Prof. Adi Ditkowski}
\date{Tel Aviv University}

\begin{document}
\maketitle

\section{Introduction and Motivation}

\begin{definition}[Variation of Constants Method]
For the linear first-order ODE
\[y' + p(t)y = g(t)\]
the method of variation of constants seeks a particular solution by allowing the constant in the homogeneous solution to vary with time.
\end{definition}

\begin{keypoint}
The variation of constants method works for ANY continuous forcing function $g(t)$, unlike undetermined coefficients which requires specific forms.
\end{keypoint}

\section{Theoretical Development}

\subsection{The Fundamental Idea}

\begin{theorem}[Variation of Constants Formula]
Given the linear ODE $y' + p(t)y = g(t)$ with homogeneous solution $y_h = Ce^{-\int p(t)dt}$, a particular solution is:
\[y_p = y_h(t) \int \frac{g(t)}{y_h(t)} dt\]
\end{theorem}

\begin{proof}
Assume $y = C(t) \cdot y_h(t)$ where $y_h$ solves $y_h' + p(t)y_h = 0$.

Differentiating:
\[y' = C'(t)y_h(t) + C(t)y_h'(t)\]

Substituting into the original equation:
\[C'(t)y_h(t) + C(t)y_h'(t) + p(t)C(t)y_h(t) = g(t)\]

Since $y_h' + p(t)y_h = 0$, we have $y_h' = -p(t)y_h$:
\[C'(t)y_h(t) + C(t)(-p(t)y_h(t)) + p(t)C(t)y_h(t) = g(t)\]
\[C'(t)y_h(t) = g(t)\]

Therefore:
\[C'(t) = \frac{g(t)}{y_h(t)}\]
\[C(t) = \int \frac{g(t)}{y_h(t)} dt + K\]

The general solution is:
\[y = y_h(t) \left(\int \frac{g(t)}{y_h(t)} dt + K\right) = y_p + Ky_h\]
\end{proof}

\subsection{Connection to Integrating Factor}

\begin{insight}
The variation of constants formula is equivalent to the integrating factor method:
\[\mu(t) = e^{\int p(t)dt}, \quad y_h = \frac{C}{\mu(t)}\]
\[y_p = \frac{1}{\mu(t)} \int \mu(t)g(t)dt\]
\end{insight}

\section{Solution Algorithm}

\begin{method}[Step-by-Step Procedure]
\begin{enumerate}
    \item \textbf{Identify} the equation in standard form: $y' + p(t)y = g(t)$
    \item \textbf{Solve} the homogeneous equation: $y_h = Ce^{-\int p(t)dt}$
    \item \textbf{Set up} the variation: $y = C(t) \cdot y_h(t)$
    \item \textbf{Differentiate} and substitute to find: $C'(t) = \frac{g(t)}{y_h(t)}$
    \item \textbf{Integrate} to find $C(t)$: $C(t) = \int \frac{g(t)}{y_h(t)}dt + K$
    \item \textbf{Construct} the general solution: $y = C(t) \cdot y_h(t)$
    \item \textbf{Apply} initial conditions if given
\end{enumerate}
\end{method}

\section{Worked Examples}

\begin{example}[Exponential Forcing]
Solve $y' - 3y = e^{5t}$ with $y(0) = 2$.

\textbf{Solution:}
\begin{enumerate}
    \item Homogeneous solution: $y_h = Ce^{3t}$
    \item Variation setup: $y = C(t)e^{3t}$
    \item Finding $C'(t)$:
    \[C'(t)e^{3t} = e^{5t} \Rightarrow C'(t) = e^{2t}\]
    \item Integrating:
    \[C(t) = \frac{1}{2}e^{2t} + K\]
    \item General solution:
    \[y = \left(\frac{1}{2}e^{2t} + K\right)e^{3t} = \frac{1}{2}e^{5t} + Ke^{3t}\]
    \item Apply initial condition:
    \[2 = \frac{1}{2} + K \Rightarrow K = \frac{3}{2}\]
    \item Final solution:
    \[y = \frac{1}{2}e^{5t} + \frac{3}{2}e^{3t}\]
\end{enumerate}
\end{example}

\begin{example}[Non-Standard Forcing]
Solve $y' + \frac{2}{t}y = t\ln(t)$ for $t > 0$.

\textbf{Solution:}
\begin{enumerate}
    \item Homogeneous solution: $y_h = \frac{C}{t^2}$
    \item Variation: $y = \frac{C(t)}{t^2}$
    \item Finding $C'(t)$:
    \[\frac{C'(t)}{t^2} = t\ln(t) \Rightarrow C'(t) = t^3\ln(t)\]
    \item Integration by parts:
    \[C(t) = \int t^3\ln(t)dt = \frac{t^4\ln(t)}{4} - \frac{t^4}{16} + K\]
    \item General solution:
    \[y = \frac{t^2\ln(t)}{4} - \frac{t^2}{16} + \frac{K}{t^2}\]
\end{enumerate}
\end{example}

\section{Advantages and Applications}

\begin{keypoint}
\textbf{Advantages over Undetermined Coefficients:}
\begin{itemize}
    \item Works for any continuous $g(t)$
    \item No need to guess solution form
    \item Systematic procedure always succeeds
    \item Extends naturally to higher-order equations
    \item Provides Green's function interpretation
\end{itemize}
\end{keypoint}

\begin{warning}
\textbf{Common Errors:}
\begin{itemize}
    \item Incorrect homogeneous solution
    \item Forgetting to simplify before integration
    \item Missing the constant of integration
    \item Not including full general solution
    \item Sign errors in exponential arguments
\end{itemize}
\end{warning}

\section{Physical Interpretation}

\begin{insight}
The variation of constants represents:
\begin{itemize}
    \item \textbf{RC Circuit:} Time-varying charge accumulation
    \item \textbf{Population Model:} Variable immigration/emigration
    \item \textbf{Heat Transfer:} Time-dependent source term
    \item \textbf{Mechanical System:} External forcing modulation
\end{itemize}

The integral $\int \frac{g(t)}{y_h(t)}dt$ accumulates the forcing effect, weighted inversely by the system's natural response.
\end{insight}

\section{Connection to Green's Functions}

\begin{definition}[Green's Function Preview]
The solution can be written as:
\[y(t) = y_h(t)y(t_0)/y_h(t_0) + \int_{t_0}^t G(t,s)g(s)ds\]
where $G(t,s) = y_h(t)/y_h(s)$ for $s \leq t$ is the Green's function.
\end{definition}

\section{Exam Strategy}

\begin{examtip}
\textbf{Prof. Ditkowski's Exam Focus:}
\begin{enumerate}
    \item Always show the homogeneous solution first
    \item Explicitly write $C'(t) = g(t)/y_h(t)$
    \item Simplify integrands before integrating
    \item State the general solution as $y = y_p + y_h$
    \item Verify solution by substitution if time permits
    \item Use variation for non-standard forcing functions
\end{enumerate}
\end{examtip}

\begin{formula}
\textbf{Key Formulas to Memorize:}
\begin{align}
y_h &= Ce^{-\int p(t)dt} \\
C'(t) &= \frac{g(t)}{y_h(t)} \\
y_p &= y_h(t) \int \frac{g(t)}{y_h(t)}dt \\
y &= y_p + Cy_h
\end{align}
\end{formula}

\end{document}