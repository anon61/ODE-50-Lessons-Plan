\documentclass[12pt]{article}
\usepackage{amsmath, amssymb, amsthm, tikz, pgfplots}
\usepackage{geometry, enumitem, mdframed, array, xcolor}
\geometry{margin=1in}

\newtheorem{definition}{Definition}
\newtheorem{theorem}{Theorem}
\newtheorem{method}{Method}
\newtheorem{example}{Example}
\newtheorem{lemma}{Lemma}
\newmdenv[linecolor=blue,linewidth=2pt]{keypoint}
\newmdenv[linecolor=red,linewidth=2pt]{warning}
\newmdenv[linecolor=green,linewidth=2pt]{insight}
\newmdenv[linecolor=purple,linewidth=2pt]{examtip}
\newmdenv[linecolor=orange,linewidth=2pt]{transformation}

\title{Lesson 19: Riccati Equations with Known Particular Solution}
\author{ODE 1 - Prof. Adi Ditkowski}
\date{}

\begin{document}
\maketitle

\section{Definition and Recognition}

\begin{definition}[Riccati Equation]
A Riccati differential equation has the form:
$$\frac{dy}{dx} = q_0(x) + q_1(x)y + q_2(x)y^2$$
where $q_0(x)$, $q_1(x)$, and $q_2(x)$ are continuous functions and $q_2(x) \not\equiv 0$.
\end{definition}

\begin{keypoint}
\textbf{Special Cases:}
\begin{itemize}
    \item If $q_2(x) \equiv 0$: Linear equation
    \item If $q_0(x) \equiv 0$: Bernoulli equation with $n = 2$
    \item If $q_1(x) \equiv 0$: Separable after substitution
    \item General case: Requires particular solution or transformation to second-order
\end{itemize}
\end{keypoint}

\section{The Fundamental Transformation}

\begin{theorem}[Riccati to Bernoulli Reduction]
If $y_p(x)$ is a particular solution of the Riccati equation, then the substitution
$$y = y_p + v$$
transforms the Riccati equation into the Bernoulli equation:
$$\frac{dv}{dx} = (q_1 + 2q_2y_p)v + q_2v^2$$
\end{theorem}

\begin{proof}
Given: $y' = q_0 + q_1y + q_2y^2$ and $y_p' = q_0 + q_1y_p + q_2y_p^2$

Substitute $y = y_p + v$:
\begin{align}
y_p' + v' &= q_0 + q_1(y_p + v) + q_2(y_p + v)^2 \\
y_p' + v' &= q_0 + q_1y_p + q_1v + q_2y_p^2 + 2q_2y_pv + q_2v^2 \\
v' &= q_1v + 2q_2y_pv + q_2v^2 \quad \text{(using } y_p' = q_0 + q_1y_p + q_2y_p^2\text{)} \\
v' &= (q_1 + 2q_2y_p)v + q_2v^2 \quad \square
\end{align}
\end{proof}

\begin{transformation}
\textbf{Complete Solution Algorithm:}
\begin{enumerate}
    \item Find or verify particular solution $y_p$
    \item Substitute $y = y_p + v$
    \item Obtain Bernoulli equation: $v' = (q_1 + 2q_2y_p)v + q_2v^2$
    \item Use substitution $w = v^{-1}$ (since $n = 2$)
    \item Solve linear equation: $w' = -(q_1 + 2q_2y_p)w - q_2$
    \item Back-substitute: $v = 1/w$, then $y = y_p + v$
\end{enumerate}
\end{transformation}

\section{Finding Particular Solutions}

\begin{method}[Inspection Techniques]
\textbf{Common forms to try:}
\begin{enumerate}
    \item \textbf{Constants:} Try $y_p = c$ when coefficients allow
    \item \textbf{Linear:} Try $y_p = ax + b$ for polynomial coefficients
    \item \textbf{Rational:} Try $y_p = a/x$ or $y_p = a/(x+b)$
    \item \textbf{Exponential:} Try $y_p = ae^{bx}$ for constant coefficients
    \item \textbf{Trigonometric:} Try $y_p = a\tan(bx)$ or $a\cot(bx)$
    \item \textbf{Special:} $y_p = -q_1/(2q_2)$ when this ratio is constant
\end{enumerate}
\end{method}

\begin{example}[Polynomial Particular Solution]
Solve: $y' = \frac{2}{x^2} - \frac{2y}{x} + y^2$

\textbf{Finding $y_p$:} Try $y_p = \frac{a}{x}$
$$-\frac{a}{x^2} = \frac{2}{x^2} - \frac{2a}{x^2} + \frac{a^2}{x^2}$$
$$-a = 2 - 2a + a^2 \implies a^2 - a + 2 = 0$$

This gives $a = 2$ or $a = -1$. Use $y_p = \frac{2}{x}$.

\textbf{Transformation:} Let $y = \frac{2}{x} + v$
$$v' = -\frac{2v}{x} + \frac{4v}{x} + v^2 = \frac{2v}{x} + v^2$$

\textbf{Bernoulli to Linear:} Let $w = v^{-1}$
$$w' = -\frac{2w}{x} - 1$$

\textbf{Solution:} $w = \frac{C}{x^2} - \frac{x}{3}$

\textbf{Final Answer:} $y = \frac{2}{x} + \frac{1}{C/x^2 - x/3}$
\end{example}

\section{Special Riccati Forms}

\begin{insight}
\textbf{Constant Coefficient Riccati:} $y' = a + by + cy^2$
\begin{itemize}
    \item If $b^2 - 4ac > 0$: Two constant particular solutions
    \item If $b^2 - 4ac = 0$: One constant particular solution
    \item If $b^2 - 4ac < 0$: No real constant solutions
\end{itemize}
For $b^2 - 4ac < 0$, try $y_p = \alpha \tan(\beta x)$ where $\beta = \sqrt{4ac - b^2}/(2c)$
\end{insight}

\begin{example}[Trigonometric Particular Solution]
Solve: $y' = 1 + y^2$

\textbf{Observation:} This matches $\frac{d}{dx}[\tan x] = \sec^2 x = 1 + \tan^2 x$

\textbf{Particular solution:} $y_p = \tan x$

\textbf{General solution:} Let $y = \tan x + v$
$$v' = 2\tan x \cdot v + v^2$$

After solving the Bernoulli equation:
$$y = \tan x + \frac{\sin x}{C - \cos x}$$
\end{example}

\section{Connection to Linear Second-Order}

\begin{theorem}[Riccati-Linear Duality]
The Riccati equation $y' = q_0 + q_1y + q_2y^2$ is equivalent to the second-order linear equation:
$$u'' - (q_1 + \frac{q_2'}{q_2})u' + q_0q_2 \cdot u = 0$$
via the transformation $y = -\frac{1}{q_2} \cdot \frac{u'}{u}$
\end{theorem}

\begin{examtip}
\textbf{Prof. Ditkowski's Patterns:}
\begin{itemize}
    \item Often provides $y_p$ or strong hints ("verify that...")
    \item Tests connection to Bernoulli reduction
    \item Likes rational particular solutions $y_p = a/x$
    \item May ask for multiple particular solutions
    \item Tests the second-order connection
    \item Partial credit for correct transformation setup
\end{itemize}
\end{examtip}

\section{Geometric Interpretation}

\begin{lemma}[Cross-Ratio Property]
If $y_1, y_2, y_3, y_4$ are four solutions of a Riccati equation, their cross-ratio:
$$\frac{(y_1 - y_3)(y_2 - y_4)}{(y_1 - y_4)(y_2 - y_3)}$$
is constant (independent of $x$).
\end{lemma}

\begin{warning}
\textbf{Common Pitfalls:}
\begin{itemize}
    \item Not verifying that $y_p$ satisfies the equation
    \item Sign errors in the transformation to Bernoulli
    \item Forgetting that Bernoulli with $n = 2$ needs $w = v^{-1}$
    \item Missing singular solutions when $v = 0$
\end{itemize}
\end{warning}

\section{Solution Structure}

\begin{keypoint}
\textbf{General Solution Form:}
$$y = y_p + \frac{1}{w(x)}$$
where $w(x)$ satisfies the linear equation:
$$w' + (q_1 + 2q_2y_p)w = -q_2$$

The general solution has the structure:
$$y = y_p + \frac{1}{\phi(x) + C\psi(x)}$$
where $\phi$ and $\psi$ depend on the particular solution chosen.
\end{keypoint}

\end{document}