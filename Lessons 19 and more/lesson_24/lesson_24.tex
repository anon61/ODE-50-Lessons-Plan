\documentclass[12pt]{article}
\usepackage{amsmath, amssymb, amsthm, enumitem, geometry, mdframed, xcolor}
\geometry{margin=1in}

\newtheorem{definition}{Definition}
\newtheorem{theorem}{Theorem}
\newtheorem{method}{Method}
\newtheorem{example}{Example}
\newmdenv[linecolor=blue,linewidth=2pt]{keypoint}
\newmdenv[linecolor=red,linewidth=2pt]{warning}
\newmdenv[linecolor=green,linewidth=2pt]{insight}
\newmdenv[linecolor=purple,linewidth=2pt]{examtip}
\newmdenv[linecolor=cyan,linewidth=2pt]{algorithm}

\title{Lesson 24: Special Integrating Factors - $\mu(xy)$ and Beyond}
\author{ODE 1 - Prof. Adi Ditkowski}
\date{}

\begin{document}
\maketitle

\section{Beyond Simple Integrating Factors}

When neither $\mu(x)$ nor $\mu(y)$ exists, we need to consider more sophisticated integrating factors.

\section{Integrating Factor of the Form $\mu(xy)$}

\begin{theorem}[Test for $\mu(xy)$]
If $\frac{\partial M/\partial y - \partial N/\partial x}{xN - yM}$ depends only on the product $xy$, then there exists an integrating factor $\mu(xy)$ where $z = xy$.
\end{theorem}

\begin{algorithm}
\textbf{Finding $\mu(xy)$:}
\begin{enumerate}
    \item Calculate $R = \frac{M_y - N_x}{xN - yM}$
    \item Check if $R$ depends only on $xy$ (substitute $z = xy$)
    \item If yes, solve $\frac{d\mu}{dz} = \mu \cdot R(z)$
    \item The integrating factor is $\mu(xy)$
\end{enumerate}
\end{algorithm}

\section{Other Special Forms}

\begin{keypoint}
\textbf{Common Special Integrating Factors:}
\begin{enumerate}
    \item $\mu = x^a y^b$ (power form)
    \item $\mu = (ax + by)^n$ (linear combination)
    \item $\mu = e^{f(x,y)}$ (exponential form)
    \item $\mu = f(x \pm y)$ (sum/difference)
    \item $\mu = g(x/y)$ (ratio form)
\end{enumerate}
\end{keypoint}

\section{Power Form: $\mu = x^a y^b$}

\begin{method}[Finding Powers]
For $\mu = x^a y^b$, multiply the original equation and apply exactness:
$$\frac{\partial}{\partial y}(x^a y^b M) = \frac{\partial}{\partial x}(x^a y^b N)$$

This gives:
$$x^a y^{b-1}(bM + y M_y) = x^{a-1} y^b (aN + x N_x)$$

Solve for $a$ and $b$ by comparing coefficients.
\end{method}

\begin{example}[Power Form]
Find an integrating factor for: $(y^2)dx + (xy)dy = 0$

Try $\mu = x^a y^b$:
$$(x^a y^{b+2})dx + (x^{a+1} y^{b+1})dy = 0$$

For exactness:
$\frac{\partial}{\partial y}(x^a y^{b+2}) = x^a(b+2)y^{b+1}$
$\frac{\partial}{\partial x}(x^{a+1} y^{b+1}) = (a+1)x^a y^{b+1}$

Setting equal: $b+2 = a+1$, so $a = b+1$

Choose $b = -1$, then $a = 0$: $\mu = y^{-1} = \frac{1}{y}$

Result: $(y)dx + (x)dy = 0$, which gives $d(xy) = 0$, so $xy = C$.
\end{example}

\section{Linear Combination: $\mu = (ax + by)^n$}

\begin{insight}
This form is useful when the equation has homogeneous-like properties or when $M$ and $N$ have similar structures involving linear combinations of $x$ and $y$.
\end{insight}

\section{Systematic Approach}

\begin{algorithm}
\textbf{Complete Strategy for Finding Integrating Factors:}
\begin{enumerate}
    \item Check if equation is already exact
    \item Test for $\mu(x)$: $(M_y - N_x)/N$ function of $x$ only
    \item Test for $\mu(y)$: $(N_x - M_y)/M$ function of $y$ only  
    \item Test for $\mu(xy)$: $(M_y - N_x)/(xN - yM)$ function of $xy$ only
    \item Try special forms:
    \begin{itemize}
        \item $\mu = x^a y^b$ (solve system for $a,b$)
        \item $\mu = (x \pm y)^n$ 
        \item $\mu = e^{f(x,y)}$ for simple $f$
    \end{itemize}
    \item Use inspection/physical intuition
\end{enumerate}
\end{algorithm}

\begin{warning}
\textbf{Common Challenges:}
\begin{itemize}
    \item Calculations become increasingly complex
    \item Multiple forms may work - choose the simplest
    \item Not all equations have elementary integrating factors
    \item Verification is crucial after finding $\mu$
\end{itemize}
\end{warning}

\begin{examtip}
\textbf{Prof. Ditkowski's Hints:}
\begin{itemize}
    \item Often provides hints: ``Try $\mu = \ldots$''
    \item Looks for recognition of standard patterns
    \item Partial credit for systematic approach
    \item May give the integrating factor and ask you to solve
    \item Sometimes asks for verification rather than derivation
\end{itemize}
\end{examtip}

\section{Special Cases Summary}

\begin{keypoint}
\textbf{Quick Reference Guide:}
\begin{center}
\begin{tabular}{|l|l|}
\hline
\textbf{If you see...} & \textbf{Try...} \\
\hline
$M, N$ with same powers of $x,y$ & $\mu = x^a y^b$ \\
Linear terms dominate & $\mu = (ax + by)^n$ \\
Exponential structure & $\mu = e^{f(x,y)}$ \\
Symmetric in $x,y$ & $\mu = f(xy)$ or $\mu = g(x \pm y)$ \\
Rational functions & $\mu = \frac{x^a y^b}{(x^c + y^c)^n}$ \\
\hline
\end{tabular}
\end{center}
\end{keypoint}

\end{document}