\documentclass[12pt]{article}
\usepackage{amsmath, amssymb, amsthm, enumitem, geometry, mdframed, xcolor}
\geometry{margin=1in}

\newtheorem{definition}{Definition}
\newtheorem{theorem}{Theorem}
\newtheorem{method}{Method}
\newtheorem{example}{Example}
\newmdenv[linecolor=blue,linewidth=2pt]{keypoint}
\newmdenv[linecolor=red,linewidth=2pt]{warning}
\newmdenv[linecolor=green,linewidth=2pt]{insight}
\newmdenv[linecolor=purple,linewidth=2pt]{examtip}
\newmdenv[linecolor=cyan,linewidth=2pt]{algorithm}

\title{Lesson 25: Orthogonal Trajectories and Applications}
\author{ODE 1 - Prof. Adi Ditkowski}
\date{}

\begin{document}
\maketitle

\section{Definition and Geometric Interpretation}

\begin{definition}[Orthogonal Trajectories]
Given a one-parameter family of curves $F(x,y,c) = 0$, the \textbf{orthogonal trajectories} are curves that intersect each member of the family at right angles (90°).
\end{definition}

\begin{keypoint}
At any point of intersection, the product of the slopes of two orthogonal curves equals $-1$:
\[m_1 \cdot m_2 = -1\]
This is the fundamental principle behind finding orthogonal trajectories.
\end{keypoint}

\section{The Systematic Method}

\begin{algorithm}
\textbf{Finding Orthogonal Trajectories:}
\begin{enumerate}
    \item Start with family $F(x,y,c) = 0$
    \item Differentiate to eliminate parameter $c$:
    \[\frac{\partial F}{\partial x} + \frac{\partial F}{\partial y}\frac{dy}{dx} = 0\]
    \item Express as differential equation: $\frac{dy}{dx} = f(x,y)$
    \item Replace $\frac{dy}{dx}$ with $-\frac{dx}{dy}$ to get orthogonal equation:
    \[-\frac{dx}{dy} = f(x,y)\]
    \item Solve the new differential equation
    \item The solution gives the orthogonal trajectories
\end{enumerate}
\end{algorithm}

\begin{warning}
You MUST eliminate the parameter $c$ before forming the differential equation. Failing to do so is the most common error and leads to incorrect results!
\end{warning}

\section{Classic Examples}

\begin{example}[Straight Lines Through Origin]
Find orthogonal trajectories of $y = cx$.

\textbf{Solution:}
\begin{enumerate}
    \item Differentiate: $\frac{dy}{dx} = c$
    \item Eliminate $c$: Since $c = \frac{y}{x}$, we have $\frac{dy}{dx} = \frac{y}{x}$
    \item Orthogonal equation: $-\frac{dx}{dy} = \frac{y}{x}$ or $\frac{dx}{dy} = -\frac{y}{x}$
    \item Rearrange: $x dx = -y dy$
    \item Integrate: $\frac{x^2}{2} = -\frac{y^2}{2} + C$
    \item Simplify: $x^2 + y^2 = K$
\end{enumerate}
The orthogonal trajectories are circles centered at the origin!
\end{example}

\begin{example}[Exponential Curves]
Find orthogonal trajectories of $y = ce^{2x}$.

\textbf{Solution:}
\begin{enumerate}
    \item Differentiate: $\frac{dy}{dx} = 2ce^{2x}$
    \item Eliminate $c$: From original, $c = ye^{-2x}$, so $\frac{dy}{dx} = 2y$
    \item Orthogonal equation: $-\frac{dx}{dy} = 2y$ or $\frac{dx}{dy} = -2y$
    \item Integrate: $x = -y^2 + C$
    \item Result: $x + y^2 = K$ (parabolas opening left)
\end{enumerate}
\end{example}

\section{Physical Applications}

\begin{keypoint}
\textbf{Common Physical Interpretations:}
\begin{itemize}
    \item \textbf{Electrostatics:} Electric field lines $\perp$ equipotential curves
    \item \textbf{Heat Transfer:} Heat flow lines $\perp$ isothermal curves  
    \item \textbf{Fluid Dynamics:} Streamlines $\perp$ velocity potential curves
    \item \textbf{Magnetism:} Magnetic field lines $\perp$ magnetic potential curves
\end{itemize}
\end{keypoint}

\begin{examtip}
\textbf{Prof. Ditkowski's Patterns:}
\begin{itemize}
    \item Always asks to sketch both families and verify they're perpendicular
    \item Loves self-orthogonal families (like $xy = c$)
    \item Tests elimination of parameters carefully
    \item May provide physical context and ask for interpretation
    \item Often combines with initial/boundary conditions
\end{itemize}
\end{examtip}

\section{Special Cases}

\begin{insight}
\textbf{Self-Orthogonal Families:}
Some families are orthogonal to themselves! 
Example: The rectangular hyperbolas $xy = c$ have orthogonal trajectories $xy = k$, belonging to the same family with different parameter values.
\end{insight}

\end{document}