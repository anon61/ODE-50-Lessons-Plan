\documentclass[12pt]{article}
\usepackage{amsmath, amssymb, enumitem}
\usepackage{geometry, tikz}
\geometry{margin=1in}

\title{Practice Problems: Lesson 7 - Non-Unique Solutions}
\author{Master these non-uniqueness patterns!}
\date{}

\begin{document}
\maketitle

\section*{Part A: Identifying Non-Uniqueness}

For each IVP, determine if solutions are unique. If not, explain why:

\begin{enumerate}
    \item $y' = 3y^{2/3}$, $y(0) = 0$
    
    \item $y' = |y|^{1/2}\text{sign}(y)$, $y(0) = 0$
    
    \item $y' = 2\sqrt{y}$, $y(1) = 1$
    
    \item $y' = y^{1/3}$, $y(0) = 1$
    
    \item $y' = \begin{cases}
        2\sqrt{y} & \text{if } y \geq 0 \\
        0 & \text{if } y < 0
    \end{cases}$, $y(0) = 0$
\end{enumerate}

\section*{Part B: Finding Multiple Solutions}

Find at least two different solutions for each IVP:

\begin{enumerate}[start=6]
    \item $y' = 2|y|^{1/2}$, $y(0) = 0$
    
    \item $(y')^2 = 4y$, $y(0) = 0$, with $y \geq 0$
    
    \item $y' = 3y^{2/3}$, $y(1) = 0$
    
    \item $|y'| = |y|^{1/2}$, $y(0) = 0$
\end{enumerate}

\section*{Part C: Solution Families}

\begin{enumerate}[start=10]
    \item For $y' = 2\sqrt{y}$, $y(0) = 0$:
    \begin{enumerate}[label=(\alph*)]
        \item Show that $y \equiv 0$ is a solution
        \item Show that $y = \begin{cases} 0 & x \leq c \\ (x-c)^2 & x > c \end{cases}$ is a solution for any $c \geq 0$
        \item Sketch at least 4 different solutions
        \item Where does the Lipschitz condition fail?
    \end{enumerate}
    
    \item Consider the general equation $y' = k|y|^\alpha$, $y(0) = 0$:
    \begin{enumerate}[label=(\alph*)]
        \item For which values of $\alpha$ is the solution unique?
        \item Find the general solution family for $0 < \alpha < 1$
        \item What happens as $\alpha \to 0^+$?
        \item What happens as $\alpha \to 1^-$?
    \end{enumerate}
\end{enumerate}

\section*{Part D: Lipschitz Analysis}

\begin{enumerate}[start=12]
    \item Check the Lipschitz condition at $y = 0$ for:
    \begin{enumerate}[label=(\alph*)]
        \item $f(y) = y^2$
        \item $f(y) = \sqrt{|y|}$
        \item $f(y) = y\ln|y|$ (with $f(0) = 0$)
        \item $f(y) = |y|^{3/2}$
        \item $f(y) = y^{2/3}\sin(1/y)$ (with $f(0) = 0$)
    \end{enumerate}
    
    \item For each function in Problem 12, state whether the IVP with $y(0) = 0$ has unique solutions.
\end{enumerate}

\section*{Part E: Clairaut and Singular Solutions}

\begin{enumerate}[start=14]
    \item Consider Clairaut's equation: $y = xy' + (y')^2$
    \begin{enumerate}[label=(\alph*)]
        \item Find the general solution (family of lines)
        \item Find the singular solution (envelope)
        \item Verify both satisfy the original equation
        \item Sketch the solution family and envelope
    \end{enumerate}
    
    \item For the equation $y = xy' - (y')^3$:
    \begin{enumerate}[label=(\alph*)]
        \item Find all solutions through the point $(0, 0)$
        \item Are there infinitely many such solutions?
    \end{enumerate}
\end{enumerate}

\section*{Part F: Lost Solutions}

\begin{enumerate}[start=16]
    \item Solve $y' = 2y^{1/2}$ by separation of variables:
    \begin{enumerate}[label=(\alph*)]
        \item What solution do you get?
        \item What solution is missed?
        \item Why was it missed?
    \end{enumerate}
    
    \item For $y' = y^2(1-y)^2$:
    \begin{enumerate}[label=(\alph*)]
        \item Find all equilibrium solutions
        \item Solve by separation of variables
        \item Which solutions might be missed in the process?
    \end{enumerate}
\end{enumerate}

\section*{Part G: Solution Crossing}

\begin{enumerate}[start=18]
    \item Can two solutions of $y' = y^2$ cross? Why or why not?
    
    \item Can two solutions of $y' = |y|^{1/2}$ cross? If yes, where?
    
    \item Consider $y' = f(y)$ where $f$ is Lipschitz everywhere:
    \begin{enumerate}[label=(\alph*)]
        \item Prove that two solutions cannot cross
        \item What does this imply about solution uniqueness?
    \end{enumerate}
\end{enumerate}

\section*{Part H: Exam-Style Problems}

\begin{enumerate}[start=21]
    \item Professor Ditkowski asks: "Give an example of an IVP with exactly 3 solutions."
    \begin{enumerate}[label=(\alph*)]
        \item Construct such an example
        \item Verify all three solutions
        \item Explain why there are exactly three
    \end{enumerate}
    
    \item Consider the "raindrop equation": $y' = -k\sqrt{y}$, $y(T) = 0$
    \begin{enumerate}[label=(\alph*)]
        \item Interpret $y$ physically (hint: radius)
        \item Find all solutions backward in time from $t = T$
        \item What does non-uniqueness mean physically?
    \end{enumerate}
    
    \item For the IVP $y' = |y-1|^{1/2} + |y+1|^{1/2}$, $y(0) = 1$:
    \begin{enumerate}[label=(\alph*)]
        \item Where might uniqueness fail?
        \item Is the solution unique through $(0,1)$?
        \item What about through $(0,-1)$?
        \item What about through $(0,0)$?
    \end{enumerate}
    
    \item The equation $y' = g(y)$ has $g(0) = g(1) = g(2) = 0$ and $g$ is not Lipschitz at these points:
    \begin{enumerate}[label=(\alph*)]
        \item Describe possible solution behaviors
        \item Can a solution starting at $y(0) = 0.5$ reach $y = 2$?
        \item How many solutions might connect $y(0) = 0$ to $y(10) = 2$?
    \end{enumerate}
\end{enumerate}

\section*{Part I: Advanced Theory}

\begin{enumerate}[start=25]
    \item Prove that if $f(y) = |y|^\alpha$ with $0 < \alpha < 1$, then $f$ is not Lipschitz at $y = 0$.
    
    \item Show that the IVP $y' = y\ln|y|$, $y(0) = 0$ (with $f(0) = 0$) has a unique solution despite $f$ not being Lipschitz at 0. (Hint: Use Osgood's criterion)
    
    \item Consider "patching" solutions:
    \begin{enumerate}[label=(\alph*)]
        \item If $y_1(x)$ solves the ODE for $x < a$ with $y_1(a) = 0$
        \item And $y_2(x)$ solves it for $x > b$ with $y_2(b) = 0$
        \item When can you "patch" them with $y \equiv 0$ on $[a,b]$?
    \end{enumerate}
\end{enumerate}

\section*{Solution Hints}

\textbf{Part A:}
1. Not unique ($\alpha = 2/3 < 1$)
2. Not unique (like $\sqrt{|y|}$)
3. Unique (initial value away from 0)
4. Unique (initial value away from 0)
5. Not unique (piecewise with flat spot)

\textbf{Part B Key Ideas:}
- Always check $y \equiv 0$ first
- Look for delayed start solutions
- Consider both positive and negative branches

\textbf{Remember:} Non-uniqueness occurs when $\partial f/\partial y$ is unbounded or discontinuous!

\end{document}