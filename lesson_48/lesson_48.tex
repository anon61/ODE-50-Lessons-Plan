\documentclass[12pt]{article}
\usepackage{amsmath, amssymb, amsthm, tikz, pgfplots}
\usepackage{geometry, enumitem, mdframed, array, xcolor}
\usepackage{algorithm2e}
\geometry{margin=1in}

% Custom environments
\newtheorem{definition}{Definition}
\newtheorem{theorem}{Theorem}
\newtheorem{method}{Method}
\newtheorem{example}{Example}
\newmdenv[linecolor=blue,linewidth=2pt]{keypoint}
\newmdenv[linecolor=red,linewidth=2pt]{warning}
\newmdenv[linecolor=green,linewidth=2pt]{insight}
\newmdenv[linecolor=purple,linewidth=2pt]{examtip}
\newmdenv[linecolor=cyan,linewidth=2pt]{series}

\title{ODE Lesson 48: Power Series Solutions at Ordinary Points}
\author{ODE 1 - Prof. Adi Ditkowski}
\date{}

\begin{document}
\maketitle

\section{Introduction and Motivation}

Many important ODEs cannot be solved in terms of elementary functions. Power series methods provide systematic approaches to find solutions as infinite series.

\begin{definition}[Ordinary and Singular Points]
For the ODE in standard form:
$$y'' + p(x)y' + q(x)y = 0$$
\begin{itemize}
\item $x = x_{0}$ is an \textbf{ordinary point} if both $p(x)$ and $q(x)$ are analytic at $x_{0}$
\item $x = x_{0}$ is a \textbf{singular point} if either $p(x)$ or $q(x)$ is not analytic at $x_{0}$
\end{itemize}
\end{definition}

\begin{keypoint}
At ordinary points, we can always find two linearly independent power series solutions of the form:
$$y = \sum_{n=0}^{\infty} a_{n}(x-x_{0})^{n}$$
\end{keypoint}

\section{The Power Series Method}

\begin{method}[Power Series Solution Procedure]
\begin{enumerate}
\item Verify that $x_{0}$ is an ordinary point
\item Assume: $y = \sum_{n=0}^{\infty} a_{n}(x-x_{0})^{n}$
\item Compute derivatives:
$$y' = \sum_{n=1}^{\infty} na_{n}(x-x_{0})^{n-1}, \quad y'' = \sum_{n=2}^{\infty} n(n-1)a_{n}(x-x_{0})^{n-2}$$
\item Substitute into the ODE
\item Shift indices to align powers of $(x-x_{0})$
\item Equate coefficients of like powers to zero
\item Solve the resulting recurrence relation
\item Express $a_{n}$ in terms of $a_{0}$ and $a_{1}$
\end{enumerate}
\end{method}

\begin{warning}
Index shifting is crucial! When you have $(x-x_{0})^{n-2}$, substitute $m = n-2$ so $n = m+2$:
$$\sum_{n=2}^{\infty} f(n)(x-x_{0})^{n-2} = \sum_{m=0}^{\infty} f(m+2)(x-x_{0})^{m}$$
\end{warning}

\section{Convergence Theory}

\begin{theorem}[Convergence of Power Series Solutions]
If $x_{0}$ is an ordinary point of $y'' + p(x)y' + q(x)y = 0$, then there exist two linearly independent solutions:
$$y_{1} = \sum_{n=0}^{\infty} a_{n}(x-x_{0})^{n}, \quad y_{2} = \sum_{n=0}^{\infty} b_{n}(x-x_{0})^{n}$$
convergent in $|x-x_{0}| < R$, where $R$ is at least the distance from $x_{0}$ to the nearest singular point in the complex plane.
\end{theorem}

\begin{insight}
The radius of convergence is determined by singularities in the complex plane, even if the real line looks fine!
\end{insight}

\section{Key Examples}

\subsection{Example 1: Airy Equation}
$$y'' - xy = 0$$
\begin{series}
Solution around $x = 0$ (ordinary point):
\begin{align}
y &= a_{0}\left(1 + \frac{x^{3}}{2\cdot 3} + \frac{x^{6}}{2\cdot 3 \cdot 5 \cdot 6} + \cdots\right) \\
&\quad + a_{1}\left(x + \frac{x^{4}}{3\cdot 4} + \frac{x^{7}}{3\cdot 4 \cdot 6 \cdot 7} + \cdots\right)
\end{align}
Recurrence: $(n+2)(n+1)a_{n+2} = a_{n-1}$ for $n \geq 1$
\end{series}

\subsection{Example 2: Hermite Equation}
$$y'' - 2xy' + 2\lambda y = 0$$

\begin{examtip}
When $\lambda = n$ (non-negative integer), one solution becomes a polynomial - the Hermite polynomial $H_{n}(x)$.
\end{examtip}

\section{Common Recurrence Relations}

\begin{center}
\begin{tabular}{|l|l|l|}
\hline
\textbf{Equation} & \textbf{Recurrence Relation} & \textbf{Special Property} \\
\hline
$y'' - xy = 0$ (Airy) & $(n+2)(n+1)a_{n+2} = a_{n-1}$ & 3-term recurrence \\
\hline
$y'' - 2xy' + 2ny = 0$ & $(n+2)(n+1)a_{n+2} = 2(n-k)a_{n}$ & Terminates if $k \in \mathbb{N}$ \\
\hline
$(1-x^{2})y'' - 2xy' + n(n+1)y = 0$ & Complex & Legendre polynomials \\
\hline
\end{tabular}
\end{center}

\section{Algorithm for Finding Recurrence Relations}

\begin{algorithm}[H]
\SetAlgoLined
\KwIn{ODE with ordinary point at $x_{0}$}
\KwOut{Recurrence relation for coefficients}
Write ODE in standard form\;
Substitute $y = \sum a_{n}(x-x_{0})^{n}$\;
Compute $y'$ and $y''$ term by term\;
Substitute all series into ODE\;
\For{each summation}{
    Shift index so all have $(x-x_{0})^{m}$\;
}
Combine all terms with same power\;
Set coefficient of each $(x-x_{0})^{m}$ to zero\;
Solve for highest index coefficient\;
\caption{Power Series Method}
\end{algorithm}

\section{Exam Strategy}

\begin{examtip}
Prof. Ditkowski's typical power series problem:
\begin{enumerate}
\item Identify all singular points (1-2 points)
\item Find recurrence relation (3-4 points)
\item Compute first 3-4 non-zero terms (2-3 points)
\item State radius of convergence (1 point)
\end{enumerate}
Total: 7-10 points out of 100
\end{examtip}

\begin{warning}
Common errors that cost points:
\begin{itemize}
\item Wrong index shifting (lose 2-3 points)
\item Sign errors in recurrence (lose 1-2 points)
\item Not simplifying coefficients (lose 1 point)
\item Forgetting initial conditions determine $a_{0}, a_{1}$ (lose 1 point)
\end{itemize}
\end{warning}

\end{document}