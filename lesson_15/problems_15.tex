\documentclass[12pt]{article}
\usepackage{amsmath, amssymb, amsthm, enumitem, geometry}
\usepackage{mdframed, xcolor}
\geometry{margin=1in}

\newmdenv[linecolor=blue,linewidth=2pt]{hint}
\newcommand{\solution}[1]{\textbf{Solution:} #1}

\title{Practice Problems: Integrating Factor Deep Dive}
\author{Lesson 15 - Prof. Ditkowski's ODE 1}
\date{}

\begin{document}
\maketitle

\section*{Part A: Theoretical Understanding (5 problems)}

\begin{enumerate}
\item Prove that if $\mu_1(t)$ and $\mu_2(t)$ are both integrating factors for $y' + p(t)y = g(t)$, then $\mu_2(t) = C\mu_1(t)$ for some constant $C$.

\item Show that the integrating factor $\mu(t) = e^{\int p(t)dt}$ is the unique positive integrating factor with $\mu(t_0) = 1$ for any fixed $t_0$.

\item For the equation $y' + p(t)y = g(t)$, prove that:
\begin{equation}
\frac{d}{dt}[\mu(t)W(t)] = 0
\end{equation}
where $W(t)$ is the Wronskian of any two solutions and $\mu(t)$ is the integrating factor.

\item Explain why the integrating factor method always produces an exact equation, while not every first-order ODE can be made exact.

\item If $p(t)$ is periodic with period $T$, prove that $\mu(t+T) = \mu(t) \cdot \mu(T)$.
\end{enumerate}

\section*{Part B: Computing Complex Integrating Factors (6 problems)}

\begin{enumerate}[resume]
\item Find the integrating factor for: $y' + \frac{2t}{1+t^2}y = e^t$

\item Find the integrating factor for: $y' + (\tan t + \sec t)y = \cos t$

\item Find the integrating factor for: $y' + \frac{1}{t\ln t}y = \frac{1}{t^2}$ for $t > e$

\item Find the integrating factor for: $y' - \frac{3}{2t}y + \frac{1}{2t^2}y = 0$ (Hint: This isn't quite standard form!)

\item For the equation $(t^2 + 1)y' + 2ty = t$, find the integrating factor:
   a) After converting to standard form
   b) Directly without converting to standard form
   Compare your answers.

\item Find the integrating factor for the piecewise coefficient:
\begin{equation}
p(t) = \begin{cases}
2 & 0 \leq t < 1 \\
\frac{1}{t} & t \geq 1
\end{cases}
\end{equation}
\end{enumerate}

\section*{Part C: Alternative Approaches (5 problems)}

\begin{enumerate}[resume]
\item Given that $y_h = t^2e^{-t}$ solves the homogeneous equation $y' + p(t)y = 0$, find:
   a) The function $p(t)$
   b) The integrating factor $\mu(t)$
   c) Verify that $\mu(t) = 1/y_h(t)$ up to a constant

\item For the equation $ty' + (2-t)y = t^2e^t$:
   a) Find the integrating factor without converting to standard form
   b) Show that multiplying by your integrating factor creates the exact form $d/dt[f(t)y] = h(t)$

\item Consider the equation $y' + p(t)y = p(t)$.
   a) Show that $y = 1$ is a particular solution
   b) Use this to find the general solution without computing $\mu(t)$ explicitly
   c) What does this tell you about the relationship between $p(t)$ and the solution?

\item The equation $(1 + t^2)y' - 2ty = (1 + t^2)^2$ can be solved by substitution. Let $v = y/(1+t^2)$.
   a) Show that this transforms the equation to $v' = 1 + t^2$
   b) Solve for $v$ and hence $y$
   c) Find the integrating factor and verify it gives the same solution

\item For the Bernoulli-like equation $y' + p(t)y = g(t)y^0$ (which is actually linear):
   a) Explain why the standard integrating factor works
   b) Show that if we mistakenly treat it as Bernoulli and use substitution $v = y^{1-0} = y$, we get the same equation
\end{enumerate}

\section*{Part D: Connections to Exact Equations (5 problems)}

\begin{enumerate}[resume]
\item Show that after multiplying $y' + \frac{2}{t}y = t^2$ by its integrating factor, the resulting equation can be written as an exact equation $M(t,y)dt + N(t,y)dy = 0$. Find $M$ and $N$ and verify exactness.

\item For the general equation $y' + p(t)y = g(t)$:
   a) After multiplying by $\mu(t)$, express as $M(t,y)dt + N(t,y)dy = 0$
   b) Find the potential function $F(t,y)$ such that $\partial F/\partial t = M$ and $\partial F/\partial y = N$
   c) Show that the solution curves are level sets of $F$

\item Consider the exact equation $(2ty + t^2)dt + t^2dy = 0$.
   a) Verify it's exact
   b) Find an integrating factor that would convert $y' + (2/t)y = -1$ to this exact form
   c) Solve both ways and verify you get the same answer

\item The equation $(\sin t)y' + (\cos t)y = 1$ becomes exact after multiplication by $\mu(t)$.
   a) Find $\mu(t)$
   b) Write the exact form and find the potential function
   c) Solve using both the integrating factor method and the exact equation method

\item Prove that if an equation is already exact, then $\mu(t) = 1$ is an integrating factor, and any other integrating factor must be a constant.
\end{enumerate}

\section*{Part E: Advanced Theory and Applications (5 problems)}

\begin{enumerate}[resume]
\item (Adjoint Equation) For $L[y] = y' + p(t)y$:
   a) Show that the adjoint operator is $L^*[v] = -v' + p(t)v$
   b) Prove that if $\mu$ satisfies $L^*[\mu] = 0$, then $\mu$ is an integrating factor
   c) Find the relationship between solutions of $L[y] = 0$ and $L^*[v] = 0$

\item (Green's Function Preview) For the equation $y' + p(t)y = g(t)$ with $y(0) = 0$:
   a) Show that the solution can be written as $y(t) = \int_0^t G(t,s)g(s)ds$
   b) Find $G(t,s)$ in terms of the integrating factor $\mu(t)$
   c) Verify that $G(t,s) = \mu(s)/\mu(t)$ for $s \leq t$

\item (Numerical Analysis) Consider $y' + 100y = e^{-t}$ on $[0, 10]$.
   a) Find the integrating factor $\mu(t)$
   b) Estimate $\mu(10)$ and discuss potential numerical overflow
   c) Suggest a reformulation to avoid numerical issues

\item (Discontinuous Coefficients) For:
\begin{equation}
y' + p(t)y = 1, \quad p(t) = \begin{cases} 1 & t < 1 \\ 2 & t \geq 1 \end{cases}
\end{equation}
   a) Find the integrating factor on each interval
   b) Solve on each interval with arbitrary constants
   c) Find conditions for $C^0$ continuity at $t = 1$
   d) Can the solution be $C^1$ at $t = 1$?

\item (Generalized Integrating Factors) Sometimes we seek $\mu(t,y)$ for non-linear equations.
   a) For $(y^2 + 2ty)dt + t^2dy = 0$, show that $\mu = 1/y^2$ makes it exact
   b) Explain why this doesn't contradict the uniqueness theorem for linear equations
   c) Find the general solution using this integrating factor
\end{enumerate}

\section*{Part F: Exam-Style Comprehensive Problems (4 problems)}

\begin{enumerate}[resume]
\item Consider the family of equations $y' + p_n(t)y = t^n$ where $p_n(t) = n/t$.
   a) Find the integrating factor $\mu_n(t)$
   b) Solve the equation for general $n$
   c) For which values of $n$ does the solution remain bounded as $t \to \infty$?
   d) For which values of $n$ is the solution continuous at $t = 0$?

\item The equation $y' + (\cot t)y = \csc t$ on $(0, \pi)$:
   a) Find two different forms of the integrating factor using different antiderivatives
   b) Show both lead to the same general solution
   c) Find the solution satisfying $y(\pi/2) = 1$
   d) Determine the behavior as $t \to 0^+$ and $t \to \pi^-$

\item For the equation $ty' + y = te^t$:
   a) Explain why you cannot use the standard form on $(-\infty, \infty)$
   b) Solve separately for $t > 0$ and $t < 0$
   c) Show that no solution exists that is continuous at $t = 0$ unless it satisfies a specific condition
   d) Find the integrating factor directly without converting to standard form

\item (Comprehensive Theory) Let $y' + p(t)y = g(t)$ where $p(t) = 2\alpha\cos(2t)$ and $g(t) = \sin(t)$.
   a) Find the integrating factor (Hint: Use the identity $\int \cos(2t)dt = \sin(2t)/2$)
   b) Show that if $\alpha = 0$, the solution is periodic
   c) For $\alpha \neq 0$, find the general solution
   d) Prove that no periodic solution exists when $\alpha \neq 0$
   e) Discuss the physical interpretation if this represents a driven oscillator
\end{enumerate}

\section*{Solutions and Hints}

\begin{hint}
For Problem 1: Use the fact that both $\mu_1$ and $\mu_2$ satisfy $\mu' = p(t)\mu$.
\end{hint}

\begin{hint}
For Problem 6: Remember that $\int \sec t \, dt = \ln|\sec t + \tan t|$.
\end{hint}

\begin{hint}
For Problem 11: After converting to standard form, $p(t) = 2t/(t^2+1)$. For the direct approach, look for $\mu$ such that $d/dt[\mu(t^2+1)y] = \mu \cdot t$.
\end{hint}

\begin{hint}
For Problem 22: The Green's function represents the influence of the forcing at time $s$ on the solution at time $t > s$.
\end{hint}

\textit{[Complete solutions would be provided showing multiple approaches where applicable, emphasizing the deep connections between different concepts.]}

\end{document}