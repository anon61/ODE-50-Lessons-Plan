\documentclass[12pt]{article}
\usepackage{amsmath, amssymb, amsthm, tikz, pgfplots}
\usepackage{geometry, enumitem, mdframed, array, xcolor}
\usepackage{nicematrix}
\geometry{margin=1in}

\newtheorem{definition}{Definition}
\newtheorem{theorem}{Theorem}
\newtheorem{proposition}{Proposition}
\newtheorem{method}{Method}
\newtheorem{example}{Example}
\newmdenv[linecolor=blue,linewidth=2pt]{keypoint}
\newmdenv[linecolor=red,linewidth=2pt]{warning}
\newmdenv[linecolor=green,linewidth=2pt]{insight}
\newmdenv[linecolor=purple,linewidth=2pt]{examtip}
\newmdenv[linecolor=brown,linewidth=2pt]{computation}

\title{Lesson 27: Fundamental Matrix Solutions - Construction}
\author{ODE 1 - Prof. Adi Ditkowski}
\date{}

\begin{document}
\maketitle

\section{Definition and Properties}

\begin{definition}[Fundamental Matrix]
A matrix $\Phi(t)$ is a \textbf{fundamental matrix} for the system $\mathbf{x}' = A(t)\mathbf{x}$ if:
\begin{enumerate}
\item Each column of $\Phi(t)$ is a solution to the system
\item The columns are linearly independent for all $t$
\item $\Phi'(t) = A(t)\Phi(t)$
\end{enumerate}
\end{definition}

\begin{keypoint}
For an $n \times n$ system, the fundamental matrix $\Phi(t)$ is $n \times n$ with structure:
$$\Phi(t) = [\mathbf{x}$_{1}$(t) \mid \mathbf{x}$_{2}$(t) \mid \cdots \mid \mathbf{x}$_{n}$(t)]$$
where each $\mathbf{x}$_{i}$(t)$ is a linearly independent solution vector.
\end{keypoint}

\begin{theorem}[Fundamental Matrix Properties]
If $\Phi(t)$ is a fundamental matrix, then:
\begin{enumerate}
\item $\det(\Phi(t)) \neq 0$ for all $t$ (never singular)
\item $\Phi'(t) = A(t)\Phi(t)$ (matrix differential equation)
\item General solution: $\mathbf{x}(t) = \Phi(t)\mathbf{c}$ for any constant vector $\mathbf{c}$
\item $\Phi(t)^{-1}$ exists for all $t$
\end{enumerate}
\end{theorem}

\section{Principal Fundamental Matrix}

\begin{definition}[Principal Fundamental Matrix]
The \textbf{principal fundamental matrix} $\Phi(t)$ at $t$_{0}$$ satisfies:
$$\Phi(t$_{0}$) = I$$
where $I$ is the identity matrix.
\end{definition}

\begin{insight}
The principal fundamental matrix provides the simplest IVP solution formula:
$$\mathbf{x}(t) = \Phi(t)\mathbf{x}$_{0}$$$
for the initial condition $\mathbf{x}(t$_{0}$) = \mathbf{x}$_{0}$$.
\end{insight}

\section{Solution of Initial Value Problems}

\begin{theorem}[IVP Solution Formula]
For the IVP:
$$\mathbf{x}' = A(t)\mathbf{x}, \quad \mathbf{x}(t$_{0}$) = \mathbf{x}$_{0}$$$
The unique solution is:
$$\mathbf{x}(t) = \Phi(t)\Phi(t$_{0}$)^{-1}\mathbf{x}$_{0}$$$
where $\Phi(t)$ is any fundamental matrix.
\end{theorem}

\begin{computation}
Solution procedure:
\begin{enumerate}
\item Find $n$ linearly independent solutions $\mathbf{x}$_{1}$(t), \ldots, \mathbf{x}$_{n}$(t)$
\item Construct $\Phi(t) = [\mathbf{x}$_{1}$(t) \mid \cdots \mid \mathbf{x}$_{n}$(t)]$
\item Compute $\Phi(t$_{0}$)$
\item Find $\Phi(t$_{0}$)^{-1}$
\item Calculate $\mathbf{x}(t) = \Phi(t)\Phi(t$_{0}$)^{-1}\mathbf{x}$_{0}$$
\end{enumerate}
\end{computation}

\section{Construction Methods}

\begin{method}[For Constant Coefficient Systems]
For $\mathbf{x}' = A\mathbf{x}$ with constant $A$:
\begin{enumerate}
\item Find eigenvalues $\lambda$_{1}$, \ldots, \lambda$_{n}$$ of $A$
\item Find corresponding eigenvectors $\mathbf{v}$_{1}$, \ldots, \mathbf{v}$_{n}$$
\item Solutions: $\mathbf{x}$_{i}$(t) = e^{\lambda$_{i}$ t}\mathbf{v}$_{i}$$
\item Fundamental matrix: $\Phi(t) = [e^{\lambda$_{1}$ t}\mathbf{v}$_{1}$ \mid \cdots \mid e^{\lambda$_{n}$ t}\mathbf{v}$_{n}$]$
\end{enumerate}
\end{method}

\section{Examples}

\begin{example}[2\times2 System]
Consider $\mathbf{x}' = \begin{bmatrix} 1 & 2 \\ 3 & 2 \end{bmatrix}\mathbf{x}$.

\textbf{Step 1:} Find eigenvalues
$$\det(A - \lambda I) = \det\begin{bmatrix} 1-\lambda & 2 \\ 3 & 2-\lambda \end{bmatrix} = \lambda$^{2}$ - 3\lambda - 4 = 0$$
$$\lambda$_{1}$ = 4, \quad \lambda$_{2}$ = -1$$

\textbf{Step 2:} Find eigenvectors
For $\lambda$_{1}$ = 4$: $\mathbf{v}$_{1}$ = \begin{bmatrix} 2 \\ 3 \end{bmatrix}$
For $\lambda$_{2}$ = -1$: $\mathbf{v}$_{2}$ = \begin{bmatrix} 1 \\ -1 \end{bmatrix}$

\textbf{Step 3:} Construct fundamental matrix
$$\Phi(t) = \begin{bmatrix} 2e^{4t} & e^{-t} \\ 3e^{4t} & -e^{-t} \end{bmatrix}$$

\textbf{Step 4:} Verify $\Phi'(t) = A\Phi(t)$
$$\Phi'(t) = \begin{bmatrix} 8e^{4t} & -e^{-t} \\ 12e^{4t} & e^{-t} \end{bmatrix}$$
$$A\Phi(t) = \begin{bmatrix} 1 & 2 \\ 3 & 2 \end{bmatrix}\begin{bmatrix} 2e^{4t} & e^{-t} \\ 3e^{4t} & -e^{-t} \end{bmatrix} = \begin{bmatrix} 8e^{4t} & -e^{-t} \\ 12e^{4t} & e^{-t} \end{bmatrix} \checkmark$$
\end{example}

\begin{example}[IVP Solution]
Using the fundamental matrix from Example 1, solve:
$$\mathbf{x}' = \begin{bmatrix} 1 & 2 \\ 3 & 2 \end{bmatrix}\mathbf{x}, \quad \mathbf{x}(0) = \begin{bmatrix} 5 \\ 7 \end{bmatrix}$$

\textbf{Solution:}
$$\Phi(0) = \begin{bmatrix} 2 & 1 \\ 3 & -1 \end{bmatrix}$$
$$\Phi(0)^{-1} = \frac{1}{-5}\begin{bmatrix} -1 & -1 \\ -3 & 2 \end{bmatrix} = \begin{bmatrix} 1/5 & 1/5 \\ 3/5 & -2/5 \end{bmatrix}$$
$$\Phi(0)^{-1}\mathbf{x}$_{0}$ = \begin{bmatrix} 1/5 & 1/5 \\ 3/5 & -2/5 \end{bmatrix}\begin{bmatrix} 5 \\ 7 \end{bmatrix} = \begin{bmatrix} 12/5 \\ 1/5 \end{bmatrix}$$
$$\mathbf{x}(t) = \Phi(t)\begin{bmatrix} 12/5 \\ 1/5 \end{bmatrix} = \begin{bmatrix} \frac{24}{5}e^{4t} + \frac{1}{5}e^{-t} \\ \frac{36}{5}e^{4t} - \frac{1}{5}e^{-t} \end{bmatrix}$$
\end{example}

\section{Relationship Between Fundamental Matrices}

\begin{theorem}[Fundamental Matrix Relationship]
If $\Phi$_{1}$(t)$ and $\Phi$_{2}$(t)$ are both fundamental matrices for the same system, then:
$$\Phi$_{2}$(t) = \Phi$_{1}$(t)C$$
where $C$ is a constant nonsingular matrix.
\end{theorem}

\begin{warning}
Common errors:
\begin{itemize}
\item Forgetting to verify linear independence of solutions
\item Wrong matrix multiplication order in IVP formula
\item Not checking $\Phi'(t) = A(t)\Phi(t)$
\item Confusing $\Phi(t)$ with individual solution vectors
\end{itemize}
\end{warning}

\section{Special Cases and Extensions}

\begin{insight}
For repeated eigenvalues, the fundamental matrix includes terms with $t$:
$$\Phi(t) = e^{\lambda t}[I + tN + \frac{t$^{2}$}{2!}N$^{2}$ + \cdots]$$
where $N$ is the nilpotent part of $A - \lambda I$.
\end{insight}

\begin{examtip}
Prof. Ditkowski's favorite exam questions:
\begin{itemize}
\item Construct $\Phi(t)$ from given solutions
\item Verify fundamental matrix property
\item Use $\Phi(t)$ to solve specific IVP
\item Find relationship between two fundamental matrices
\item Compute $\Phi(t)\Phi(s)^{-1}$ for various $s, t$
\end{itemize}
\end{examtip}

\section{Matrix Exponential Connection}

\begin{proposition}
For constant matrix $A$, the principal fundamental matrix at $t$_{0}$ = 0$ is:
$$\Phi(t) = e^{At} = I + At + \frac{A$^{2t}$^2}{2!} + \cdots$$
\end{proposition}

\end{document}