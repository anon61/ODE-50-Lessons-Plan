\documentclass[12pt]{article}
\usepackage{amsmath, amssymb, amsthm, tikz, pgfplots}
\usepackage{geometry, enumitem, mdframed, array, xcolor}
\geometry{margin=1in}

% Custom environments
\newtheorem{definition}{Definition}
\newtheorem{theorem}{Theorem}
\newtheorem{method}{Method}
\newtheorem{example}{Example}
\newmdenv[linecolor=blue,linewidth=2pt]{keypoint}
\newmdenv[linecolor=red,linewidth=2pt]{warning}
\newmdenv[linecolor=green,linewidth=2pt]{insight}
\newmdenv[linecolor=purple,linewidth=2pt]{examtip}

\title{Direct Integration: The Foundation of ODE Solutions}
\author{ODE 1 - Prof. Adi Ditkowski}
\date{Lesson 11}

\begin{document}
\maketitle

\section{Introduction}

\begin{definition}[Directly Integrable ODE]
An ODE is \textbf{directly integrable} if it can be written in the form:
$$\frac{d^n y}{dx^n} = f(x)$$
where $f(x)$ is a function of $x$ only (not involving $y$ or its derivatives).
\end{definition}

\begin{keypoint}
Direct integration is the simplest solution method, requiring only antiderivatives. Yet it forms the foundation for all other ODE techniques!
\end{keypoint}

\section{First-Order Direct Integration}

\subsection{General Method}

\begin{method}[Solving $y' = f(x)$]
\begin{enumerate}
    \item Recognize the form: $\frac{dy}{dx} = f(x)$
    \item Integrate both sides: $y = \int f(x)\,dx + C$
    \item If initial condition $y(x_0) = y_0$ is given, determine $C$
    \item Verify the solution by differentiation
\end{enumerate}
\end{method}

\begin{example}[Basic Direct Integration]
Solve: $\frac{dy}{dx} = 3x^2 - 2x + 1$ with $y(0) = 5$

\textbf{Solution:}
\begin{align}
y &= \int (3x^2 - 2x + 1)\,dx\\
&= x^3 - x^2 + x + C
\end{align}

Using $y(0) = 5$:
$$5 = 0^3 - 0^2 + 0 + C \implies C = 5$$

Therefore: $y = x^3 - x^2 + x + 5$
\end{example}

\subsection{Definite Integral Form}

\begin{theorem}[IVP Solution via Definite Integral]
The solution to $\frac{dy}{dx} = f(x)$ with $y(x_0) = y_0$ is:
$$y(x) = y_0 + \int_{x_0}^{x} f(t)\,dt$$
\end{theorem}

\begin{insight}
The definite integral form automatically incorporates the initial condition - no need to find $C$ separately!
\end{insight}

\section{Higher-Order Direct Integration}

\subsection{General Pattern}

\begin{theorem}[n-th Order Direct Integration]
For $\frac{d^n y}{dx^n} = f(x)$, the general solution contains $n$ arbitrary constants:
$$y(x) = \int^{(n)} f(x)\,dx + C_1 x^{n-1} + C_2 x^{n-2} + \cdots + C_{n-1}x + C_n$$
where $\int^{(n)}$ denotes $n$-fold integration.
\end{theorem}

\begin{example}[Second-Order]
Solve: $y'' = \cos(x)$ with $y(0) = 1$, $y'(0) = 0$

\textbf{First integration:}
$$y' = \int \cos(x)\,dx = \sin(x) + C_1$$

Using $y'(0) = 0$: $0 = \sin(0) + C_1 \implies C_1 = 0$

\textbf{Second integration:}
$$y = \int \sin(x)\,dx = -\cos(x) + C_2$$

Using $y(0) = 1$: $1 = -\cos(0) + C_2 = -1 + C_2 \implies C_2 = 2$

\textbf{Solution:} $y = -\cos(x) + 2$
\end{example}

\section{Special Cases and Common Integrals}

\subsection{Important Antiderivatives for ODEs}

\begin{center}
\begin{tabular}{|c|c|c|}
\hline
\textbf{$f(x)$} & \textbf{$\int f(x)\,dx$} & \textbf{Domain Note} \\
\hline
$x^n$ $(n \neq -1)$ & $\frac{x^{n+1}}{n+1} + C$ & All $x$ \\
\hline
$\frac{1}{x}$ & $\ln|x| + C$ & $x \neq 0$ \\
\hline
$e^{ax}$ & $\frac{1}{a}e^{ax} + C$ & All $x$ \\
\hline
$\sin(ax)$ & $-\frac{1}{a}\cos(ax) + C$ & All $x$ \\
\hline
$\cos(ax)$ & $\frac{1}{a}\sin(ax) + C$ & All $x$ \\
\hline
$\frac{1}{x^2 + a^2}$ & $\frac{1}{a}\arctan\left(\frac{x}{a}\right) + C$ & All $x$ \\
\hline
$\frac{1}{\sqrt{a^2 - x^2}}$ & $\arcsin\left(\frac{x}{a}\right) + C$ & $|x| < a$ \\
\hline
\end{tabular}
\end{center}

\begin{warning}
Always include absolute values in $\ln|x|$ and check domain restrictions!
\end{warning}

\subsection{The Zero Derivative Case}

\begin{theorem}[Polynomial Solutions]
If $\frac{d^n y}{dx^n} = 0$, then $y$ is a polynomial of degree at most $n-1$:
$$y(x) = C_1 x^{n-1} + C_2 x^{n-2} + \cdots + C_{n-1}x + C_n$$
\end{theorem}

\section{Solution Verification}

\begin{method}[Verification Checklist]
\begin{enumerate}
    \item Differentiate your solution $n$ times
    \item Substitute into the original ODE
    \item Verify the equation holds identically
    \item Check initial conditions (if given)
    \item Verify domain of validity
\end{enumerate}
\end{method}

\begin{examtip}
Prof. Ditkowski awards partial credit for solution verification even if your answer is incorrect. Always show this step!
\end{examtip}

\section{Common Errors to Avoid}

\begin{warning}
Critical mistakes that lose points:
\begin{itemize}
    \item Forgetting $+C$ in indefinite integrals
    \item Missing absolute values: $\int \frac{1}{x}dx = \ln|x| + C$
    \item Wrong number of constants for higher-order equations
    \item Not checking domain restrictions
    \item Arithmetic errors in determining constants
\end{itemize}
\end{warning}

\section{Physical Applications}

\begin{example}[Free Fall Motion]
For an object in free fall: $\frac{d^2 y}{dt^2} = -g$

where $y$ is height, $t$ is time, $g$ is gravitational acceleration.

\textbf{Solution:}
\begin{align}
v = \frac{dy}{dt} &= -gt + v_0 \quad \text{(velocity)}\\
y &= -\frac{1}{2}gt^2 + v_0 t + y_0 \quad \text{(position)}
\end{align}

The constants $v_0$ and $y_0$ represent initial velocity and position.
\end{example}

\end{document}