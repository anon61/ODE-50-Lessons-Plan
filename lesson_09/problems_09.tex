\documentclass[12pt]{article}
\usepackage{amsmath, amssymb, enumitem}
\usepackage{geometry}
\geometry{margin=1in}

\title{Practice Problems: Lesson 9 - Direction Fields and Isoclines}
\author{Master the visualization of ODEs!}
\date{}

\begin{document}
\maketitle

\section*{Part A: Basic Concepts (6 problems)}

\begin{enumerate}
\item For $\frac{dy}{dx} = x + y$, find the isoclines for slopes $c = 0, 1, -1, 2$.

\item Identify all equilibrium points for $\frac{dy}{dx} = (x-1)(y+2)$.

\item Sketch the direction field for $\frac{dy}{dx} = -y$ on the region $[-2,2] \times [-2,2]$.

\item Given a direction field, how can you identify points where solutions have horizontal tangents?

\item For $\frac{dy}{dx} = x^2$, explain why all isoclines are vertical lines.

\item True or False: A solution curve can cross an isocline at most once. Explain.
\end{enumerate}

\section*{Part B: Core Techniques (6 problems)}

\begin{enumerate}[resume]
\item Construct the direction field for $\frac{dy}{dx} = \frac{y}{x}$ (exclude $x=0$). What are the isoclines?

\item For $\frac{dy}{dx} = xy$, find equations for all isoclines and sketch the nullcline.

\item Draw the direction field for $\frac{dy}{dx} = y - x^2$ and identify regions where solutions are increasing.

\item Given $\frac{dy}{dx} = \sin(x) - y$, find the isocline for slope $0$ and sketch nearby solution behavior.

\item For $\frac{dy}{dx} = \frac{x}{y}$, describe the direction field along the axes. What's special about these points?

\item Sketch solution curves for $\frac{dy}{dx} = 2x$ passing through $(0,0)$, $(1,1)$, and $(-1,2)$.
\end{enumerate}

\section*{Part C: Applications (5 problems)}

\begin{enumerate}[resume]
\item A direction field shows all arrows pointing toward the line $y = 2x + 1$. What can you conclude about long-term behavior?

\item For the logistic equation $\frac{dy}{dx} = y(1-y)$, analyze the direction field to determine stability of equilibria.

\item Given $\frac{dy}{dx} = (x-1)^2 + (y-1)^2 - 1$, describe the direction field on the circle centered at $(1,1)$ with radius $1$.

\item For $\frac{dy}{dx} = -x/y$, explain why solution curves must be circles centered at the origin.

\item A chemical reaction follows $\frac{dy}{dx} = k(a-y)(b-y)$ where $a > b > 0$. Analyze the direction field behavior.
\end{enumerate}

\section*{Part D: Advanced/Theoretical (5 problems)}

\begin{enumerate}[resume]
\item Prove that if $f(x,y)$ is continuous, solution curves cannot cross except at equilibrium points.

\item For $\frac{dy}{dx} = f(y)$ (autonomous equation), explain why all isoclines are horizontal lines.

\item Show that if a direction field has a line of symmetry, solution curves respect this symmetry.

\item Given two ODEs with direction fields that differ only in magnitude (not direction) of arrows, how do their solution curves relate?

\item Prove that near a saddle point, there exist exactly four special solution curves (separatrices).
\end{enumerate}

\section*{Part E: Exam-Style Questions (6 problems)}

\begin{enumerate}[resume]
\item \textbf{[Prof. Ditkowski Special]} Sketch the complete direction field for $\frac{dy}{dx} = x^2 - y^2$. Find all equilibria, draw five distinct isoclines, and sketch three solution curves showing different behaviors.

\item Given only the direction field (figure provided on exam), determine the ODE from:
    \begin{enumerate}[label=\alph*)]
        \item $\frac{dy}{dx} = x - y$
        \item $\frac{dy}{dx} = x + y$
        \item $\frac{dy}{dx} = xy$
        \item $\frac{dy}{dx} = x/y$
    \end{enumerate}

\item For $\frac{dy}{dx} = y^2 - x$, without solving:
    \begin{enumerate}[label=\alph*)]
        \item Find all nullclines
        \item Determine regions where solutions are concave up
        \item Sketch the solution passing through $(1,1)$
        \item Describe behavior as $x \to \infty$
    \end{enumerate}

\item \textbf{[Multi-part]} Consider $\frac{dy}{dx} = (x-1)(y-1)(y+1)$:
    \begin{enumerate}[label=\alph*)]
        \item Find all equilibrium points
        \item Classify stability of each equilibrium using the direction field
        \item Identify all separatrices
        \item Sketch the complete phase portrait
    \end{enumerate}

\item A direction field shows spiraling arrows converging to a point. Which ODE could produce this?
    \begin{enumerate}[label=\alph*)]
        \item $\frac{dy}{dx} = -x - y$
        \item $\frac{dy}{dx} = x^2 + y^2$
        \item $\frac{dy}{dx} = xy$
        \item Cannot determine
    \end{enumerate}

\item \textbf{[Conceptual]} Explain how to determine from a direction field alone whether an ODE has periodic solutions. Apply your method to analyze $\frac{dy}{dx} = -x + y^3$.
\end{enumerate}

\section*{Answer Key with Hints}

\textbf{Problem 1:} $y = -x$ (slope 0), $y = -x + 1$ (slope 1), $y = -x - 1$ (slope -1), $y = -x + 2$ (slope 2)

\textbf{Problem 2:} Single equilibrium at $(1, -2)$

\textbf{Problem 7:} Isoclines are rays from origin: $y = cx$

\textbf{Problem 13:} Along $y = 2x + 1$, slope equals $0$, so this is an attractor

\textbf{Problem 19:} Horizontal isoclines mean $f$ doesn't depend on $x$ explicitly

\textbf{Problem 24:} Use nullclines $y = \pm 1$ and $x = 1$ to divide plane into regions

\end{document}