\documentclass[12pt]{article}
\usepackage{amsmath, amssymb, amsthm, tikz, pgfplots}
\usepackage{geometry, enumitem, mdframed, array, xcolor}
\geometry{margin=1in}

% Custom environments
\newtheorem{definition}{Definition}
\newtheorem{theorem}{Theorem}
\newtheorem{method}{Method}
\newtheorem{example}{Example}
\newmdenv[linecolor=blue,linewidth=2pt]{keypoint}
\newmdenv[linecolor=red,linewidth=2pt]{warning}
\newmdenv[linecolor=green,linewidth=2pt]{insight}
\newmdenv[linecolor=purple,linewidth=2pt]{examtip}
\newmdenv[linecolor=cyan,linewidth=2pt]{algorithm}

\title{ODE Lesson 22: Finding Potential Functions - Systematic Approach}
\author{ODE 1 - Prof. Adi Ditkowski}
\date{}

\begin{document}
\maketitle

\section{The Potential Function}

\begin{definition}[Potential Function]
For an exact equation $M(x,y)dx + N(x,y)dy = 0$, the \textbf{potential function} $H(x,y)$ satisfies:
\begin{align}
\frac{\partial H}{\partial x} &= M(x,y) \\
\frac{\partial H}{\partial y} &= N(x,y)
\end{align}
The general solution is then given by $H(x,y) = C$.
\end{definition}

\begin{keypoint}
The potential function is unique up to an additive constant. All methods for finding $H$ yield the same result.
\end{keypoint}

\section{Method 1: Integration with Respect to $x$}

\begin{algorithm}
\textbf{Method 1 - Integrate $M$ with respect to $x$:}
\begin{enumerate}
    \item Since $\frac{\partial H}{\partial x} = M(x,y)$, integrate:
    \[H(x,y) = \int M(x,y)\,dx + g(y)\]
    where $g(y)$ is an arbitrary function of $y$ alone.

    \item Differentiate the result with respect to $y$:
    \[\frac{\partial H}{\partial y} = \frac{\partial}{\partial y}\left[\int M(x,y)\,dx\right] + g'(y)\]

    \item Set this equal to $N(x,y)$:
    \[\frac{\partial}{\partial y}\left[\int M(x,y)\,dx\right] + g'(y) = N(x,y)\]

    \item Solve for $g'(y)$:
    \[g'(y) = N(x,y) - \frac{\partial}{\partial y}\left[\int M(x,y)\,dx\right]\]

    \item Integrate to find $g(y)$:
    \[g(y) = \int g'(y)\,dy\]

    \item Write the complete potential function:
    \[H(x,y) = \int M(x,y)\,dx + g(y)\]
\end{enumerate}
\end{algorithm}

\begin{example}[Method 1 Application]
Solve $(3x^{2y} + y^{3})dx + (x^{3} + 3xy^{2})dy = 0$

\textbf{Step 1:} Verify exactness (assumed done): $\frac{\partial M}{\partial y} = 3x^{2} + 3y^{2} = \frac{\partial N}{\partial x}$ $\checkmark$

\textbf{Step 2:} Integrate $M = 3x^{2y} + y^{3}$ with respect to $x$:
\[H = x^{3y} + xy^{3} + g(y)\]

\textbf{Step 3:} Differentiate with respect to $y$:
\[\frac{\partial H}{\partial y} = x^{3} + 3xy^{2} + g'(y)\]

\textbf{Step 4:} Set equal to $N = x^{3} + 3xy^{2}$:
\[x^{3} + 3xy^{2} + g'(y) = x^{3} + 3xy^{2}\]

\textbf{Step 5:} Therefore $g'(y) = 0$, so $g(y) = 0$ (we can choose the constant to be 0).

\textbf{Solution:} $H(x,y) = x^{3y} + xy^{3}$, so the general solution is $x^{3y} + xy^{3} = C$.
\end{example}

\section{Method 2: Integration with Respect to $y$}

\begin{algorithm}
\textbf{Method 2 - Integrate $N$ with respect to $y$:}
\begin{enumerate}
    \item Since $\frac{\partial H}{\partial y} = N(x,y)$, integrate:
    \[H(x,y) = \int N(x,y)\,dy + f(x)\]
    where $f(x)$ is an arbitrary function of $x$ alone.

    \item Differentiate with respect to $x$:
    \[\frac{\partial H}{\partial x} = \frac{\partial}{\partial x}\left[\int N(x,y)\,dy\right] + f'(x)\]

    \item Set equal to $M(x,y)$ and solve for $f'(x)$:
    \[f'(x) = M(x,y) - \frac{\partial}{\partial x}\left[\int N(x,y)\,dy\right]\]

    \item Integrate to find $f(x)$ and write complete $H(x,y)$.
\end{enumerate}
\end{algorithm}

\begin{insight}
Choose Method 1 when $M$ is simpler to integrate. Choose Method 2 when $N$ is simpler. The choice can significantly reduce computation time on exams!
\end{insight}

\section{Method 3: Line Integral Approach}

\begin{algorithm}
\textbf{Method 3 - Path Integration:}

Since the equation is exact, the line integral is path-independent:
\[H(x,y) = \int_{(x_{0},y_{0})}^{(x,y)} M\,dx + N\,dy\]

Common choice: Use path from $(0,0) \to (x,0) \to (x,y)$:
\[H(x,y) = \int_{0}^x M(t,0)\,dt + \int_{0}^y N(x,s)\,ds\]
\end{algorithm}

\begin{examtip}
Prof. Ditkowski often asks: "Solve using two different methods and verify they give the same result." This tests your understanding that the potential function is unique.
\end{examtip}

\section{Verification Process}

\begin{warning}
Always verify your solution! Check that:
\begin{enumerate}
    \item $\frac{\partial H}{\partial x} = M(x,y)$ $\checkmark$
    \item $\frac{\partial H}{\partial y} = N(x,y)$ $\checkmark$
\end{enumerate}
This catches errors and ensures partial credit.
\end{warning}

\section{Common Integration Patterns}

\begin{keypoint}
\textbf{Memorize these common potential functions:}
\begin{center}
\begin{tabular}{|l|l|}
\hline
\textbf{If you see} & \textbf{Think potential} \\
\hline
$y\,dx + x\,dy$ & $H = xy$ \\
$2xy\,dx + x^{2}\,dy$ & $H = x^{2y}$ \\
$\frac{y}{x^{2}}\,dx - \frac{1}{x}\,dy$ & $H = -\frac{y}{x} \\
e^{x}\sin y\,dx + e^{x}\cos y\,dy & H = e^{x}\sin y \\
\frac{x}{\sqrt{x^{2}+y^{2}}}\,dx + \frac{y}{\sqrt{x^{2}+y^{2}}}\,dy$ & $H = \sqrt{x^{2}+y^{2}}$ \\
\hline
\end{tabular}
\end{center}
\end{keypoint}

\section{Initial Value Problems}

\begin{method}[Solving IVPs with Exact Equations]
\begin{enumerate}
    \item Find the potential function $H(x,y)$
    \item Use initial condition $(x_{0}, y_{0})$ to find $C$:
    \[H(x_{0}, y_{0}) = C\]
    \item Write particular solution: $H(x,y) = C$
\end{enumerate}
\end{method}

\begin{example}[IVP Example]
Solve $(2xy + 1)dx + (x^{2} + 2y)dy = 0$ with $y(1) = 2$.

\textbf{Solution:} From Method 1: $H = x^{2y} + x + y^{2}$

Using $y(1) = 2$: $H(1,2) = (1)^{2}(2) + 1 + (2)^{2} = 7$

Particular solution: $x^{2y} + x + y^{2} = 7$
\end{example}

\section{Efficiency Tips}

\begin{insight}
\textbf{Strategic Integration Choices:}
\begin{itemize}
    \item If $M$ contains $\ln$, $\arctan$, or complex expressions in $y$ \rightarrow Use Method 2
    \item If $N$ contains $\ln$, $\arctan$, or complex expressions in $x$ \rightarrow Use Method 1
    \item If both are complex but simplify when one variable is 0 \rightarrow Use Method 3
    \item For polynomials, choose based on lower degree
\end{itemize}
\end{insight}

\end{document}