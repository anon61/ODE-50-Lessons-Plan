\documentclass[12pt]{article}
\usepackage{amsmath, amssymb, amsthm, tikz, pgfplots}
\usepackage{geometry, enumitem, mdframed, array, xcolor, booktabs}
\geometry{margin=1in}

% Custom environments
\newtheorem{definition}{Definition}
\newtheorem{theorem}{Theorem}
\newtheorem{method}{Method}
\newtheorem{example}{Example}
\newmdenv[linecolor=blue,linewidth=2pt]{keypoint}
\newmdenv[linecolor=red,linewidth=2pt]{warning}
\newmdenv[linecolor=green,linewidth=2pt]{insight}
\newmdenv[linecolor=purple,linewidth=2pt]{examtip}
\newmdenv[linecolor=orange,linewidth=2pt]{formula}

\title{ODE Lesson 45: Method of Undetermined Coefficients}
\author{ODE 1 with Prof. Adi Ditkowski}
\date{Tel Aviv University}

\begin{document}

\maketitle

\section{The Non-Homogeneous Linear ODE}

\begin{definition}[Non-Homogeneous Linear ODE]
An $n$-th order non-homogeneous linear ODE with constant coefficients has the form:
$$L[y] = a_n y^{(n)} + a_{n-1} y^{(n-1)} + \cdots + a_1 y' + a_0 y = f(t)$$
where $f(t) \not\equiv 0$ is called the forcing function or non-homogeneous term.
\end{definition}

\begin{theorem}[Structure of General Solution]
The general solution of $L[y] = f(t)$ is:
$$y(t) = y_h(t) + y_p(t)$$
where:
\begin{itemize}
    \item $y_h(t)$ is the general solution of the homogeneous equation $L[y] = 0$
    \item $y_p(t)$ is any particular solution of $L[y] = f(t)$
\end{itemize}
\end{theorem}

\begin{proof}
If $L[y_p] = f(t)$ and $L[y_h] = 0$, then:
$$L[y_h + y_p] = L[y_h] + L[y_p] = 0 + f(t) = f(t)$$
Conversely, if $y_1$ and $y_2$ both satisfy $L[y] = f(t)$, then:
$$L[y_1 - y_2] = L[y_1] - L[y_2] = f(t) - f(t) = 0$$
So $y_1 - y_2$ is a homogeneous solution.
\end{proof}

\section{Suitable Forcing Functions}

\begin{definition}[UC-Suitable Functions]
A function $f(t)$ is suitable for the method of undetermined coefficients if it belongs to a finite-dimensional space that is closed under differentiation.
\end{definition}

\begin{keypoint}
The method works for linear combinations of:
\begin{enumerate}
    \item Polynomials: $t^n, t^{n-1}, \ldots, t, 1
    \item Exponentials: $e^{at}$ where a \in \mathbb{C}$
    \item Trigonometric: $\sin(bt), \cos(bt)$ where $b \in \mathbb{R}$
    \item Products: t^n $e^{at}$, e^{at}\sin(bt), e^{at}\cos(bt), t^n e^{at}\sin(bt), etc.
\end{enumerate}
\end{keypoint}

\section{The Method: Non-Resonant Case}

\begin{method}[Undetermined Coefficients - Basic]
For $L[y] = f(t)$ where $f(t)$ is UC-suitable:
\begin{enumerate}
    \item Solve the homogeneous equation $L[y] = 0$ to find $y_h$
    \item Based on the form of $f(t)$, guess the form of $y_p$ with undetermined coefficients
    \item Compute the derivatives of $y_p$
    \item Substitute into $L[y_p] = f(t)$
    \item Equate coefficients of like terms to determine the unknowns
    \item Write the general solution: $y = y_h + y_p$
\end{enumerate}
\end{method}

\subsection{Guessing Rules for Common Functions}

\begin{center}
\begin{tabular}{|l|l|}
\hline
\textbf{Forcing Function $f(t)$} & \textbf{Trial Solution $y_p(t)$} \\
\hline
$P_n(t)$ (polynomial degree $n$) & $A_n t^n + A_{n-1}t^{n-1} + \cdots + A_1 t + A_0 \\
\hline
$e^{at}$ & Ae^{at} \\
\hline
\sin(bt) or $\cos(bt)$ & $A\cos(bt) + B\sin(bt) \\
\hline
P_n(t) $e^{at}$ & (A_n t^n + \cdots + A_0)$e^{at}$ \\
\hline
e^{at}\sin(bt) or e^{at}\cos(bt) & $e^{at}$[A\cos(bt) + B\sin(bt)] \\
\hline
P_n(t) $e^{at}$ \sin(bt) & $e^{at}$[(A_n t^n + \cdots + A_0)\cos(bt) + \\
& (B_n t^n + \cdots + B_0)\sin(bt)]$ \\
\hline
\end{tabular}
\end{center}

\begin{warning}
Always include both $\sin$ and $\cos$ terms when the forcing function contains either trigonometric function, as derivatives will produce both.
\end{warning}

\section{Resonance and the Modification Rule}

\begin{definition}[Resonance]
Resonance occurs when the trial solution $y_p$ (or part of it) is a solution of the homogeneous equation $L[y] = 0$.
\end{definition}

\begin{theorem}[Modification for Resonance]
If the standard trial solution is a homogeneous solution corresponding to a root $r$ of multiplicity $m$, multiply the trial solution by $t^m$.
\end{theorem}

\begin{insight}
Resonance physically corresponds to driving a system at its natural frequency, leading to unbounded growth in amplitude (the $t^m$ factor).
\end{insight}

\subsection{Resonance Detection Algorithm}

\begin{method}[Checking for Resonance]
\begin{enumerate}
    \item Find all roots of the characteristic equation (with multiplicities)
    \item For forcing function f(t) = e^{at}g(t):
    \begin{itemize}
        \item If $a$ is not a characteristic root: no modification
        \item If $a$ is a simple root: multiply trial by $t$
        \item If $a$ is a root of multiplicity $m$: multiply trial by $t^m$
    \end{itemize}
    \item For f(t) = $e^{at}$[\sin(bt) \text{ or } \cos(bt)]:
    \begin{itemize}
        \item Check if $a \pm ib$ are characteristic roots
        \item Multiply by $t^m$ where $m$ is the multiplicity
    \end{itemize}
\end{enumerate}
\end{method}

\section{The Superposition Principle}

\begin{theorem}[Superposition]
If $f(t) = f_1(t) + f_2(t) + \cdots + f_k(t)$ and $L[y_{p_i}] = f_i(t)$ for each $i$, then:
$$L[y_{p_1} + y_{p_2} + \cdots + y_{p_k}] = f_1(t) + f_2(t) + \cdots + f_k(t)$$
Therefore: $y_p = y_{p_1} + y_{p_2} + \cdots + y_{p_k}$
\end{theorem}

\begin{keypoint}
Handle each term in $f(t)$ separately, then add the particular solutions. This often simplifies the algebra significantly.
\end{keypoint}

\section{Comprehensive Examples}

\begin{example}[Polynomial Forcing]
Solve: $y'' - 3y' + 2y = 6t^2 - 4t + 1$

\textbf{Solution:}
\begin{enumerate}
    \item Homogeneous: $r^2 - 3r + 2 = 0 \Rightarrow r = 1, 2$

    Thus: y_h = c_1 e^t + c_2 $e^{2t}$

    \item Trial: y_p = At^2 + Bt + C$

    \item Derivatives: $y_p' = 2At + B$, $y_p'' = 2A$

    \item Substitute:
    $$2A - 3(2At + B) + 2(At^2 + Bt + C) = 6t^2 - 4t + 1$$
    $$2At^2 + (-6A + 2B)t + (2A - 3B + 2C) = 6t^2 - 4t + 1$$

    \item Match coefficients:
    \begin{align}
    t^2: & \quad 2A = 6 \Rightarrow A = 3 \\
    t^1: & \quad -6A + 2B = -4 \Rightarrow B = 7 \\
    t^0: & \quad 2A - 3B + 2C = 1 \Rightarrow C = 8
    \end{align}

    \item Solution: y = c_1 e^t + c_2 $e^{2t}$ + 3t^2 + 7t + 8
\end{enumerate}
\end{example}

\begin{example}[Resonant Exponential]
Solve: y'' - 4y' + 4y = $e^{2t}$

\textbf{Solution:}
\begin{enumerate}
    \item Homogeneous: (r-2)^2 = 0 \Rightarrow r = 2 (double root)

    Thus: y_h = (c_1 + c_2 t)$e^{2t}$

    \item Standard trial Ae^{2t} appears in y_h$ (resonance!)

    Since 2 is a double root, multiply by $t^2: y_p = At^2 $e^{2t}$

    \item Derivatives:
    \begin{align}
    y_p' &= 2Ate^{2t} + 2At^2e^{2t} = 2Ate^{2t}(1 + t) \\
    y_p'' &= 2Ae^{2t}(1 + 4t + 2t^2)
    \end{align}

    \item Substitute and simplify: 2Ae^{2t} = $e^{2t}$

    \item Thus A = 1/2 and y_p = \frac{t^2}{2}$e^{2t}$
\end{enumerate}
\end{example}

\begin{example}[Resonant Trigonometric]
Solve: y'' + \omega^2 y = F_0 \cos(\omega t)

\textbf{Solution:}
\begin{enumerate}
    \item Homogeneous: $r^2 + \omega^2 = 0 \Rightarrow r = \pm i\omega$

    Thus: $y_h = c_1 \cos(\omega t) + c_2 \sin(\omega t)$

    \item Resonance! Multiply by $t$: $y_p = t[A\cos(\omega t) + B\sin(\omega t)]$

    \item After substitution: $A = 0$, $B = \frac{F_0}{2\omega}$

    \item Solution exhibits linear growth: $y_p = \frac{F_0 t}{2\omega}\sin(\omega t)$
\end{enumerate}
\end{example}

\section{Common Pitfalls and Tips}

\begin{warning}
Common mistakes to avoid:
\begin{enumerate}
    \item Forgetting lower-degree terms in polynomial trials
    \item Omitting sine or cosine in trigonometric trials
    \item Missing resonance when the forcing matches homogeneous solutions
    \item Wrong multiplicity in resonance modification
    \item Arithmetic errors in coefficient matching
\end{enumerate}
\end{warning}

\begin{examtip}
Prof. Ditkowski's exam strategy:
\begin{enumerate}
    \item Always find $y_h$ first and write it clearly
    \item Check for resonance before writing trial solution
    \item For mixed forcing terms, use superposition
    \item Verify one coefficient as a check
    \item State the general solution explicitly
\end{enumerate}
\end{examtip}

\section{Complete Trial Solution Table}

\begin{formula}
\begin{center}
\begin{tabular}{|l|c|l|}
\hline
\textbf{Forcing $f(t)$} & \textbf{Char. Root?} & \textbf{Trial $y_p$} \\
\hline
$P_n(t)$ & $r = 0$ not a root & $A_n t^n + \cdots + A_0$ \\
& $r = 0$ mult. $m$ & $t^m(A_n t^n + \cdots + A_0) \\
\hline
$e^{at}$ & r = a not a root & Ae^{at} \\
& r = a mult. $m & At^m $e^{at}$ \\
\hline
\sin(bt), \cos(bt) & $r = \pm ib$ not roots & $A\cos(bt) + B\sin(bt)$ \\
& $r = \pm ib$ mult. $m$ & $t^m[A\cos(bt) + B\sin(bt)] \\
\hline
P_n(t)$e^{at}$ & r = a not a root & (A_n t^n + \cdots + A_0)$e^{at}$ \\
& r = a$ mult. $m & t^m(A_n t^n + \cdots + A_0)$e^{at}$ \\
\hline
e^{at}\sin(bt) & $r = a \pm ib not roots & $e^{at}$[A\cos(bt) + B\sin(bt)] \\
e^{at}\cos(bt) & r = a \pm ib$ mult. $m$ & t^m $e^{at}$[A\cos(bt) + B\sin(bt)] \\
\hline
\end{tabular}
\end{center}
\end{formula}

\end{document}