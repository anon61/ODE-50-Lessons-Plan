\documentclass[12pt]{article}
\usepackage{amsmath, amssymb, amsthm, tikz, pgfplots}
\usepackage{geometry, enumitem, mdframed, array, xcolor}
\usepackage{algorithm2e}
\geometry{margin=1in}

% Custom environments
\newtheorem{definition}{Definition}
\newtheorem{theorem}{Theorem}
\newtheorem{method}{Method}
\newtheorem{example}{Example}
\newmdenv[linecolor=blue,linewidth=2pt]{keypoint}
\newmdenv[linecolor=red,linewidth=2pt]{warning}
\newmdenv[linecolor=green,linewidth=2pt]{insight}
\newmdenv[linecolor=purple,linewidth=2pt]{examtip}
\newmdenv[linecolor=orange,linewidth=2pt]{frobenius}

\title{ODE Lesson 49: The Frobenius Method for Regular Singular Points}
\author{ODE 1 - Prof. Adi Ditkowski}
\date{}

\begin{document}
\maketitle

\section{Classification of Singular Points}

\begin{definition}[Singular Point Types]
For the ODE in standard form: $y'' + p(x)y' + q(x)y = 0$

A singular point $x = x_{0}$ is:
\begin{itemize}
\item \textbf{Regular} if $(x-x_{0})p(x)$ and $(x-x_{0})^{2q}(x)$ are both analytic at $x_{0}$
\item \textbf{Irregular} if at least one of these is not analytic at $x_{0}$
\end{itemize}
\end{definition}

\begin{keypoint}
Quick test for $x = 0$: Write the ODE as
$$x^{2y}'' + xP(x)y' + Q(x)y = 0$$
If $P(x)$ and $Q(x)$ are analytic at $x = 0$, then $x = 0$ is a regular singular point.
\end{keypoint}

\section{The Frobenius Method}

\begin{theorem}[Frobenius Theorem]
Near a regular singular point $x_{0}$, at least one solution has the form:
$$y = (x-x_{0})^{r} \sum_{n=0}^{\infty} a_{n}(x-x_{0})^{n}, \quad a_{0} \neq 0$$
where $r$ is determined by the indicial equation.
\end{theorem}

\begin{method}[Frobenius Procedure]
\begin{enumerate}
\item Verify $x_{0}$ is a regular singular point
\item Assume: $y = \sum_{n=0}^{\infty} a_{n}(x-x_{0})^{n+r}$ with $a_{0} \neq 0$
\item Compute derivatives:
$$y' = \sum_{n=0}^{\infty} (n+r)a_{n}(x-x_{0})^{n+r-1}$$
$$y'' = \sum_{n=0}^{\infty} (n+r)(n+r-1)a_{n}(x-x_{0})^{n+r-2}$$
\item Substitute into ODE and simplify
\item Find the indicial equation from the lowest power of $(x-x_{0})$
\item Solve for $r$ (roots $r_{1} \geq r_{2}$ if real)
\item Find recurrence relations for each root
\item Determine second solution based on root difference
\end{enumerate}
\end{method}

\section{The Indicial Equation}

\begin{frobenius}
For the ODE at $x = 0$: $x^{2y}'' + xP(x)y' + Q(x)y = 0$

With $P(x) = \sum_{k=0}^{\infty} p_{kx}^k$ and $Q(x) = \sum_{k=0}^{\infty} q_{kx}^k$

The indicial equation is:
$$r(r-1) + p_{0r} + q_{0} = 0$$
or equivalently:
$$r^{2} + (p_{0}-1)r + q_{0} = 0$$
\end{frobenius}

\section{Three Cases for Solutions}

\begin{keypoint}
Let $r_{1}$ and $r_{2}$ be roots of the indicial equation with $\text{Re}(r_{1}) \geq \text{Re}(r_{2})$.
\end{keypoint}

\subsection{Case 1: $r_{1} - r_{2} \neq$ integer}

Two independent Frobenius series solutions:
$$y_{1} = x^{r_{1}}\sum_{n=0}^{\infty} a_{n} x^{n}, \quad y_{2} = x^{r_{2}}\sum_{n=0}^{\infty} b_{n} x^{n}$$

\subsection{Case 2: $r_{1} = r_{2} = r$ (repeated roots)}

First solution: $y_{1} = x^{r}\sum_{n=0}^{\infty} a_{n} x^{n}$

Second solution: $y_{2} = y_{1}\ln(x) + x^{r}\sum_{n=1}^{\infty} c_{n} x^{n}$

\subsection{Case 3: $r_{1} - r_{2} = N$ (positive integer)}

First solution: $y_{1} = x^{r_{1}}\sum_{n=0}^{\infty} a_{n} x^{n}$

Second solution may be:
\begin{itemize}
\item Pure Frobenius: $y_{2} = x^{r_{2}}\sum_{n=0}^{\infty} b_{n} x^{n}$ (if no conflict)
\item With logarithm: $y_{2} = Cy_{1}\ln(x) + x^{r_{2}}\sum_{n=0}^{\infty} b_{n} x^{n}$
\end{itemize}

\begin{warning}
In Case 3, you must check if the recurrence relation for $r_{2}$ leads to a contradiction at $n = N$. If it does, the logarithmic term is necessary with $C \neq 0$.
\end{warning}

\section{Important Examples}

\subsection{Bessel's Equation of Order $\nu$}

$$x^{2y}'' + xy' + (x^{2} - \nu^{2})y = 0$$

\begin{frobenius}
\begin{itemize}
\item Indicial equation: $r^{2} - \nu^{2} = 0$
\item Roots: $r_{1} = \nu$, $r_{2} = -\nu$
\item If $\nu \neq$ integer: Two Frobenius series $J_{\nu}(x)$ and $J_{-\nu}(x)$
\item If $\nu = n$ (integer): Second solution $Y_{n}(x)$ contains $\ln(x)$
\end{itemize}
\end{frobenius}

\subsection{Euler's Equation}

$$x^{2y}'' + \alpha xy' + \beta y = 0$$

\begin{insight}
Euler's equation has constant $P(x) = \alpha$ and $Q(x) = \beta$, making it the simplest equation with a regular singular point. Solutions are $y = x^{r}$ where $r$ satisfies $r(r-1) + \alpha r + \beta = 0$.
\end{insight}

\section{Systematic Approach for Exam Problems}

\begin{examtip}
Prof. Ditkowski's typical Frobenius problem structure:
\begin{enumerate}
\item[Step 1:] Identify and classify singular points (2 pts)
\item[Step 2:] Find indicial equation and roots (3 pts)
\item[Step 3:] State which case applies (1 pt)
\item[Step 4:] Write general form of solutions (2 pts)
\item[Step 5:] Find first 2-3 coefficients if asked (2 pts)
\end{enumerate}
Total: 8-10 points
\end{examtip}

\section{Recurrence Relations in Frobenius Method}

For $y = \sum_{n=0}^{\infty} a_{n} x^{n+r}$, after substitution:

General form near $x = 0$:
$$\sum_{n=0}^{\infty} \left[(n+r)(n+r-1) + p_{0}(n+r) + q_{0}\right]a_{n} x^{n+r} + \text{higher order terms} = 0$$

\begin{algorithm}[H]
\SetAlgoLined
\KwIn{Regular singular point ODE}
\KwOut{Frobenius series solution}
Write ODE as $x^{2y}'' + xP(x)y' + Q(x)y = 0$\;
Expand $P(x) = p_{0} + p_{1x} + \cdots$ and $Q(x) = q_{0} + q_{1x} + \cdots$\;
Form indicial equation: $r^{2} + (p_{0}-1)r + q_{0} = 0$\;
Solve for $r_{1}, r_{2}$\;
\If{$r_{1} - r_{2} \notin \mathbb{Z}$}{
    Find two Frobenius series\;
}
\ElseIf{$r_{1} = r_{2}$}{
    Find one series, second has $\ln(x)$\;
}
\Else{
    Check for logarithmic case\;
}
\caption{Frobenius Method Algorithm}
\end{algorithm}

\section{Common Equations and Their Indicial Equations}

\begin{center}
\begin{tabular}{|l|c|c|}
\hline
\textbf{Equation Name} & \textbf{Standard Form at $x=0$} & \textbf{Indicial Equation} \\
\hline
Bessel order $\nu$ & $x^{2y}'' + xy' + (x^{2}-\nu^{2})y = 0$ & $r^{2} - \nu^{2} = 0$ \\
\hline
Modified Bessel & $x^{2y}'' + xy' - (x^{2}+\nu^{2})y = 0$ & $r^{2} - \nu^{2} = 0$ \\
\hline
Laguerre & $xy'' + (1-x)y' + ny = 0$ & $r = 0$ (repeated) \\
\hline
Euler & $x^{2y}'' + \alpha xy' + \beta y = 0$ & $r^{2} + (\alpha-1)r + \beta = 0$ \\
\hline
\end{tabular}
\end{center}

\end{document}