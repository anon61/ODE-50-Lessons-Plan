\documentclass[12pt]{article}
\usepackage{amsmath, amssymb, amsthm, geometry, enumitem}
\geometry{margin=1in}

\title{Lesson 49: Practice Problems - Frobenius Method}
\author{ODE 1 - Prof. Adi Ditkowski}
\date{}

\usepackage{mdframed}
\usepackage{xcolor}
\usepackage{tikz}
\begin{document}
\maketitle

\section*{Part A: Classifying Singular Points (6 problems)}

\begin{enumerate}
\item Classify all singular points (regular or irregular) for:
$$x$^{2}$(x-1)y'' + 2xy' + y = 0$$

\item For Bessel's equation $x$^{2y}$'' + xy' + (x$^{2}$-4)y = 0$, verify that $x = 0$ is a regular singular point.

\item Determine the nature of $x = 0$ for:
$$x$^{3y}$'' + xy' + y = 0$$

\item Show that both $x = 0$ and $x = 1$ are regular singular points of:
$$x(1-x)y'' + [c-(a+b+1)x]y' - aby = 0$$
(This is the hypergeometric equation)

\item Classify the singular point at $x = 0$ for:
$$x$^{2y}$'' + \sin(x)y' + xy = 0$$

\item Find and classify all singular points:
$$(x$^{2}$-1)$^{2y}$'' + (x-1)y' + y = 0$$
\end{enumerate}

\section*{Part B: Finding Indicial Equations (6 problems)}

\begin{enumerate}[resume]
\item Find the indicial equation at $x = 0$ for:
$$xy'' + y' - y = 0$$

\item Determine the indicial equation for:
$$x$^{2y}$'' + x(1+x)y' - y = 0$$

\item Find the indicial equation and its roots for:
$$2x$^{2y}$'' + xy' - (1+x)y = 0$$

\item For the equation $x$^{2y}$'' + 3xy' + (1-x)y = 0$, find the indicial equation and identify which case applies.

\item Find the indicial equation at $x = 1$ for:
$$(x-1)$^{2y}$'' + (x-1)y' + y = 0$$

\item Determine the indicial equation for Laguerre's equation:
$$xy'' + (1-x)y' + ny = 0$$
\end{enumerate}

\section*{Part C: Determining Solution Forms (5 problems)}

\begin{enumerate}[resume]
\item For $x$^{2y}$'' + xy' + (x$^{2}$-1/4)y = 0$, find the roots of the indicial equation and state the form of the general solution.

\item Given the indicial roots $r$_{1}$ = 3$ and $r$_{2}$ = -2$, write the general form of the solution.

\item If the indicial equation has repeated root $r = 1/2$, write the form of both linearly independent solutions.

\item For Bessel's equation of order 2, explain why the second solution must contain $\ln(x)$.

\item Given roots $r$_{1}$ = 1$ and $r$_{2}$ = 0$, determine whether a logarithmic term is needed by checking the recurrence at $n = 1$.
\end{enumerate}

\section*{Part D: Computing Frobenius Series (5 problems)}

\begin{enumerate}[resume]
\item Find the first three non-zero terms of the Frobenius series solution for:
$$xy'' + 2y' + xy = 0$$ using the larger root.

\item For $x$^{2y}$'' + xy' + (x$^{2}$-1)y = 0$, find the recurrence relation for the root $r = 1$.

\item Solve using Frobenius method:
$$2xy'' + (1+2x)y' + y = 0$$
Find coefficients $a$_{0}$, a$_{1}$, a$_{2}$$ for the larger root.

\item For the equation $x$^{2y}$'' + x$^{2y}$' - 2y = 0$:
\begin{enumerate}[label=(\alph*)]
\item Find the indicial equation
\item Find the recurrence relation for $r = 2$
\item Compute the first four coefficients
\end{enumerate}

\item Apply Frobenius method to find one solution of:
$$x$^{2y}$'' + x(x+1)y' - y = 0$$
\end{enumerate}

\section*{Part E: Special Cases and Logarithmic Solutions (3 problems)}

\begin{enumerate}[resume]
\item Show that for $x$^{2y}$'' + 3xy' + (1+x)y = 0$ with repeated root $r = -1$, the second solution must contain $\ln(x)$.

\item For the equation with roots differing by an integer:
$$x$^{2y}$'' + xy' - y = 0$$
Determine if both solutions can be pure Frobenius series or if logarithms are needed.

\item Verify that Euler's equation $x$^{2y}$'' - xy' + y = 0$ has solutions $y$_{1}$ = x$ and $y$_{2}$ = x\ln(x)$.
\end{enumerate}

\section*{Part F: Exam-Style Problems (5 problems)}

\begin{enumerate}[resume]
\item \textbf{[10 points]} Consider the modified Bessel equation:
$$x$^{2y}$'' + xy' - (x$^{2}$ + n$^{2}$)y = 0$$ where $n = 2$.
\begin{enumerate}[label=(\alph*)]
\item [2 pts] Show that $x = 0$ is a regular singular point
\item [3 pts] Find the indicial equation and its roots
\item [2 pts] Which case applies for the general solution?
\item [3 pts] Write the form of both linearly independent solutions
\end{enumerate}

\item \textbf{[9 points]} For the equation:
$$x(x-1)y'' + 3y' + y = 0$$
\begin{enumerate}[label=(\alph*)]
\item [3 pts] Find and classify all singular points
\item [3 pts] Find the indicial equation at $x = 0$
\item [3 pts] Find the first three terms of the Frobenius series for the larger root
\end{enumerate}

\item \textbf{[8 points]} Given:
$$2x$^{2y}$'' + x(1+x)y' - 2y = 0$$
\begin{enumerate}[label=(\alph*)]
\item [3 pts] Find the indicial equation and roots
\item [2 pts] Set up the recurrence relation
\item [3 pts] Determine if the second solution requires logarithms
\end{enumerate}

\item \textbf{[10 points]} \textit{Comprehensive Problem}
$$x$^{2y}$'' + x(1-x)y' - (1+3x)y = 0$$
\begin{enumerate}[label=(\alph*)]
\item [2 pts] Verify $x = 0$ is a regular singular point
\item [3 pts] Find the indicial equation and solve for $r$
\item [3 pts] Find the recurrence relation for the larger root
\item [2 pts] State the form of the general solution
\end{enumerate}

\item \textbf{[12 points]} \textit{Prof. Ditkowski Special - Hypergeometric Type}
Consider: $x(1-x)y'' + [2-(3+x)]y' - y = 0$
\begin{enumerate}[label=(\alph*)]
\item [3 pts] Show both $x = 0$ and $x = 1$ are regular singular points
\item [3 pts] Find the indicial equation at $x = 0$
\item [3 pts] Find the indicial equation at $x = 1$
\item [3 pts] Around which point would you prefer to expand and why?
\end{enumerate}
\end{enumerate}

\section*{Solutions and Hints}

\textbf{Selected Solutions:}

\textbf{Problem 1:}
- $x = 0$: Check $xp(x) = 2x$^{2}$/(x(x-1))$ and $x$^{2q}$(x) = x$^{2}$/(x(x-1))$ at $x = 0$ \rightarrow Regular
- $x = 1$: Check $(x-1)p(x)$ and $(x-1)$^{2q}$(x)$ at $x = 1$ \rightarrow Regular

\textbf{Problem 7:}
Standard form: $y'' + (1/x)y' - (1/x)y = 0$
- $p$_{0}$ = 1$, $q$_{0}$ = 0$
- Indicial equation: $r(r-1) + r = 0$ \rightarrow $r$^{2}$ = 0$ \rightarrow $r = 0$ (repeated)

\textbf{Problem 13:}
- Indicial equation: $r$^{2}$ - 1/4 = 0$
- Roots: $r$_{1}$ = 1/2$, $r$_{2}$ = -1/2$ (differ by 1)
- Form: Check if pure Frobenius works for $r$_{2}$$ or needs log term

\textbf{Problem 18:}
For larger root $r = 0$:
- $y = a$_{0}$(1 - x$^{2}$/2 + x$^{4}$/24 - \cdots)$
- This gives the Bessel function $J$_{0}$(x)$ series

\textbf{Problem 26:}
- Indicial roots: $r = 2, -1$ (differ by 3)
- At $x = 0$: Regular singular point
- Second solution likely needs no logarithm (check recurrence)

\textbf{Key Insights:}
- Always check $r$_{1}$ - r$_{2}$$ first
- Integer differences require careful analysis
- Bessel-type equations are exam favorites
- When $r = 0$ appears, one solution is a regular power series

\end{document}