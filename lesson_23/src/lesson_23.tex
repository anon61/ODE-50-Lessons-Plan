\documentclass[12pt]{article}
\usepackage{amsmath, amssymb, amsthm, tikz, pgfplots}
\usepackage{geometry, enumitem, mdframed, array, xcolor}
\geometry{margin=1in}

% Custom environments
\newtheorem{definition}{Definition}
\newtheorem{theorem}{Theorem}
\newtheorem{method}{Method}
\newtheorem{example}{Example}
\newmdenv[linecolor=blue,linewidth=2pt]{keypoint}
\newmdenv[linecolor=red,linewidth=2pt]{warning}
\newmdenv[linecolor=green,linewidth=2pt]{insight}
\newmdenv[linecolor=purple,linewidth=2pt]{examtip}
\newmdenv[linecolor=cyan,linewidth=2pt]{algorithm}

\title{ODE Lesson 23: Integrating Factors - $\mu(x)$ and $\mu(y)$ Cases}
\author{ODE 1 - Prof. Adi Ditkowski}
\date{}

\begin{document}
\maketitle

\section{The Concept of Integrating Factors}

\begin{definition}[Integrating Factor]
An \textbf{integrating factor} $\mu(x,y)$ for the equation
\[M(x,y)dx + N(x,y)dy = 0\]
is a function such that the equation
\[\mu(x,y)M(x,y)dx + \mu(x,y)N(x,y)dy = 0\]
is exact.
\end{definition}

\begin{keypoint}
Multiplying by an integrating factor doesn't change the solutions - it only changes the form of the equation. If $y = f(x)$ is a solution to the original equation, it remains a solution to the modified equation.
\end{keypoint}

\section{Condition for Exactness After Multiplication}

For $\mu M dx + \mu N dy = 0$ to be exact, we need:
\[\frac{\partial(\mu M)}{\partial y} = \frac{\partial(\mu N)}{\partial x}\]

Expanding using the product rule:
\[\mu \frac{\partial M}{\partial y} + M\frac{\partial \mu}{\partial y} = \mu \frac{\partial N}{\partial x} + N\frac{\partial \mu}{\partial x}\]

Rearranging:
\[M\frac{\partial \mu}{\partial y} - N\frac{\partial \mu}{\partial x} = \mu\left(\frac{\partial N}{\partial x} - \frac{\partial M}{\partial y}\right)\]

\begin{warning}
This is a partial differential equation for $\mu$ - generally very difficult to solve! We look for special cases where $\mu$ depends on only one variable.
\end{warning}

\section{Case 1: Integrating Factor $\mu(x)$}

\begin{theorem}[Existence of $\mu(x)$]
An integrating factor depending only on $x$ exists if and only if
\[\frac{1}{N}\left(\frac{\partial M}{\partial y} - \frac{\partial N}{\partial x}\right) = g(x)\]
where $g(x)$ is a function of $x alone. The integrating factor is then:
\[\mu(x) = e^{\int g(x)\,dx}\]
\end{theorem}

\begin{proof}
If \mu = \mu(x)$, then $\frac{\partial \mu}{\partial y} = 0$. The exactness condition becomes:
\[-N\frac{d\mu}{dx} = \mu\left(\frac{\partial N}{\partial x} - \frac{\partial M}{\partial y}\right)\]
\[\frac{1}{\mu}\frac{d\mu}{dx} = \frac{1}{N}\left(\frac{\partial M}{\partial y} - \frac{\partial N}{\partial x}\right)\]
This is solvable only if the right side depends solely on $x$.
\end{proof}

\begin{algorithm}
\textbf{Finding $\mu(x)$:}
\begin{enumerate}
    \item Compute $\frac{\partial M}{\partial y}$ and $\frac{\partial N}{\partial x}$
    \item Calculate $R(x,y) = \frac{M_{y} - N_{x}}{N}$
    \item If $R = g(x)$ (function of $x only), then:
    \[\mu(x) = e^{\int g(x)\,dx}\]
    \item If R$ contains $y$, then $\mu(x)$ doesn't exist
\end{enumerate}
\end{algorithm}

\section{Case 2: Integrating Factor $\mu(y)$}

\begin{theorem}[Existence of $\mu(y)$]
An integrating factor depending only on $y$ exists if and only if
\[\frac{1}{M}\left(\frac{\partial N}{\partial x} - \frac{\partial M}{\partial y}\right) = h(y)\]
where $h(y)$ is a function of $y alone. The integrating factor is:
\[\mu(y) = e^{\int h(y)\,dy}\]
\end{theorem}

\begin{algorithm}
\textbf{Finding \mu(y)$:}
\begin{enumerate}
    \item Compute $\frac{\partial M}{\partial y}$ and $\frac{\partial N}{\partial x}$
    \item Calculate $S(x,y) = \frac{N_{x} - M_{y}}{M}$
    \item If $S = h(y)$ (function of $y only), then:
    \[\mu(y) = e^{\int h(y)\,dy}\]
    \item If S$ contains $x$, then $\mu(y)$ doesn't exist
\end{enumerate}
\end{algorithm}

\section{Complete Solution Process}

\begin{keypoint}
\textbf{Step-by-Step Solution with Integrating Factors:}
\begin{enumerate}
    \item Test for exactness (if exact, skip to step 6)
    \item Check if $\mu(x)$ exists: Is $(M_{y} - N_{x})/N$ a function of $x$ only?
    \item If not, check if $\mu(y)$ exists: Is $(N_{x} - M_{y})/M$ a function of $y$ only?
    \item Find the integrating factor using the appropriate formula
    \item Multiply the original equation by $\mu$
    \item Verify the new equation is exact
    \item Solve the exact equation using methods from Lesson 22
\end{enumerate}
\end{keypoint}

\section{Important Examples}

\begin{example}[Standard $\mu(x)$ Case]
Solve $(2y + 3x^{2})dx + xdy = 0$

\textbf{Step 1:} Test exactness: $M_{y} = 2$, $N_{x} = 1$. Not exact!

\textbf{Step 2:} Check for $\mu(x)$:
\[\frac{M_{y} - N_{x}}{N} = \frac{2 - 1}{x} = \frac{1}{x}\]
This is a function of $x$ only!

\textbf{Step 3:} Find $\mu(x):
\[\mu(x) = $e^{\int \frac{1}${x}dx} = $e^{\ln|x|}$ = x\]

\textbf{Step 4:} Multiply by \mu = x$:
\[(2xy + 3x^{3})dx + x^{2dy} = 0\]

\textbf{Step 5:} Verify exactness: $M_{y} = 2x$, $N_{x} = 2x$. $$\checkmark$$

\textbf{Step 6:} Find potential function:
\[H = \int x^{2}\,dy = x^{2y} + f(x)\]
\[\frac{\partial H}{\partial x} = 2xy + f'(x) = 2xy + 3x^{3}\]
\[f'(x) = 3x^{3} \Rightarrow f(x) = \frac{3x^{4}}{4}\]

\textbf{Solution:} $x^{2y} + \frac{3x^{4}}{4} = C$
\end{example}

\begin{example}[Linear Equation Connection]
The linear equation $y' + P(x)y = Q(x)$ can be written as:
\[(Py - Q)dx + dy = 0\]

Check for $\mu(x)$:
\[\frac{M_{y} - N_{x}}{N} = \frac{P - 0}{1} = P(x)\]

Therefore: \mu(x) = $e^{\int P(x)dx}$ - exactly the integrating factor from Block 5!
\end{example}

\section{Common Patterns to Recognize}

\begin{examtip}
\textbf{Quick Recognition Guide:}
\begin{center}
\begin{tabular}{|l|c|c|}
\hline
\textbf{If you see} & \textbf{Try} & \textbf{Integrating Factor} \\
\hline
$N = x^{n}$ & $\mu(x)$ & Often $\mu = x^{k}$ \\
$M = y^{n}$ & $\mu(y)$ & Often $\mu = y^{k}$ \\
Linear in $y$ & $\mu(x)$ & \mu = $e^{\int P(x)dx}$ \\
Homogeneous & Either & Check both tests \\
$N = f(x)$ only & $\mu(x)$ & Guaranteed to exist \\
$M = g(y)$ only & $\mu(y)$ & Guaranteed to exist \\
\hline
\end{tabular}
\end{center}
\end{examtip}

\section{Memory Aids}

\begin{insight}
\textbf{Mnemonic Devices:}
\begin{itemize}
    \item "$\mu(x)$: My Nexus over N" - $(M_{y} - N_{x})/N$ for $x$ dependence
    \item "$\mu(y)$: Nexus My over M" - $(N_{x} - M_{y})/M$ for $y$ dependence
    \item Notice: Numerators are negatives of each other!
    \item The variable in $\mu$ matches what you divide by (sort of):
    \begin{itemize}
        \item Divide by $N$ (has $x$ in deNominator) $\rightarrow$ $\mu(x)$
        \item Divide by $M$ (has $y$ sound in naMe) $\rightarrow$ $\mu(y)$
    \end{itemize}
\end{itemize}
\end{insight}

\section{Verification is Crucial}

\begin{warning}
After finding an integrating factor, ALWAYS:
\begin{enumerate}
    \item Multiply the original equation by $\mu$
    \item Verify the new equation is exact by checking $\frac{\partial(\mu M)}{\partial y} = \frac{\partial(\mu N)}{\partial x}$
    \item Only then proceed to find the potential function
\end{enumerate}
Skipping verification is a common source of errors!
\end{warning}

\end{document}