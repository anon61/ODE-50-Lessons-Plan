\documentclass[12pt]{article}
\usepackage{amsmath, amssymb, enumitem, geometry, xcolor}
\geometry{margin=1in}

\title{Lesson 23: Practice Problems - Integrating Factors $\mu(x)$ and $\mu(y)$}
\author{ODE 1 - Prof. Adi Ditkowski}
\date{}

\usepackage{mdframed}
\usepackage{tikz}
\begin{document}
\maketitle

\section*{Part A: Testing for $\mu(x)$ and $\mu(y)$ (Problems 1-6)}

\begin{enumerate}
\item For $(3x + 2y)dx + xdy = 0$:
\begin{enumerate}[label=(\alph*)]
    \item Show the equation is not exact
    \item Test if $\mu(x)$ exists
    \item Test if $\mu(y)$ exists
    \item Find the integrating factor
\end{enumerate}

\item For $(y^{2} + 2xy)dx + xydy = 0$:
\begin{enumerate}[label=(\alph*)]
    \item Verify non-exactness
    \item Determine which type of integrating factor exists
    \item Find and apply the integrating factor
\end{enumerate}

\item Test both $\mu(x)$ and $\mu(y)$ for: $(2y)dx + (3x + 4y^{2})dy = 0$

\item Test both $\mu(x)$ and $\mu(y)$ for: $(xy + 1)dx + (x^{2} - 1)dy = 0$

\item For $ydx - xdy = 0$, show that both $\mu(x) = 1/x^{2}$ and $\mu(y) = 1/y^{2}$ work.

\item Determine all possible integrating factors of the form $\mu(x)$ for:
$(2y)dx + xdy = 0$
\end{enumerate}

\section*{Part B: Finding and Using $\mu(x)$ (Problems 7-12)}

\begin{enumerate}[start=7]
\item Solve $(2xy + y^{2})dx + xdy = 0$ by finding $\mu(x)$

\item Solve $(3y + 2x)dx + xdy = 0$ using an integrating factor

\item Solve $(y + x^{2})dx + 2xdy = 0$

\item Solve $(2y + 3x$^{2y}$)dx + xdy = 0$

\item Find $\mu(x)$ and solve: $(e$^{y}$ + 2x)dx + xe$^{y}$ dy = 0$

\item Solve the initial value problem: $(y + x^{3})dx + 2xdy = 0$, $y(1) = 2$
\end{enumerate}

\section*{Part C: Finding and Using $\mu(y)$ (Problems 13-18)}

\begin{enumerate}[start=13]
\item Solve $ydx + (2x + 3y^{2})dy = 0$ by finding $\mu(y)$

\item Solve $2ydx + (3x - y)dy = 0$ using an integrating factor

\item Solve $(y^{2} + 1)dx + xydy = 0$

\item Find $\mu(y)$ and solve: $\sin y\,dx + (x\cos y + 1)dy = 0$

\item Solve $(2y^{3})dx + (3xy^{2} - 1)dy = 0$

\item Solve the IVP: $ydx + (3x - 2y^{2})dy = 0$, $y(0) = 1$
\end{enumerate}

\section*{Part D: Choice Between $\mu(x)$ and $\mu(y)$ (Problems 19-23)}

\begin{enumerate}[start=19]
\item For $(2xy^{2} + y)dx + xdy = 0$:
\begin{enumerate}[label=(\alph*)]
    \item Show both $\mu(x)$ and $\mu(y)$ exist
    \item Find both integrating factors
    \item Solve using each and verify same solution
\end{enumerate}

\item Find the simpler integrating factor and solve:
$(3x$^{2y}$ + 2y^{2})dx + x$^{3dy}$ = 0$

\item Choose the appropriate integrating factor for:
$(y\cos x + 1)dx + \sin x\,dy = 0$

\item For $(ax + by^{2})dx + ydy = 0$, find conditions on $a$ and $b$ for:
\begin{enumerate}[label=(\alph*)]
    \item $\mu(x)$ to exist
    \item $\mu(y)$ to exist
\end{enumerate}

\item Solve by choosing the simpler integrating factor:
$(x$^{2y}$^3 + 2y)dx + xdy = 0$
\end{enumerate}

\section*{Part E: Linear Equation Connection (Problems 24-26)}

\begin{enumerate}[start=24]
\item Show that $y' + \frac{2}{x}y = x^{2}$ leads to $\mu(x) = x^{2}$ and solve.

\item Convert to standard form and find integrating factor:
$xy' - 2y = x$^{3e}$^x$

\item Show that every linear equation $y' + P(x)y = Q(x)$ has $\mu(x) = $e^{\int$ P(x)dx}$
\end{enumerate}

\section*{Part F: Exam-Style Problems (Problems 27-32)}

\begin{enumerate}[start=27]
\item (Prof. Ditkowski 2023) Given that $\mu = x$^{n}$$ is an integrating factor for $(2xy + y^{3})dx + (x^{2} + xy^{2})dy = 0$, find $n$ and solve.

\item Show that $(3x$^{2y}$ + y^{2})dx + (x^{3} + xy)dy = 0$ becomes exact when multiplied by $\mu = 1/xy$. Is this $\mu(x)$ or $\mu(y)$? Explain.

\item Find all integrating factors of the form $\mu = x$^{a}$ y$^{b}$$ for:
$2ydx + xdy = 0$

\item Given $(f(x) + 2y)dx + xdy = 0$ has $\mu(x) = x^{2}$:
\begin{enumerate}[label=(\alph*)]
    \item Find $f(x)$
    \item Solve the equation
\end{enumerate}

\item For what value of $k$ does $(ky + x^{2})dx + (2x + y^{2})dy = 0$ have:
\begin{enumerate}[label=(\alph*)]
    \item An integrating factor $\mu(x)$?
    \item An integrating factor $\mu(y)$?
\end{enumerate}

\item A student claims that if an equation has both $\mu(x)$ and $\mu(y)$, then it must be exact. Prove or disprove with an example.
\end{enumerate}

\section*{Solutions and Key Insights}

\textbf{Problem 1:}
(a) $M$_{y}$ = 2$, $N$_{x}$ = 1$, not equal \rightarrow not exact
(b) $(M$_{y}$ - N$_{x}$)/N = (2-1)/x = 1/x$ \rightarrow $\mu(x)$ exists!
(c) $(N$_{x}$ - M$_{y}$)/M = (1-2)/(3x+2y)$ \rightarrow contains both $x$ and $y$, no $\mu(y)$
(d) $\mu(x) = $e^{\int(1/x)dx}$ = x$

\textbf{Problem 7:}
Test: $(M$_{y}$ - N$_{x}$)/N = (2x+2y-1)/x = (2x+2y-1)/x$
This contains $y$, so no $\mu(x)$... Wait! Let's recheck:
$M = 2xy + y^{2}$, $N = x$
$(M$_{y}$ - N$_{x}$)/N = (2x + 2y - 0)/x = 2 + 2y/x$
Still has $y$. Try $\mu(y)$:
$(N$_{x}$ - M$_{y}$)/M = (1 - 2x - 2y)/(2xy + y^{2}) = -1/y$
So $\mu(y) = $e^{\int(-1/y)dy}$ = 1/y$

\textbf{Problem 19:}
For $(2xy^{2} + y)dx + xdy = 0$:
(a) $(M$_{y}$ - N$_{x}$)/N = (4xy + 1 - 1)/x = 4y$ \rightarrow No $\mu(x)$
Actually, let me recalculate: $M$_{y}$ = 4xy + 1$, $N$_{x}$ = 1$
$(M$_{y}$ - N$_{x}$)/N = 4xy/x = 4y$ \rightarrow No $\mu(x)$
$(N$_{x}$ - M$_{y}$)/M = (1 - 4xy - 1)/(2xy^{2} + y) = -4xy/(y(2xy + 1)) = -4x/(2xy + 1)$
Hmm, this is complex. Let me reconsider the original equation...

\textbf{Problem 24:}
$y' + (2/x)y = x^{2}$ becomes $(2y/x - x^{2})dx + dy = 0$
$(M$_{y}$ - N$_{x}$)/N = (2/x - 0)/1 = 2/x$
$\mu(x) = $e^{\int(2/x)dx}$ = $e^{2\ln|x|}$ = x^{2}$
Multiply: $(2xy - x^{4})dx + x$^{2dy}$ = 0$
Now exact! $H = x$^{2y}$ - x^{5}/5$
Solution: $x$^{2y}$ - x^{5}/5 = C$ or $y = x^{3}/5 + C/x^{2}$

\textbf{Key Strategy:} When both tests give functions of mixed variables, neither $\mu(x)$ nor $\mu(y)$ exists. Move to Lesson 24 for special forms!

\textbf{Warning:} Problem 32 - Counterexample: $2ydx + xdy = 0$ has both types but is not exact!

\end{document}