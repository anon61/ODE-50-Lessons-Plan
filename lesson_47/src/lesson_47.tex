\documentclass[12pt]{article}
\usepackage{amsmath, amssymb, amsthm, tikz, pgfplots}
\usepackage{geometry, enumitem, mdframed, array, xcolor}
\geometry{margin=1in}

% Custom environments
\newtheorem{definition}{Definition}
\newtheorem{theorem}{Theorem}
\newtheorem{lemma}{Lemma}
\newtheorem{method}{Method}
\newtheorem{example}{Example}
\newmdenv[linecolor=blue,linewidth=2pt]{keypoint}
\newmdenv[linecolor=red,linewidth=2pt]{warning}
\newmdenv[linecolor=green,linewidth=2pt]{insight}
\newmdenv[linecolor=purple,linewidth=2pt]{examtip}
\newmdenv[linecolor=orange,linewidth=2pt]{formula}

\title{ODE Lesson 47: Euler-Cauchy Equations}
\author{ODE 1 with Prof. Adi Ditkowski}
\date{Tel Aviv University}

\begin{document}

\maketitle

\section{Introduction}

\begin{definition}[Euler-Cauchy Equation]
An $n$-th order Euler-Cauchy equation (also called an equidimensional equation) has the form:
$$t^{n} y^{(n)} + a_{n-1} t^{n-1} y^{(n-1)} + \cdots + a_{1} t y' + a_{0} y = 0$$
where $a_{0}, a_{1}, \ldots, a_{n-1}$ are constants and $t \neq 0$.
\end{definition}

\begin{keypoint}
The defining characteristic: the power of $t$ equals the order of the derivative in each term. This creates a scaling symmetry that makes these equations solvable by elementary methods.
\end{keypoint}

\section{Method 1: Power Function Ansatz}

\begin{theorem}[Power Solutions]
For the Euler equation, solutions have the form $y = t^{r}$ where $r$ satisfies an algebraic characteristic equation.
\end{theorem}

\subsection{Second-Order Case}

Consider the second-order Euler equation:
$$t^{2} y'' + a t y' + b y = 0, \quad t > 0$$

\begin{method}[Direct Solution Method]
\begin{enumerate}
    \item Assume $y = t^{r}$ for some constant $r$
    \item Compute derivatives:
    \begin{align}
    y &= t^{r} \\
    y' &= r t^{r-1} \\
    y'' &= r(r-1) t^{r-2}
    \end{align}
    \item Substitute into the equation:
    $$t^{2} \cdot r(r-1)t^{r-2} + at \cdot rt^{r-1} + b \cdot t^{r} = 0$$
    \item Simplify:
    $$t^{r}[r(r-1) + ar + b] = 0$$
    \item Since $t^{r} \neq 0$ for $t > 0$, we get the characteristic equation:
    $$r^{2} + (a-1)r + b = 0$$
\end{enumerate}
\end{method}

\begin{formula}
\textbf{Characteristic Equation for Second-Order Euler:}
$$\boxed{r(r-1) + ar + b = 0 \quad \text{or} \quad r^{2} + (a-1)r + b = 0}$$
\end{formula}

\subsection{Solution Forms Based on Roots}

\begin{theorem}[Complete Solution Classification]
Let $r_{1}, r_{2}$ be roots of the characteristic equation. Then:
\begin{enumerate}
    \item \textbf{Distinct real roots}: $y = c_{1} t^{r_{1}} + c_{2} t^{r_{2}}$
    \item \textbf{Repeated real root} $r$: $y = (c_{1} + c_{2} \ln t) t^{r}$
    \item \textbf{Complex roots} $r = \alpha \pm i\beta$:
    $$y = t^\alpha [c_{1} \cos(\beta \ln t) + c_{2} \sin(\beta \ln t)]$$
\end{enumerate}
\end{theorem}

\begin{proof}[Proof of Repeated Root Case]
When $r$ is a repeated root, we use reduction of order. Given $y_{1} = t^{r}$, we seek $y_{2} = v(t) \cdot t^{r}$.

The Euler equation in standard form:
$$y'' + \frac{a}{t}y' + \frac{b}{t^{2}}y = 0$$

Using the reduction formula with $p(t) = a/t$:
$v = \int \frac{$e^{-\int (a/t) dt}$}{t^{2r}} dt = \int \frac{t^{-a}}{t^{2r}} dt$

For a repeated root, $r = (1-a)/2$, so $2r = 1-a$. Thus:
$$v = \int \frac{t^{-a}}{t^{1-a}} dt = \int \frac{1}{t} dt = \ln t$$

Therefore, $y_{2} = t^{r} \ln t$.
\end{proof}

\section{Method 2: Logarithmic Transformation}

\begin{theorem}[Transformation to Constant Coefficients]
The substitution $x = \ln t (equivalently, t = $e^{x}$) transforms an Euler equation into a constant coefficient equation.
\end{theorem}

\subsection{Derivative Transformations}

\begin{lemma}[Change of Variables]
Under the substitution x = \ln t$, if $v(x) = y(t)$:
\begin{align}
t \frac{dy}{dt} &= \frac{dv}{dx} \\
t^{2} \frac{d^{2y}}{dt^{2}} &= \frac{d^{2v}}{dx^{2}} - \frac{dv}{dx} \\
t^{3} \frac{d^{3y}}{dt^{3}} &= \frac{d^{3v}}{dx^{3}} - 3\frac{d^{2v}}{dx^{2}} + 2\frac{dv}{dx}
\end{align}
\end{lemma}

\begin{proof}[Proof for Second Derivative]
Using the chain rule:
$$\frac{dy}{dt} = \frac{dv}{dx} \cdot \frac{dx}{dt} = \frac{1}{t} \frac{dv}{dx}$$

For the second derivative:
$$\frac{d^{2y}}{dt^{2}} = \frac{d}{dt}\left(\frac{1}{t} \frac{dv}{dx}\right) = -\frac{1}{t^{2}}\frac{dv}{dx} + \frac{1}{t} \frac{d}{dt}\left(\frac{dv}{dx}\right)$$

Since $\frac{d}{dt}\left(\frac{dv}{dx}\right) = \frac{d^{2v}}{dx^{2}} \cdot \frac{dx}{dt} = \frac{1}{t}\frac{d^{2v}}{dx^{2}}$:

$$\frac{d^{2y}}{dt^{2}} = \frac{1}{t^{2}}\left(\frac{d^{2v}}{dx^{2}} - \frac{dv}{dx}\right)$$
\end{proof}

\subsection{The Transformed Equation}

\begin{theorem}[Constant Coefficient Form]
The Euler equation $t^{2y}'' + aty' + by = 0$ becomes:
$$v'' + (a-1)v' + bv = 0$
where v(x) = y($e^{x}$) and x = \ln t$.
\end{theorem}

\begin{insight}
The characteristic polynomial of the transformed equation is identical to the characteristic equation from the direct method! This confirms the deep connection between the two approaches.
\end{insight}

\section{Higher-Order Euler Equations}

\begin{theorem}[General $n$-th Order]
For the $n$-th order Euler equation, assume $y = t^{r}$. The characteristic equation is:
$$\prod_{k=0}^{n-1}(r-k) + \sum_{j=1}^{n-1} a_{j} \prod_{k=0}^{j-1}(r-k) + a_{0} = 0$$
\end{theorem}

\begin{example}[Third-Order]
For $t^{3y}''' + at^{2y}'' + bty' + cy = 0$:

Assuming $y = t^{r}$:
- $y' = rt^{r-1}$
- $y'' = r(r-1)t^{r-2}$
- $y''' = r(r-1)(r-2)t^{r-3}$

Characteristic equation:
$$r(r-1)(r-2) + ar(r-1) + br + c = 0$$
\end{example}

\section{Important Special Cases}

\subsection{The Cauchy-Euler Equation}

\begin{definition}[Standard Cauchy-Euler]
The equation $t^{2y}'' + ty' + (\lambda^{2} t^{2} + \nu^{2})y = 0$ appears in many applications, especially in Bessel function theory.
\end{definition}

\subsection{Modified Euler Equations}

\begin{example}[Shifted Center]
For equations centered at $t = t_{0}$:
$$(t-t_{0})^{2} y'' + a(t-t_{0})y' + by = 0$$
Use the substitution $s = t - t_{0}$ first.
\end{example}

\section{Solution Strategies}

\begin{method}[Complete Solution Process]
\begin{enumerate}
    \item \textbf{Identify}: Check if powers match derivatives
    \item \textbf{Choose domain}: Usually $t > 0$
    \item \textbf{Select method}:
    \begin{itemize}
        \item Direct ($y = t^{r}$) for straightforward cases
        \item Transform ($x = \ln t$) for complex or higher-order
    \end{itemize}
    \item \textbf{Find characteristic equation}
    \item \textbf{Solve for roots}
    \item \textbf{Build solution} based on root types:
    \begin{itemize}
        \item Real distinct: $t^{r_{1}}, t^{r_{2}}, \ldots$
        \item Real repeated: $t^{r}, t^{r}\ln t, t^{r}(\ln t)^{2}, \ldots$
        \item Complex: $t^\alpha\cos(\beta\ln t), t^\alpha\sin(\beta\ln t)$
    \end{itemize}
\end{enumerate}
\end{method}

\section{Applications}

\begin{insight}
Euler equations arise naturally in:
\begin{enumerate}
    \item \textbf{Radial symmetry}: Solutions to Laplace's equation in polar/spherical coordinates
    \item \textbf{Self-similar solutions}: PDEs with scaling symmetry
    \item \textbf{Power-law media}: Heat conduction in materials with power-law properties
    \item \textbf{Financial mathematics}: Option pricing models
    \item \textbf{Fractal geometry}: Equations on self-similar domains
\end{enumerate}
\end{insight}

\section{Connection to Other Topics}

\subsection{Series Solutions}

\begin{theorem}[Frobenius Connection]
The point $t = 0 is a regular singular point of the Euler equation. The indicial equation from the Frobenius method is exactly our characteristic equation.
\end{theorem}

\subsection{Relationship to Constant Coefficients}

\begin{keypoint}
\begin{center}
\begin{tabular}{|l|l|}
\hline
\textbf{Constant Coefficient} & \textbf{Euler Equation} \\
\hline
y = $e^{rt}$ ansatz & y = t^{r} ansatz \\
\hline
Repeated root gives te^{rt} & Repeated root gives t^{r}\ln t \\
\hline
Complex roots give e^{\alpha t}\sin(\beta t) & Complex roots give t^\alpha\sin(\beta\ln t)$ \\
\hline
Linear time scale & Logarithmic time scale \\
\hline
\end{tabular}
\end{center}
\end{keypoint}

\section{Common Pitfalls}

\begin{warning}
\begin{enumerate}
    \item \textbf{Domain}: Solutions differ for $t > 0$ and $t < 0$
    \item \textbf{Characteristic equation}: It's $r(r-1) + ar + b$, not $r^{2} + ar + b$
    \item \textbf{Complex roots}: Arguments are $\beta\ln t$, not $\beta t$
    \item \textbf{At $t = 0$}: This is a singular point; solutions may not extend through it
    \item \textbf{Initial conditions}: Often given at $t = 1$ to avoid the singularity
\end{enumerate}
\end{warning}

\begin{examtip}
Prof. Ditkowski often combines Euler equations with:
\begin{itemize}
    \item Initial value problems at $t = 1$
    \item Boundary value problems on $[1, e]$
    \item Questions about solution behavior as $t \to 0^+$ or $t \to \infty$
    \item Transformation to constant coefficients
\end{itemize}
\end{examtip}

\section{Summary Table}

\begin{formula}
\begin{center}
\begin{tabular}{|c|c|c|}
\hline
\textbf{Equation} & \textbf{Characteristic Eq.} & \textbf{Solutions} \\
\hline
$t^{2y}'' + aty' + by = 0$ & $r^{2} + (a-1)r + b = 0$ & Based on roots \\
\hline
Distinct real $r_{1}, r_{2}$ & -- & $c_{1t}^{r_{1}} + c_{2t}^{r_{2}}$ \\
\hline
Repeated root $r$ & -- & $(c_{1} + c_{2}\ln t)t^{r}$ \\
\hline
Complex $\alpha \pm i\beta$ & -- & $t^\alpha[c_{1}\cos(\beta\ln t) + c_{2}\sin(\beta\ln t)]$ \\
\hline
\end{tabular}
\end{center}
\end{formula}

\end{document}